\documentclass{article}


\usepackage[latin1]{inputenc}
\usepackage[francais]{babel}
\usepackage{amsmath,amssymb,amsfonts}
\usepackage[T1]{fontenc}
\usepackage{enumerate}



\newtheorem{Theoreme}{Th{\'e}or{\`e}me}[section]
\newtheorem{Definition}{D{\'e}finition}[section]
\newtheorem{Prop}{Proposition}[section]
\newtheorem{Lemme}{Lemme}[section]

\begin{document}

\title{Correction du DM 1}
\maketitle


\begin{enumerate}[(a)]
 
\item
$$\frac{4+i}{5i-3} = \frac{(4+i)(-3-5i)}{(5i-3)(-3-5i)} 
= \frac{-12 -20i -3i+5}{25+9} = \frac{-7}{36} -i\frac{23}{36}$$

\vspace{4mm}

\item
$$|z| = \sqrt{(\frac{\sqrt{10}}{2})^2 + (\frac{\sqrt{10}}{2})^2}
= \sqrt{\frac{5}{4} + \frac{5}{4} } = \frac{\sqrt{10} }{\sqrt{2} } $$
Dans l'�criture 
$z=re^{i\theta}$, on a $r= |z| = \frac{\sqrt{10}}{\sqrt{2}}$.
Ainsi 
$$e^{i\theta} = \frac{z \sqrt{2} }{\sqrt{10} }  = \frac{z}{r}
 \frac{ \sqrt{2} } { \sqrt{10} } (- \frac{ \sqrt{10} }{2} - i \frac{ \sqrt{10} }{2 } ) =
-\frac{ \sqrt{2} }{2} - i \frac{\sqrt{2}}{2} 
= -(\frac{\sqrt{2}}{2} + i \frac{\sqrt{2} }{2} ) $$
On sait que 
$\frac{\sqrt{2} }{2} + i \frac{\sqrt{2}}{2}$ 
correspond � 
$\cos(\frac{\pi}{4} ) + i \sin ( \frac{\pi}{4} ) = e^{i \frac{\pi}{4} } $, c'est 
� dire que son argument est $\frac{\pi}{4}$.
Pour trouver l'argument de 
$-(\frac{\sqrt{2}}{2} + i \frac{\sqrt{2} }{2} ) = -e^{i\frac{\pi}{4}}$ voici deux m�thodes :
\begin{itemize}
 \item 
Comme $u = -(\frac{\sqrt{2}}{2} + i \frac{\sqrt{2} }{2} )$ est l'oppos� de 
$v = (\frac{\sqrt{2}}{2} + i \frac{\sqrt{2} }{2} )$ , il correspond au sym�trique par rapport � l'origine, ce
 qui correspond � faire un demi-tour, c'est � dire une rotation d'angle $\pi$..
Pour obtenir l'argument de $u$, il faut donc rajouter $\pi$ � celui de $v$, qui lui vaut $\frac{\pi}{4}$.
Donc l'argument de 
$ -(\frac{\sqrt{2}}{2} + i \frac{\sqrt{2} }{2} )$ est $\frac{\pi}{4} + \pi = \frac{5\pi}{4} $. Donc 
$ -(\frac{\sqrt{2}}{2} + i \frac{\sqrt{2} }{2} ) = e^{i\frac{5\pi}{4} } $
(Si ce raisonnement ne vous para�t pas claire, essayer de faire un dessin dans le refaire en faisant un dessin dans le plan complexe ).
\item
On peut aussi le voire en disant que 
$-1 = e^{i\pi} $ (si cette formule ne vous para�t pas claire, r�fl�chissez-y...) .
D'o� , 
$-(\frac{\sqrt{2}}{2} + i \frac{\sqrt{2} }{2} ) = -e^{i\frac{\pi}{4} } 
=
e^{i\pi} e^{i \frac{\pi}{4} } = e^{i \frac{5\pi}{4} } $.
\end{itemize}

Finalement $z= \frac{\sqrt{10}}{\sqrt{2}} e^{i\frac{5\pi}{4} } $.


\item
On �crit $z$ sous forme cart�sienne : $z=x+iy$.
$$ \frac{z+1}{2z-1}= \frac{x+1+iy}{2x-1+i2y} = 
\frac{(x+1+iy)(2x-1 -i2y)}{(2x-1+i2y)(2x-1-i2y )}$$

$$= \frac{(x+1)(2x-1) + 2y^2 + i( y(2x-1) - 2y(x+1) ) }{(2x-1)^2 + 4y^2}$$
Pour que ce nombre soit imaginaire pur, il faut que sa partie r�elle soit nulle, c'est � 
dire que 
$(x+1)(2x-1) + 2y^2 = 0$ ce qui �quivaut � 
$$ \label{eq1} 2x^2 +x+2y^2-1=0$$
Pour la deuxi�me condition, 
$$\left | \frac{3z+i}{z-3} \right | = 3  \Leftrightarrow |3z+i|^2 = 3^2|z-3|^2 
\Leftrightarrow 
|3x + i(3y+1)|^2 = 9|x-3+iy|^2 
\Leftrightarrow 
9x^2 + (3y+1)^2 = 9(x-3)^2+9y^2 $$
$$\Leftrightarrow
9x^2+9y^2+6y+1 = 9x^2 -54x +81 + 9y^2 
\Leftrightarrow
6y + 1  = -54 x +81
\Leftrightarrow 
y =-9x +\frac{40}{3}
$$
Si on met en commun les deux calculs qu'on vient de faire, $z= x+iy$ v�rifie 
$\frac{z+1}{2z-1}$ est imaginaire pur et 
$\left |  \frac{3z+1}{z-3}  \right | $ si et seulement si $z$ v�rifie le syst�me d'�quations :


$$ \left \{  
\begin{array}{ll}
 2x^2+x-1+2y^2 \\
y = -9x + \frac{40}{3}
\end{array}
\right. $$
En \textit{ rempla�ant la deuxi�me �quation dans la premi�re} 
on obtient l'�quation 
$$2x^2+x-1+2(-9x + \frac{40}{3})^2 = 0
=2x^2+x-1+162x^2 -480x +\frac{3200}{9 } =
164x^2 - 479x + \frac{3191}{9}$$
Le discriminant de ce trinome est $\Delta = \frac{-28327}{9} $, le syst�me n'a 
donc pas de solutions, et l'ensemble des $z$ qui v�rifient les conditions de l'�nonc� est donc vide.

\item
On peut faire directement le calcul en posant $u=x+iy$ et 
$v=x'+iy'$. \\
Une autre m�thode (moins calculatoire est : 
$$|u+v|^2 + |u-v|^2 = (u+v) \overline{(u+v)} + (u-v) \overline{(u-v)}
= (u+v)(\bar{u} + \bar{v} ) + (u-v) (\bar{u} - \bar{v} ) $$
$$= u \bar{u} + u \bar{v} + v\bar{u} + u\bar{u} + u \bar{u} -u\bar{v} - v\bar{u} + v\bar{v} 
=2(|u|^2 + |v|^2)$$

\item
 On utilise la formule d'Euler :
$$\sin(\theta)\cos^2(\theta) = \left( \frac{e^{i\theta} - e^{-i\theta}}{2i} \right )
 \left (\frac{e^{i\theta} + e^{-i\theta }}{2}  \right )^2
=
\left ( \frac{e^{i\theta} - e^{-i \theta }}{2i} \right ) \left ( 
\frac{e^{2i\theta} +2 + e^{-2i\theta } }{4}  \right ) $$
$$
=
\frac{e^{3i\theta} + 2 e^{i\theta} + e^{-i\theta } - e^{i\theta} - 2e^{-i\theta} - e^{-3i\theta} }{8i}
=
\frac{1}{4} \left ( \frac{ e^{3i\theta } - e^{-3i\theta } }{2i} + \frac{e^{i \theta} - e^{-i\theta }}{2i}  \right ) =
\frac{1}{4} ( \sin(3\theta) + \sin(\theta) )$$

Donc 
$\displaystyle \int_0^{\frac{\pi}{2}} \sin(\theta) \cos^2(\theta) d\theta 
= \frac{1}{4} \left( \int_0^{\frac{\pi}{2}} \sin (3\theta ) d\theta 
+ \int_0^{\frac{\pi}{2}} \sin(\theta) d\theta \right) $.
On se convainc qu'une primitive de 
$\sin (3 \theta)$ est 
$-\frac{\cos(3 \theta ) }{3}$, donc
$$\int_0^{\frac{\pi}{2}} \sin(\theta) \cos^2(\theta) d\theta =
\frac{1}{4} \left[ -\frac{\cos(3\theta)}{3}  - \cos(\theta) \right]_0^{\frac{\pi}{2}}
=\frac{1}{4} (0+0-(\frac{-1}{3}) -(-1) ) = \frac{1}{4} \frac{4}{3} = \frac{1}{3} $$

\item

Il est facile de voir que 
$z_0 = - \sqrt[3]{2} $ est une solution de l'�quation.
A partir de l� ,
$$z^3 = -2 = z_0^3 \Leftrightarrow (\frac{z}{z_0})^3=1
\Leftrightarrow \frac{z}{z_0} \in \{ 1, e^{\frac{2i\pi}{3} } , e^{\frac{4i\pi}{3} } \}$$
Il y a donc trois solutions : 

$$z = -\sqrt[3]{2}, \ \  z=-\sqrt[3]{2} e^{\frac{2i\pi}{3}} \ , \ \ 
z=-\sqrt[3]{2}  e^{ \frac{4i\pi}{3}} $$



\end{enumerate}














\end{document}
