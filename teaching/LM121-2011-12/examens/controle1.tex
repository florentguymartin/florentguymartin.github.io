\documentclass{article}


\usepackage[latin1]{inputenc}
\usepackage[francais]{babel}
\usepackage{amsmath,amssymb}
\usepackage[T1]{fontenc}

\date{}
\begin{document}
\renewcommand{\labelitemi}{$\circ$}
\title{Controle 1 LM 121}
\maketitle
\begin{enumerate}


\item
Placer dans le plan complexe les points $M_1, \ M_2 , \ 
M_3, \ M_4$ dont les affixes sont :\\
$$\begin{array}{lcr}
 z_{M_1}  \ = \ 3+2i \\
z_{M_2} \ =  \ e^{\frac{2i\pi}{3}} \\
z_{M_3} \ = \ \frac{-3+i}{1+i} \\
z_{M_4} \ = \ 2 e^{ \frac{-4i\pi}{3} }
\end{array}$$

\item 
D�terminer les $z\in \mathbb{C}$ tels que 
$z^4=-16$ , et repr�senter les images de ces complexes dans le plan.
\item
D�crire l'ensemble  des $z\in \mathbb{C}$ (avec $z\neq - i$) tels que 
$\frac{2z+1}{z+i} \in \mathbb{R}$.

\item
Lin�ariser 
$\sin ^3 ( \theta) $ .

\item
Soit $z\in \mathbb{C}$ tel que $1+z^2+z^4+z^6+z^8+z^{10}=0$. Que vaut $|z|$ ?


\end{enumerate}



\end{document}




\end{document}
