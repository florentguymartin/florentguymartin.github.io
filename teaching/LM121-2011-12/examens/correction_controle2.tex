\documentclass{article}

\usepackage[all]{xy}
\usepackage[latin1]{inputenc}
\usepackage[francais]{babel}
\usepackage{amsmath,amssymb,amsfonts}
\usepackage{dsfont}
\usepackage[T1]{fontenc}


\newtheorem{rem}{Remarque}
\newtheorem{defi}{D{\'e}finition}[section]
\newtheorem{prop}{Proposition}[section]
\newtheorem{lemme}{Lemme}[section]
\newtheorem{cor}{Corollaire}[section]

\title{Correction contr�le continu 2 LM121 PCME 14.2}
\date{}
\begin{document}

\maketitle
\begin{enumerate}



 \item 
En coordonn�es complexes on a (en utilisant le fait que $e^{i\frac{\pi}{2} } =i$) :\\
$$r_{A, \frac{\pi}{2} } = z \mapsto i(z-1-i) +1+i = iz +2$$
$$r_{B , \frac{\pi}{2}} = z \mapsto i(z-2+i) +2 -i = iz -3i +1 $$
Ainsi 
$(r_{B, \frac{\pi}{2}} \circ r_{A, \frac{\pi}{2} }  )(z) = i(iz+2)-3i+1 = -z -i+1$
 et donc $ (r_{B, \frac{\pi}{2}} \circ r_{A, \frac{\pi}{2}} ) (z_C) =
-(-1-i)-i+1 = 2$ . Donc le point recherch� a pour coordonn�es $(2,0)$.

\item 
$u, v$ et $w$ sont libres si et seulement si leur d�terminant est non nul.
Or det$(u,v,w) = 
\begin{vmatrix}
 1 & 1 & x \\
x & 2 & 1 \\
1 & 3 & -1                 
                 \end{vmatrix}= $ 
$1 \begin{vmatrix}
    2 & 1 \\ 3 & -1 
   \end{vmatrix}
-x \begin{vmatrix}
    1 & x \\ 3 & -1 
   \end{vmatrix}
 + 1\begin{vmatrix}
    1 & x \\ 2 & 1
   \end{vmatrix}
$ (en d�veloppant par rapport � la premi�re colonne) \\
$= -5  -x(-1-3x) + 1-2x =  3x^2-x-4$.
Le discrimant de ce polyn�me est $\Delta = (-1)^2 -4 (3)(-4) = 49 =7^2$. 
Les racines de ce polyn�me sont 
donc 
$\frac{1 \pm 7}{6} = \frac{4}{3}$ ou $-1$.
Les vecteurs sont donc libres pour $x \neq \frac{4}{3} , -1$.

\item 
Un vecteur normal � 
$\mathcal{P}$ est 
$\vec{n} = u \wedge v = 
\begin{pmatrix}
1 \\ 1 \\ -1                         
\end{pmatrix}
\wedge
\begin{pmatrix}
 2 \\ -3 \\ -1 
\end{pmatrix}$

$= 
\begin{pmatrix}
 -4 \\ -1 \\ -5
\end{pmatrix}$.
$\mathcal{P}$ a donc une �quation de la forme 
$4x+y+5z =d$ pour un $d\in \mathbb{R}$. On le d�termine en rempla�ant dans cette �quation $(x,y,z)$ par les coordonn�es de 
$A$, ce qui donne :
$4+2-5=d=1$.
$\mathcal{P}$ a donc pour �quation 
$4x +y +5z =1$. \\
$\mathcal{D}$ admet la parametrisation suivante : 
$\left\{ \begin{matrix}
          x & = & 1+t \\
y & = & 1 +2t \\
z & = & 1 +t
         \end{matrix}
\right.$\\
Un point de $\mathcal{D}$ correspondant au param�tre $t$ sera dans $\mathcal{P}$ si et seulement si 
$4x +y +5z =1 = 4(1+t) +(1+2t) +5(1+t) = 10 +11t$ ce qui �quivaut � 
$11t = -9$ soit $t=\frac{-9}{11}$.
Ainsi $\mathcal{P} \cap \mathcal{D} = \{ 
\begin{pmatrix}
 \frac{2}{11} \\
\frac{-7}{11} \\
\frac{2}{11}
 \end{pmatrix} \}$

\item

On pose $x = \begin{pmatrix}
              x_1 \\ x_2 \\ x_3
             \end{pmatrix}$.
\begin{enumerate}
\item 
L'�quation est 
$\begin{pmatrix}
  1\\1\\-1
 \end{pmatrix}
 \wedge 
\begin{pmatrix}
 x_1\\x_2\\x_3
\end{pmatrix}
 = 
\begin{pmatrix}
 x_2 +x_3 \\-x_1 -x_3 \\-x_1 +x_2
\end{pmatrix}
 = \begin{pmatrix} 
    1\\2\\3
   \end{pmatrix}
$
qui �quivaut au syst�me 
\\
$\begin{matrix}
      & x_2   & +x_3 & =1 \\
-x_1 &  & -x_3   & =2 \\
-x_1  & +x_2 & & =3
 \end{matrix}$
\\
$\Leftrightarrow$
\\

$\begin{matrix}
 &  & x_2  &+x_3 & = 1 \\
L_2 \leftarrow -L_2 &x_1  & & +x_3 & =-2 \\
& -x_1  &+x_2 &  &=3
 \end{matrix} $
\\
$\Leftrightarrow$ 
\\


$\begin{matrix} 
 L_1 \leftrightarrow  L_2 & x_1 & & +x_3 & =-2 \\
             &      &x_2  &+x_3 & = 1 \\
             & -x_1 &+x_2   &    & =3
 \end{matrix}
$
\\
$\Leftrightarrow$
\\

$\begin{matrix} 
                               & x_1 & & +x_3 & =-2 \\
                                 &      &x_2  &+x_3 & = 1 \\
 L_3 \leftarrow L_3 + L_1  &  &x_2   &+x_3    & =1
 \end{matrix}
$ \\
$\Leftrightarrow$
\\

$\begin{matrix} 
                               & x_1 & & +x_3 & =-2 \\
                                 &      &x_2  &+x_3 & = 1 \\
 \end{matrix}
$ \\
On reconna�t l'�quation d'une droite, en prenant comme param�tre $t=x_3$ par exemple, on obtient \\
$\left\{ 
\begin{matrix}
x_1& = &-2-t \\
x_2&=&1-t \\
x_3&=&t
\end{matrix} \right.$



\item 
L'�quation est 
$\begin{pmatrix} 
  1 \\ 1 \\ -1 
 \end{pmatrix}
\wedge 
\begin{pmatrix}
 x_1 \\x_2 \\ x_3
\end{pmatrix}
= 
\begin{pmatrix}
 x_2+x_3 \\ -x_1 - x_3 \\ x_2-x_1 
\end{pmatrix}
 = \begin{pmatrix}
    3\\2\\1
   \end{pmatrix}$
\\
$\Leftrightarrow$ \\

$\begin{matrix}
   & x_2  &+ x_3 & = 3 \\
 -x_1  & & -x_3 & =2 \\
-x_1  & +x_2 &  &=1
 \end{matrix} $
\\
$\Leftrightarrow$ 
\\
$\begin{matrix}
 &  & x_2  & +x_3 & = 3 \\
L_2 \leftarrow -L_2 &x_1  & & +x_3 & =-2 \\
& -x_1  &+x_2 &  &=1
 \end{matrix} $
\\
$\Leftrightarrow$ 
\\


$\begin{matrix} 
 L_1 \leftrightarrow  L_2 & x_1 & & +x_3 & =-2 \\
             &      &x_2  & +x_3 & = 3 \\
             & -x_1 &+x_2   &    & =1
 \end{matrix}
$
\\
$\Leftrightarrow$
\\

$\begin{matrix} 
                               & x_1 & & +x_3 & =-2 \\
                                 &      &x_2  &+x_3 & = 3 \\
 L_3 \leftarrow L_3 + L_1  &  &x_2   &+x_3    & =-1
 \end{matrix}
$
\\
Les lignes 2 et 3 sont incompatibles, donc il n'y a pas de solution.
\end{enumerate}



 


\end{enumerate}









\end{document}
