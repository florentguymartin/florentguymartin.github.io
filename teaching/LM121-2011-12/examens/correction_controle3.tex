\documentclass{article}


\usepackage[latin1]{inputenc}
\usepackage[francais]{babel}
\usepackage{amsmath,amssymb,amsfonts}
\usepackage[T1]{fontenc}
\usepackage{enumerate}



\newtheorem{Theoreme}{Th{\'e}or{\`e}me}[section]
\newtheorem{Definition}{D{\'e}finition}[section]
\newtheorem{Prop}{Proposition}[section]
\newtheorem{Lemme}{Lemme}[section]
\date{}
\begin{document}

\title{Correction contr�le continu 3 LM 121 pcme 14.2}
\maketitle

\begin{enumerate}

\item 

En multipliant les deux matrices on tombe sur l'�galit� \\
$\begin{pmatrix} a+2b&3a-4b \\-2 &14 \end{pmatrix} = \begin{pmatrix}11 &3\\-2&14\end{pmatrix}$\\
ce qui �quivaut donc au syst�me : \\
$\begin{array}{rl}
 a+2b=&11 \\
3a-4b=&3 
\end{array}
$\\
On r�sout ce syst�me et on trouve qu'il existe une unique solution : $a=5$, $b=3$.

\item
On trouve 
$Det(A) = 6 \neq 0$ donc $A$ est inversible. On calcule son inverse avec la m�thode du pivot de Gauss : 
$$
\begin{pmatrix}
 3 & 4 & 4 \\
1 & 2 & 1 \\
0 & -2 & 4 
\end{pmatrix}
\left| 
\begin{pmatrix} 
 1 & 0 & 0 \\
0 & 1 & 0 \\
 0 & 0 & 1 
\end{pmatrix}
\right. 
\xrightarrow[]{L_1 \leftrightarrow L_2 }
\begin{pmatrix}
 1 & 2 & 1 \\
3 & 4 & 4 \\
0 & -2 & 4 
\end{pmatrix}
\left| 
\begin{pmatrix} 
 0 & 1 & 0 \\
1 & 0 & 0 \\
 0 & 0 & 1 
\end{pmatrix}
\right. 
\xrightarrow[]{L_2 \leftarrow L_2-3L_1 }$$

$$
\begin{pmatrix}
 1 & 2 & 1 \\
0 & -2 & 1 \\
0 & -2 & 4 
\end{pmatrix}
\left| 
\begin{pmatrix} 
 0 & 1 & 0 \\
1 & -3 & 0 \\
 0 & 0 & 1 
\end{pmatrix}
\right. 
\xrightarrow[]{L_2 \leftarrow \frac{L_2}{-2} }
\begin{pmatrix}
 1 & 2 & 1 \\
0 & 1 & \frac{-1}{2} \\
0 & -2 & 4 
\end{pmatrix}
\left| 
\begin{pmatrix} 
 0 & 1 & 0 \\
\frac{-1}{2} & \frac{3}{2} & 0 \\
 0 & 0 & 1 
\end{pmatrix}
\right. 
\xrightarrow[L_3 \leftarrow L_3 +2L_2]{L_1 \leftarrow L_1 - 2 L_2 }
$$

$$
\begin{pmatrix}
 1 & 0 & 2 \\
0 & 1 & \frac{-1}{2} \\
0 & 0 & 3 
\end{pmatrix}
\left| 
\begin{pmatrix} 
 1 & -2 & 0 \\
\frac{-1}{2} & \frac{3}{2} & 0 \\
 -1 & 3 & 1 
\end{pmatrix}
\right. 
\xrightarrow[]{L_3 \leftarrow \frac{L_3}{3} }
\begin{pmatrix}
 1 & 0 & 2 \\
0 & 1 & \frac{-1}{2} \\
0 & 0 & 1 
\end{pmatrix}
\left| 
\begin{pmatrix} 
 1 & -2 & 0 \\
\frac{-1}{2} & \frac{3}{2} & 0 \\
 \frac{-1}{3} & 1 & \frac{1}{3} 
\end{pmatrix}
\right. 
\xrightarrow[L_2 \leftarrow L_2+ \frac{1}{2}L_3]{L_1 \leftarrow L_1-2L_3 }
$$

$
\begin{pmatrix}
 1 & 0 & 0 \\
0 & 1 & 0\\
0 & 0 & 1 
\end{pmatrix}
\left| 
\begin{pmatrix} 
 \frac{5}{3} & -4 & \frac{-2}{3} \\
\frac{-2}{3} & 2 & \frac{1}{6} \\
 \frac{-1}{3} & 1 & \frac{1}{3} 
\end{pmatrix}
\right. 
$\\
On v�rifie bien que 
$\begin{pmatrix} 
 \frac{5}{3} & -4 & \frac{-2}{3} \\
\frac{-2}{3} & 2 & \frac{1}{6} \\
 \frac{-1}{3} & 1 & \frac{1}{3} 
\end{pmatrix}
\begin{pmatrix}
 3 & 4 & 4 \\
1 & 2 & 1 \\
0 & -2 & 4 
\end{pmatrix}
=
\begin{pmatrix}
 1 & 0 & 0 \\
0 & 1 & 0\\
0 & 0 & 1 
\end{pmatrix}$ \\
donc $A^{-1} = 
\begin{pmatrix} 
 \frac{5}{3} & -4 & \frac{-2}{3} \\
\frac{-2}{3} & 2 & \frac{1}{6} \\
 \frac{-1}{3} & 1 & \frac{1}{3} 
\end{pmatrix}$

\item
On v�rifie que les points ne sont pas align�s. $(ABC)$ est donc le plan passant par 
$A$ et dirig� par les vecteurs libres 
$\vec{AB}= \begin{pmatrix} 0 \\1 \\ 2 \end{pmatrix}$ et 
$ \vec{AC} = \begin{pmatrix} -1\\ 0 \\ 1\end{pmatrix}$.\\
Leur produit vectoriel est \\
$\begin{pmatrix} 0 \\1 \\ 2 \end{pmatrix} \wedge 
\begin{pmatrix} -1\\ 0 \\ 1\end{pmatrix}=
\begin{pmatrix} 1 \\ -2 \\ 1.\end{pmatrix}$ qui est donc un vecteur normal � $(ABC)$.
Le plan a donc une �quation de la forme
$x -2y +z =d$ . Pour trouver $d$ on remplace cette �quation avec les coordonn�es de $A$ :\\
$1-2+1=d=0$\\
Donc $x -2y +z =0$ est une �quation de $(ABC)$.

\item
Il y a �norm�ment de possiblit�s. Par exemple 
$B = \begin{pmatrix}
      0&0&1 \\
0&0&0\\0&0&0
\end{pmatrix}
$ marche.

\end{enumerate}














\end{document}
