\documentclass{article}

\usepackage[all]{xy}
\usepackage[latin1]{inputenc}
\usepackage[francais]{babel}
\usepackage{amsmath,amssymb,amsfonts}
\usepackage{dsfont}
\usepackage[T1]{fontenc}


\newtheorem{rem}{Remarque}
\newtheorem{defi}{D{\'e}finition}[section]
\newtheorem{prop}{Proposition}[section]
\newtheorem{lemme}{Lemme}[section]
\newtheorem{cor}{Corollaire}[section]

\title{Contr�le Continu 2 \ LM121 \ PCME 14.2}
\date{}
\begin{document}

\maketitle

\begin{enumerate}

 \item
Soit $A=(1,1)$ , $B=(2,-1)$ et $C=(-1,-1)$ 3 points du plan.
On consid�re $r_{A, \frac{\pi}{2} }$ la rotation de centre $A$ et d'angle $\frac{\pi}{2}$, et
$r_{B, \frac{\pi}{2} }$ la rotation de centre $B$ et d'angle $\frac{\pi}{2}$.
D�terminer $ (r_{B, \frac{\pi}{2}} \circ r_{A, \frac{\pi}{2} } ) (C)$ (� savoir, donner ses coordonn�es cart�siennes).


\item
Pour quelle(s) valeur(s) de $x\in \mathbb{R}$ les vecteurs
$u=\begin{pmatrix}
  1 \\ x \\ 1
 \end{pmatrix}$
, 
$v = \begin{pmatrix}
      1 \\ 2 \\ 3
     \end{pmatrix}
$
et 
$w= \begin{pmatrix}
     x\\
1\\
-1 
    \end{pmatrix}$
sont-ils libres?

\item

Soit 
$\mathcal{P}$ le plan passant par 
$A =(1,2,-1)$ et engendr� par les vecteurs 
$u = \begin{pmatrix}
      1 \\1 \\ -1
     \end{pmatrix} $
 et $v= \begin{pmatrix}
         2 \\ -3 \\ -1
        \end{pmatrix}$.
Soit $\mathcal{D}$ la droite passant par 
$B=(1,1,1)$ et de vecteur directeur $w=\begin{pmatrix}
                                        1 \\ 2 \\ 1
                                       \end{pmatrix}$.
D�terminer $\mathcal{P} \cap \mathcal{D}$.




\item



Soit $a$ et $b$ 2 vecteurs non nuls de $\mathbb{R}^3$. On s'int�resse � l'�quation $a \wedge x = b$.

\begin{enumerate}


\item
Si 
$a= \begin{pmatrix} 
     1 \\ 1 \\ -1
    \end{pmatrix}$
 et $b= \begin{pmatrix} 1 \\ 2 \\ 3 \end{pmatrix}$, montrer que l'ensemble des $x\in \mathbb{R}^3$ tels que $a\wedge x = b$ est une droite dont on donnera une param�trisation.
\item
Si $a=\begin{pmatrix}
       1 \\ 1 \\-1
      \end{pmatrix} $ 
et $b= \begin{pmatrix} 3 \\2 \\ 1 \end{pmatrix} $, d�crire l'ensemble des $x\in \mathbb{R}^3$ tels que $a\wedge x =b$.
\end{enumerate}




\end{enumerate}








\end{document}
