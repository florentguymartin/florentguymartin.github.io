\documentclass{article}


\usepackage[latin1]{inputenc}
\usepackage[francais]{babel}
\usepackage{amsmath,amssymb,amsfonts}
\usepackage[T1]{fontenc}




\newtheorem{Theoreme}{Th{\'e}or{\`e}me}[section]
\newtheorem{Definition}{D{\'e}finition}[section]
\newtheorem{Prop}{Proposition}[section]
\newtheorem{Lemme}{Lemme}[section]
\date{}
\begin{document}





\begin{enumerate}
\item
Calculer Det $\begin{pmatrix} 
1 & 5 & 2 \\
-1 & 1 & -2 \\
-1 & 2 &1
\end{pmatrix}$. \\

Les vecteurs $(1,-1,-1) , (5,1,2)$ et $(2,-2,1)$ sont-ils libres?

\item
Trouver $u,v$ et $w$ tels que
$\begin{pmatrix}
1 & -1 & 1 \\
2 & 1 & 3 \\
1 & -1 & 2\end{pmatrix}
\begin{pmatrix} u & 3 & 0 \\
v & -1 & -1 \\
w & 1 & 2 \end{pmatrix}
=
\begin{pmatrix} 
-3 & 5 & 3 \\
-2 & 8 & 5 \\
-5 & 6 & 5 \end{pmatrix}
$
\item


Soit $a$ et $b$ 2 vecteurs non nuls de $\mathbb{R}^3$. On s'int�resse � l'�quation $a \wedge x = b$.

\begin{enumerate}


\item
Si $a=(1,1,-1)$ et $b=(1,2,3)$, montrer que l'ensemble des $x\in \mathbb{R}^3$ tels que $a\wedge x = b$ est une droite dont on donnera une param�trisation.
\item
Si $a=(1,1,-1)$ et $b=(3,2,1)$, d�crire l'ensemble des $x\in \mathbb{R}^3$ tels que $a\wedge x =b$.


\item
On suppose jusqu'� la fin de la question $a$ et $b$ quelconques, mais non nuls.
\begin{enumerate}
\item
Montrer que si $a.b \neq 0$, l'�quation $a\wedge x =b$ n'a pas de solution. 


 


\item[ \ ] 
\textsc{On suppose dor�navant $a.b=0$.} 
\item
Montrer que $a\wedge (b\wedge a) = \|a\|^2b$. (On rappelle la formule du double produit vectoriel : $(u\wedge v)\wedge w = (u.w)v - (v.w)u$ )
\item
En d�duire un $x_0 \in \mathbb{R}^3$ tel que $a\wedge x_0 =b$.

\item
Si $a\wedge x = b$ que peut-on dire de $a\wedge(x-x_0)$?

\item 
En d�duire une description g�om�trique de l'ensemble des $x$ tels que 
$a\wedge x =b$.

\end{enumerate}
\end{enumerate}

\end{enumerate}







\end{document}
