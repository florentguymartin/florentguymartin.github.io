\documentclass{article}


\usepackage[latin1]{inputenc}
\usepackage[francais]{babel}
\usepackage{amsmath,amssymb,amsfonts}
\usepackage[T1]{fontenc}




\newtheorem{Theoreme}{Th{\'e}or{\`e}me}[section]
\newtheorem{Definition}{D{\'e}finition}[section]
\newtheorem{Prop}{Proposition}[section]
\newtheorem{Lemme}{Lemme}[section]

\begin{document}

\title{Correction du devoir maison}
\maketitle
\renewcommand{\labelitemi}{$\circ$}
\section*{Exercice 1}
Au voisinage de $0$ on a :
$$\frac{ \cos(x) }{1+x+x^2} = \frac{1-\frac{x^2}{2} + \frac{x^4}{24} +
  \circ(x^4)}{1+x+x^2} $$ \\
$=(1-\frac{x^2}{2}+\frac{x^4}{24} +\circ(x^4) \
)(1-(x+x^2)+(x+x^2)^2-(x+x^2)^3+(x+x^2)^4 +\circ (x^4) \ ) $ \\
 $= ( 1 - \frac{x^{2}}{2} + \frac{x}{24} )(1-x-x^2+x^2+2x^3+x^4-x^3-3x^4+x^4 +
\circ (x^4)
\ )$
\\
$=(1-\frac{x^2}{2}+\frac{x^4}{24})(1-x+x^3-x^4 +\circ (x^4) \ )$\\
$=1-x+x^3-x^4-\frac{x^2}{2}+\frac{x^3}{2} +\frac{x^4}{24} + \circ(x^4)$\\
$=1-x-\frac{x^2}{2} +\frac{3x^3}{2} - \frac{23x^4}{24} + \circ(x^4)$\\


\section*{Exercice 2}
On fait un DL de $f$ pour $x \neq 0$ au voisinage de $0$ :
$$f(x) = \frac{x}{e^x-1}=\frac{x}{x+\frac{x^2}{2}+\circ(x^2)}$$
$$ = \frac{x}{x(1+\frac{x}{2} + \circ(x) \ )} = \frac{1}{1+\frac{x}{2} +
  \circ(x) }= 1-\frac{x}{2} + \circ(x)$$
Comme par ailleurs l'{\'e}nonc{\'e} donne $f(0)=1$ le DL est valable au
voisinage de $0$. Comme c'est un DL {\`a} l'ordre $1$, on en d{\'e}duit que
$f$ est d{\'e}rivable en $0$, de d{\'e}riv{\'e}e $f'(0)=\frac{-1}{2}$.


\section*{Exercice 3}
On fait le changement de variable $u=\sqrt(x)$, soit $x=u^2$ qui donne
$dx =2u.du$:
$$I=\int_5^6\frac{x\sqrt{x}+1}{x+1}dx =
\int_{\sqrt{5}}^{\sqrt{6}}\frac{u^3+1}{u+1}2udu
=2\int_{\sqrt{5}}^{\sqrt{6}}\frac{u^4+u}{u^2+1}du
$$
Puis on fait la division euclidienne de $X^4+X$ par $X^2+1$ qui
donne\\
$X^4+X=(X^2+1)(X^2-1)+X+1$.\\
D'o{\`u} $\frac{X^4+X}{X^2+1} = X^2-1+\frac{X}{X^2+1}+
\frac{1}{X^2+1}$. Donc 
$$I=2\int_{\sqrt{5}}^{\sqrt{6}} u^2-1+\frac{u}{u^2+1}+\frac{1}{u^2+1}
\ du$$
En utilisant le fait qu'une primitive de $\frac{2u}{u^2+1}$ est
$\ln(u�+1)$ on a :
$$I=[\frac{2u^3}{3}-2u+\ln(u^2+1) +
2\arctan(u)]^{\sqrt{6}}_{\sqrt{5}}$$
$$=2\sqrt{6} -\frac{4\sqrt{5}}{3} + \ln(\frac{7}{6}) +
2\arctan(\sqrt{6})-2\arctan(\sqrt{5})$$

\section*{Exercice 4}
Une racine {\'e}vidente de $X^3-2X^2+X-2$ est $2$ , qui permet de
factoriser :$X^3-2X^2+X-2=(X-2)(X^2+1)$. On a donc 
$$\frac{X+2}{X^3-2X^2+X-2}=\frac{aX+b}{X^2+1}+\frac{c}{X-2}$$
En multipliant par $(X-2)$ et en {\'e}valuant en $2$ on obtient
$c=\frac{4}{5}$. En faisant tendre $X$ vers $\infty$ on trouve un
{\'e}quivalent $\frac{1}{X^2}$ {\`a} gauche, et {\`a} droite on a deux termes
"dominants" :$\frac{a}{X}+\frac{c}{X}$. Ce dernier terme doit donc
{\^e}tre nul, soit $a=-c=\frac{-4}{5}$. Pour finir on {\'e}value en $0$, qui
donne $\frac{2}{-2} = -1 = b-\frac{c}{2}$ ,
i.e. $b=\frac{-3}{5}$. Finalement
$$\frac{X+2}{X^3-2X^2+X-2}=\frac{-4X-3}{5(X^2+1)}+\frac{4}{5(X-2)}$$

\section*{Exercice 5}
On distingue deux cas :
\begin{itemize}
\item Si $\forall x \in \mathbb{R}_+$ on a $f(x)\leq 0$ , alors $0$
  est le maximum, et est atteint en $0$ (car $f(0)=0$ par hypoth{\`e}se).
\item Sinon, il existe un $x_0 \in \mathbb{R}_+$ tel que
  $f(x_0)>0$. Alors par d{\'e}finition de la limite, on sait qu'il existe
  un $A \geq 0$ tel que pour tout $x\geq A  \ , \  f(x) <f(x_0)$ (on a pris
  $\epsilon = f(x_0)>0$). Alors sur l'intervalle ferm{\'e} born{\'e} $[0,A]$,
  la fonction $f$ {\'e}tant continue y atteint ses bornes (th{\'e}or{\`e}me de
  Heine, ou Weierstrass). Donc il existe $x_1 \in [0,A]$ tel que pour
  tout $x\in[0,A] \ f(x)\leq f(x_1)$. De plus on a que $x_0 \in
  [0,A]$, car sinon on aurait $x_0 \geq A $ donc $f(x_0)<f(x_0)$ qui
  est impossible. Et donc $f(x_0)\leq f(x_1)$. Ainsi pour $x\in[0,A] $
  on a $f(x) \leq f(x_1)$ et de m{\^e}me pour $x\geq A$ on a $f(x) \leq
  f(x_0) \leq f(x_1)$. Donc la fonction $f$ a bien un maximum, atteint
  en $x_1$.
\end{itemize}
\section*{Exercice 6}
Les solutions de l'{\'e}quation homog{\`e}ne sont de la forme $Ke^x$, et on a
que si on trouve une solution particuli{\`e}re $y_1$ de
$y'-y=e^x$, et une solution $y_2$ de $y'-y=e^{2x}$, alors le caract{\`e}re
lin{\'e}aire de l'{\'e}quation fait que $y_1+y_2$ est solution de
$y'-y=e^x+e^{2x}$.\\
Pour trouver $y_1$ sous la forme $\lambda(x)e^x$ , la variation de la
constante nous donne $\lambda'(x) = 1$ donc par exemple $\lambda(x) =
x$ \ soit \  $y_1(x) = xe^x$ . Pour $y_2$ on peut appliquer la variation de
la constante, ou trouver directement "{\`a} l'oeil" $y_2(x) =
e^{2x}$. Donc $xe^x+e^{2x}$ est une solution de notre {\'e}quation de d{\'e}part. \\
Finalement les solutions sont de la forme $xe^x+e^{2x} + Ke^x$ avec
$K\in \mathbb{R}$, et la condition $y(0)=2$ nous donne $1+K=2$ donc $K=1$.







\bibliographystyle{alpha}
\bibliography{bibli}





\end{document}
