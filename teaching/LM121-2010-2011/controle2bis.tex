\documentclass{article}


\usepackage[latin1]{inputenc}
\usepackage[francais]{babel}
\usepackage{amsmath,amssymb,amsfonts}
\usepackage[T1]{fontenc}




\newtheorem{Theoreme}{Th{\'e}or{\`e}me}[section]
\newtheorem{Definition}{D{\'e}finition}[section]
\newtheorem{Prop}{Proposition}[section]
\newtheorem{Lemme}{Lemme}[section]
\date{}
\begin{document}

{\Large Controle 2 LM121}

\begin{enumerate}

\item
Soit $\mathcal{P}$ le plan donn� par la param�trisation\\
$\begin{array}{rclll}
x & = &3s& +2t & +1 \\
y&=&s&-t&\\
z&=&2s&+t&-1\end{array}$\\
et $\mathcal{D}$ la droite donn�e par \\
$\begin{array}{rrrcl}
x&+y&+z&=&1\\
x&-3y&+2z&=&3 \end{array}$\\
D�terminer les points d'intersections de $\mathcal{P}$ et $\mathcal{D}$.

\item
Soit $A=(1,1,1)\ , \ B=(1,2,3)$ et $C=(0,1,2)$. Ces points sont-ils align�s? Si non, donner une �quation du plan $(ABC)$.

\item
Pour quelle(s) valeur(s) de $a$ les vecteurs $(1,3,5) \ , \ (-2,1,4)$ et $(3,a,2)$ sont-ils libres?
\item
Soit $A =\begin{pmatrix}
1&2 \\ 3&-2\end{pmatrix}$
\begin{enumerate}
\item
Trouver $\lambda \in \mathbb{R}$ tel que $A^2+A+\lambda I_2=0$. (On rappelle que 
$I_2 = \begin{pmatrix} 1 & 0 \\ 0 & 1 \end{pmatrix}$ )

\item
En d�duire que pour tout $n\in \mathbb{N} $ il existe $a_n$ et $b_n \in \mathbb{R}$ tels que $A^n = a_nA + b_n I_2$.
\item trouver $a_4$ et $b_4$.

\end{enumerate}
\end{enumerate}






\end{document}
