\documentclass{article}


\usepackage[latin1]{inputenc}
\usepackage[francais]{babel}
\usepackage{amsmath,amssymb}
\usepackage[T1]{fontenc}

\date{}
\begin{document}
\renewcommand{\labelitemi}{$\circ$}
\title{CONTROLE 3 LM 121 MIME 11-3 }
\maketitle
\textit{Rappel : Pour r�soudre un syst�me lin�aire, on fait des op�rations sur les lignes en les indiquant sur sa copie. Toute autre m�thode ne sera pas prise en compte.} \\
\bigskip
\\

QUESTION 1 (\textit{sur 5 points}) \\
\medskip
Soit $A=
\begin{pmatrix}
3 & 4 & 4\\
1 & 2 & 1\\
0 & -2 & 4
\end{pmatrix}$.
Calculer son d�terminant, et si c'est possible, calculer $A^{-1}$.\\
\bigskip
\\
QUESTION 2 (\textit{sur 2 points})\\
\medskip
Donner deux matrices $A$ et $B \in M_3 (\mathbb{R})$ telles que $AB\neq BA$. \\
\bigskip
\\
QUESTION 3(\textit{ sur 5 points}) \\
\medskip
\\
Soit $\mathcal{P}$  le plan de $\mathbb{R}^3$ passant par $(1,1,5)$ et de vecteurs directeurs $(1,1,3)$ et $(1,0,1)$.\\
Montrer que l'ensemble des $(x,y,z) \in \mathbb{R}^3$ qui v�rifient simultan�ment \\
$Det 
\begin{pmatrix} 
1 &-1&x \\
3&1&y\\
-1&-2&z 
\end{pmatrix} =1$
et $(x,y,z) \in \mathcal{P}$, est une droite dont on donnera une param�trisation.
\\
\bigskip
\\
QUESTION 4 (\textit{sur 4 points})
$$\begin{array}{lcccc}
Soit & F :& \mathbb{R}^3 &\to &\mathbb{R}^3\\
 & & \begin{pmatrix} x\\y\\z \end{pmatrix} & \mapsto & 
\begin{pmatrix} x+y+z \\ 2x-y-z^2 \\ x-3y+3z \end{pmatrix}
\end{array}$$
$F$ est-elle une application lin�aire? (justifier votre r�ponse)
\\
\bigskip
\\
QUESTION 5 (\textit{sur 4 points}) \\
\medskip
\\
Soit $u=(1,0,1)$ , $v=(3,-1,-1)$ , $w=(-1,2,2)$ et $z=(6,-5,-4)$. \\
a) Montrer que $u,v$ et $w$ forment une base de $\mathbb{R}^3$.\\
b) Trouver la d�composition de $z$ dans cette base. Dit autrement, trouver $a,b$ et $c \in \mathbb{R}$ tels que $au+bv+cw=z$.




\bibliographystyle{alpha}
\bibliography{bibli}







\end{document}
