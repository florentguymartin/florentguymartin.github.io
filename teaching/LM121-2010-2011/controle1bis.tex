\documentclass{article}


\usepackage[latin1]{inputenc}
\usepackage[francais]{babel}
\usepackage{amsmath,amssymb,amsfonts}
\usepackage[T1]{fontenc}




\newtheorem{Theoreme}{Th{\'e}or{\`e}me}[section]
\newtheorem{Definition}{D{\'e}finition}[section]
\newtheorem{Prop}{Proposition}[section]
\newtheorem{Lemme}{Lemme}[section]
\date{}
\begin{document}

\title{Controle 1 LM121}
\maketitle
\renewcommand{\labelitemi}{$\circ$}

\begin{enumerate}

\item
Calculer $\sum_{k=0}^{10} \cos (\frac{k\pi}{5})$ . (indication : regarder $\sum_{k=0}^{10}e^{\frac{ik\pi}{5}}$.)


\item
Soit $f(z)= e^{\frac{i\pi}{3}}z +3$. Calculer $f^{(6)}(z) = f\circ f\circ f\circ f\circ f\circ f(z)$.

\item
Trouver 3 vecteurs $u,v ,w$ de $\mathbb{R}^3$ qui soient libres, et justifier qu'ils sont libres.

\item

Soit $u=(3,-2,4)$ , $v=(-2,1,-1)$ , $w=(5,-4,10)$ et $z=(1,1,1)$. \par
(a) \ $u,v$ et $w$ son-ils libres? \par
(b) \ $z$ est-il  combinaison lin�aire de $u,v$ et $w$?

\item
Trouver $\delta \in \mathbb{C} $ (en pr�cisant bien ses parties r�elles et imaginaires)  tel que $\delta^2 = 2i+3$.

\end{enumerate}

\bibliographystyle{alpha}
\bibliography{bibli}





\end{document}
