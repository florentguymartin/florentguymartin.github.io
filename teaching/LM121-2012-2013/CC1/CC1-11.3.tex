\documentclass{article}

\usepackage[latin1]{inputenc}
\usepackage[francais]{babel}
\usepackage{amsmath,amssymb,amsfonts}
\usepackage[T1]{fontenc}
\usepackage{mathrsfs}




\begin{document}
\large
\noindent 
\textbf{Universit� Pierre et Marie Curie - LM 121 - 2012/2013}\\
\begin{center}
\Large 
Contr�le continu n� 1
\end{center}

\normalsize
\medskip
\noindent
\textbf{Exercice 1:}\\
Soit $u,v \in \mathbb{C}$. Montrer que 
$|u+v|^2 + |u-v|^2 = 2(|u|^2 +|v|^2) $. 


\medskip
\noindent
\textbf{Exercice 2:}\\
D�terminer l'ensemble $\mathscr{S}$ des nombres complexes $z$ tels que 
$z^4 = -8 -8\sqrt{3}i$. 
En particulier, donner la forme cart�sienne des �l�ments de $\mathscr{S}$, et les repr�senter dans le plan complexe.




\medskip
\noindent
\textbf{Exercice 3 :}\\
Calculer 
\[ \int_0^{\frac{\pi}{2}} \cos^5(x)dx\]

\medskip
\noindent
\textbf{Exercice 4:}\\

\begin{enumerate}
\item   Consid�rons le polyn\^ome � coefficients complexes suivant : \\
$P(z) = z^3-iz^2-4z+4i$.
 \begin{enumerate}
  \item Montrer que $2$ est une racine de $P$.
  \item Trouver un polyn\^ome � coefficients complexes $Q(z)$ tel que 
    $P(z)=(z-2) Q(z)$. 
  \item Factoriser $P$. 
           \end{enumerate}
\item 
Plus g�n�ralement, soit $P(z) = a_nz^n + a_{n-1}z^{n-1} + \ldots + a_1z +a_0$ un 
polyn\^ome � coefficients complexes (i.e. $a_i\in \mathbb{C}$), et soit 
$\alpha \in \mathbb{C}$. 
D�montrer le r�sultat suivant (qui figure dans votre cours) : si $\alpha$ est une 
racine de $P$, alors il existe un polyn\^ome � coefficients complexes $Q(z)$ 
tel que $P(z) = (z-\alpha ) Q(z)$. \\
\emph{Indication : on pourra faire un r�currence sur $n$.}
\end{enumerate}


\end{document}

\bibliographystyle{plain}
\bibliography{bibli}