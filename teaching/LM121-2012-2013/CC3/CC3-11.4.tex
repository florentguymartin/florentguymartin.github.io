\documentclass[a4paper, 10pt]{article}


\usepackage[utf8]{inputenc}
\usepackage[francais]{babel}
\usepackage{amsmath,amssymb,amsfonts}
\usepackage[T1]{fontenc}
\usepackage{mathrsfs}
\usepackage{graphicx}
\usepackage{theorem}
\usepackage{mathdots}

%\usepackage[cp850]{inputenc}
%\usepackage[textures]{epsfig}
%\usepackage{alltt}
%\renewcommand{\baselinestretch}{1,25}
%\psfigdriver{dvips}

\pagestyle{empty}
% my margins

\addtolength{\oddsidemargin}{-.875in}
\addtolength{\evensidemargin}{-.875in}
\addtolength{\textwidth}{1.75in}

\addtolength{\topmargin}{-.875in}
\addtolength{\textheight}{1.75in}
\theoremstyle{plain}
\theorembodyfont{\upshape}
\newtheorem{thm}{Th\'eoreme}
\newtheorem{cor}[thm]{Corollaire}
\newtheorem{lem}[thm]{Lemme}
\newtheorem{prop}[thm]{Proposition}
\newtheorem{defn}[thm]{D\'efinition}
\newtheorem{rem}[thm]{Remarque}
\newtheorem{ex}{Exercice}{\theorembodyfont{\upshape}}


\newcommand{\R}{\mathbb{R}}
\newcommand{\Z}{\mathbb{Z}}
\newcommand{\C}{\mathbb{C}}
\newcommand{\N}{\mathbb{N}}
\newcommand{\T}{\mathbb{T}}
\newcommand{\Q}{\mathbb{Q}}
\newcommand{\D}{\mathbb{D}}
\newcommand{\K}{\mathbb{K}}
\newcommand{\Li}{\mathcal{L}}
\newcommand{\M}{\mathcal{M}}
\newcommand{\F}{\mathbf{F}}
\renewcommand{\S}{\mathbb{S}}

\newcommand{\im}{\textup{Im}}
\newcommand{\BC}{\mathcal{B}}
\newcommand{\CC}{\mathcal{C}}
\newcommand{\DC}{\mathcal{D}}
\newcommand{\can}{_{\mathrm{can}}}

\newcommand{\Mat}{\textup{Mat}}
\newcommand{\Vect}{\mathrm{Vect}}
\newcommand{\rg}{\textup{rg}}
\newcommand{\cm}{\textup{Comat}}

\begin{document}
\noindent
\large
\textbf{Universit\'e Pierre et Marie Curie 
 - LM121 -
Ann\'ee 2012-2013}\\

\begin{center}
\Large
\textbf{Interro n$^o$ 3}
\end{center}
\normalsize

\medskip
\noindent
Question de cours : \emph{donner la définition d'une base de $\R^2$.}

\bigskip
\noindent
\textbf{Exercice 1:}\\
Soit $A =\begin{pmatrix}
          3&4&4 \\
         1&2&1 \\
         0&-2&4
         \end{pmatrix}$.
Calculer son déterminant, et si c'est possible calculer $A^{-1}$.

\bigskip
\noindent
\textbf{Exercice 2:}\\
Soit $A=(1,1,1) , B= (1,2,3) , C=(0,1,2)$. Ces points sont-ils alignés? Si non, donner 
une équation du plan $(ABC)$.

\medskip
\noindent
\textbf{Exercice 3:}\\
On considère le déterminant suivant de taille $2n$ : 
\[D = 
\begin{vmatrix}
a & 0 & \cdots & \cdots & \cdots & \cdots &0 & b \\
0 & a & \ddots  &       &      &\iddots & b & 0 \\
\vdots & \ddots & \ddots & 0  & 0 &\iddots & \iddots  & \vdots \\
\vdots &  & 0 & a & b & 0 & &\vdots \\
\vdots & & 0 & b & a & 0 & & \vdots \\
\vdots & \iddots  & \iddots & 0 & 0 & \ddots & \ddots & \vdots \\
0 & b & \iddots & & & \ddots & a & 0 \\
b &0 & \cdots & \cdots & \cdots & \cdots & 0 & a
\end{vmatrix}
\] 
Calculer $D$. 


\bigskip
\noindent
\textbf{Exercice 4:}\\
Soit 
$M=
\begin{pmatrix}
4 & -4 \\1&0
\end{pmatrix}
, P= 
\begin{pmatrix}
2&1\\1&0
\end{pmatrix}
$ et
$T=
\begin{pmatrix}
2&1\\0&2
\end{pmatrix}$.
\begin{enumerate}
\item 
Calculer $P^{-1}$.
\item Pour $n\in \N^*$ trouver\footnote{Si vous bloquez, 
commencez par calculer $T^2,T^3,T^4, T^5$.} une formule simple pour 
$T^n$. Justifier soigneusement votre résultat.
\item 
Montrer que pour $n\in \N^*$, $(P^{-1}MP)^n = P^{-1}M^nP$. 
Calculer $P^{-1}MP$ et en déduire une formule pour 
$M^n$.  
\item 
On considère la suite $(u_n)$ définie par 
$u_0 =3$, $u_1 =4$ et par la relation de récurrence : 
\[u_{n+2} = 4u_{n+1}-4u_n \hspace{1cm} \text{pour} \ n\in \N \]
\begin{enumerate}

\item
Trouver une matrice $A\in M_2(\R)$ telle que 
$\begin{pmatrix}
u_{n+2} \\ u_{n+1}
\end{pmatrix} = 
A 
\begin{pmatrix}
u_{n+1} \\u_n
\end{pmatrix}$
pour tout $n\in \N$. 
En déduire que 
$\begin{pmatrix}
u_{n+1}\\u_n
\end{pmatrix} = 
A^n 
\begin{pmatrix}
4\\3
\end{pmatrix}$ 
pour tout $n\in \N$.

\item
En utilisant la question précédente, et la formule obtenue pour 
$M^n$ à la question $(3)$, 
en déduire une formule pour $u_n$. \\
Puis montrer que 
$u_n \neq 0$ pour $n\geq 4$.\\
Finalement justifier que 
$\displaystyle \lim_{n\to + \infty} \frac{u_{n+5}}{u_{n+4}}$ existe et la calculer.

\end{enumerate}
\end{enumerate}




\end{document}


