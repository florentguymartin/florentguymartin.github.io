\documentclass[a4paper, 10pt]{article}


\usepackage[utf8]{inputenc}
\usepackage[francais]{babel}
\usepackage{amsmath,amssymb,amsfonts}
\usepackage[T1]{fontenc}
\usepackage{mathrsfs}
\usepackage{mathdots}
\usepackage{theorem}

%\usepackage[cp850]{inputenc}
%\usepackage[textures]{epsfig}
%\usepackage{alltt}
%\renewcommand{\baselinestretch}{1,25}
%\psfigdriver{dvips}

\pagestyle{empty}
% my margins

\addtolength{\oddsidemargin}{-.875in}
\addtolength{\evensidemargin}{-.875in}
\addtolength{\textwidth}{1.75in}

\addtolength{\topmargin}{-.875in}
\addtolength{\textheight}{1.75in}
\theoremstyle{plain}
\theorembodyfont{\upshape}
\newtheorem{thm}{Th\'eoreme}
\newtheorem{cor}[thm]{Corollaire}
\newtheorem{lem}[thm]{Lemme}
\newtheorem{prop}[thm]{Proposition}
\newtheorem{defn}[thm]{D\'efinition}
\newtheorem{rem}[thm]{Remarque}
\newtheorem{ex}{Exercice}{\theorembodyfont{\upshape}}


\newcommand{\R}{\mathbb{R}}
\newcommand{\Z}{\mathbb{Z}}
\newcommand{\C}{\mathbb{C}}
\newcommand{\N}{\mathbb{N}}
\newcommand{\T}{\mathbb{T}}
\newcommand{\Q}{\mathbb{Q}}
\newcommand{\D}{\mathbb{D}}
\newcommand{\K}{\mathbb{K}}
\newcommand{\Li}{\mathcal{L}}
\newcommand{\M}{\mathcal{M}}
\newcommand{\F}{\mathbf{F}}
\renewcommand{\S}{\mathbb{S}}

\newcommand{\im}{\textup{Im}}
\newcommand{\BC}{\mathcal{B}}
\newcommand{\CC}{\mathcal{C}}
\newcommand{\DC}{\mathcal{D}}
\newcommand{\can}{_{\mathrm{can}}}

\newcommand{\Mat}{\textup{Mat}}
\newcommand{\Vect}{\mathrm{Vect}}
\newcommand{\rg}{\textup{rg}}
\newcommand{\cm}{\textup{Comat}}

\begin{document}
\noindent
\large
\textbf{Universit\'e Pierre et Marie Curie 
 - LM121 -
Ann\'ee 2012-2013}\\

\begin{center}
\Large
\textbf{Correction Interro n$^o$ 3}
\end{center}
\normalsize

\medskip
\noindent
\textbf{Exercice 1:}\\
On trouve 
$\det(A) = 6 \neq 0$ donc $A$ est inversible. On calcule son inverse avec la méthode du pivot de Gauss : 

\begin{align*}
\begin{pmatrix}
 3 & 4 & 4 \\
1 & 2 & 1 \\
0 & -2 & 4 
\end{pmatrix}
\left| 
\begin{pmatrix} 
 1 & 0 & 0 \\
0 & 1 & 0 \\
 0 & 0 & 1 
\end{pmatrix}
\right. 
&
\xrightarrow[]{L_1 \leftrightarrow L_2 }
&
\begin{pmatrix}
 1 & 2 & 1 \\
3 & 4 & 4 \\
0 & -2 & 4 
\end{pmatrix}
\left| 
\begin{pmatrix} 
 0 & 1 & 0 \\
1 & 0 & 0 \\
 0 & 0 & 1 
\end{pmatrix}
\right. 
&
\xrightarrow[]{L_2 \leftarrow L_2-3L_1 } \\
\begin{pmatrix}
 1 & 2 & 1 \\
0 & -2 & 1 \\
0 & -2 & 4 
\end{pmatrix}
\left| 
\begin{pmatrix} 
 0 & 1 & 0 \\
1 & -3 & 0 \\
 0 & 0 & 1 
\end{pmatrix}
\right. 
&
\xrightarrow[]{L_2 \leftarrow \frac{L_2}{-2} }
&
\begin{pmatrix}
 1 & 2 & 1 \\
0 & 1 & \frac{-1}{2} \\
0 & -2 & 4 
\end{pmatrix}
\left| 
\begin{pmatrix} 
 0 & 1 & 0 \\
\frac{-1}{2} & \frac{3}{2} & 0 \\
 0 & 0 & 1 
\end{pmatrix}
\right. 
&
\xrightarrow[L_3 \leftarrow L_3 +2L_2]{L_1 \leftarrow L_1 - 2 L_2 }
\\
\begin{pmatrix}
 1 & 0 & 2 \\
0 & 1 & \frac{-1}{2} \\
0 & 0 & 3 
\end{pmatrix}
\left| 
\begin{pmatrix} 
 1 & -2 & 0 \\
\frac{-1}{2} & \frac{3}{2} & 0 \\
 -1 & 3 & 1 
\end{pmatrix}
\right. 
&
\xrightarrow[]{L_3 \leftarrow \frac{L_3}{3} }
&
\begin{pmatrix}
 1 & 0 & 2 \\
0 & 1 & \frac{-1}{2} \\
0 & 0 & 1 
\end{pmatrix}
\left| 
\begin{pmatrix} 
 1 & -2 & 0 \\
\frac{-1}{2} & \frac{3}{2} & 0 \\
 \frac{-1}{3} & 1 & \frac{1}{3} 
\end{pmatrix}
\right. 
&
\xrightarrow[L_2 \leftarrow L_2+ \frac{1}{2}L_3]{L_1 \leftarrow L_1-2L_3 }
\\
\begin{pmatrix}
 1 & 0 & 0 \\
0 & 1 & 0\\
0 & 0 & 1 
\end{pmatrix}
\left| 
\begin{pmatrix} 
 \frac{5}{3} & -4 & \frac{-2}{3} \\
\frac{-2}{3} & 2 & \frac{1}{6} \\
 \frac{-1}{3} & 1 & \frac{1}{3} 
\end{pmatrix}
\right. 
\end{align*}
On vérifie bien que 
$\begin{pmatrix} 
 \frac{5}{3} & -4 & \frac{-2}{3} \\
\frac{-2}{3} & 2 & \frac{1}{6} \\
 \frac{-1}{3} & 1 & \frac{1}{3} 
\end{pmatrix}
\begin{pmatrix}
 3 & 4 & 4 \\
1 & 2 & 1 \\
0 & -2 & 4 
\end{pmatrix}
=
\begin{pmatrix}
 1 & 0 & 0 \\
0 & 1 & 0\\
0 & 0 & 1 
\end{pmatrix}$, 
donc $A^{-1} = 
\begin{pmatrix} 
 \frac{5}{3} & -4 & \frac{-2}{3} \\
\frac{-2}{3} & 2 & \frac{1}{6} \\
 \frac{-1}{3} & 1 & \frac{1}{3} 
\end{pmatrix}$.

\bigskip
\noindent
\textbf{Exercice 2:}\\
On vérifie que les points ne sont pas alignés. $(ABC)$ est donc le plan passant par 
$A$ et dirigé par les vecteurs libres 
$\overrightarrow{AB}= \begin{pmatrix} 0 \\1 \\ 2 \end{pmatrix}$ et 
$ \overrightarrow{AC} = \begin{pmatrix} -1\\ 0 \\ 1\end{pmatrix}$.\\
Leur produit vectoriel est \\
$\begin{pmatrix} 0 \\1 \\ 2 \end{pmatrix} \wedge 
\begin{pmatrix} -1\\ 0 \\ 1\end{pmatrix}=
\begin{pmatrix} 1 \\ -2 \\ 1.\end{pmatrix}$ qui est donc un vecteur normal à $(ABC)$.
Le plan a donc une équation de la forme
$x -2y +z =d$ . Pour trouver $d$ on remplace cette équation avec les coordonnées de $A$ :\\
$1-2+1=d=0$\\
Donc $x -2y +z =0$ est une équation de $(ABC)$.
\vspace{0.4cm}
\\
\noindent
\textbf{Exercice 3:}\\

\[
\begin{vmatrix}
a & 0 & \cdots & \cdots & \cdots & \cdots &0 & b \\
0 & a & \ddots  &       &      &\iddots & b & 0 \\
\vdots & \ddots & \ddots & 0  & 0 &\iddots & \iddots  & \vdots \\
\vdots &  & 0 & a & b & 0 & &\vdots \\
\vdots & & 0 & b & a & 0 & & \vdots \\
\vdots & \iddots  & \iddots & 0 & 0 & \ddots & \ddots & \vdots \\
0 & b & \iddots & & & \ddots & a & 0 \\
b &0 & \cdots & \cdots & \cdots & \cdots & 0 & a
\end{vmatrix}
\xrightarrow{ 
\begin{matrix}
C_1 & \leftarrow &C_1 +C_{2n} \\
C_2 &\leftarrow & C_2 +C_{2n-1} \\
& \vdots & \\
C_{n} & \leftarrow &  C_n + C_{n+1}
\end{matrix}
}
\begin{vmatrix}
a+b & 0 & \cdots & \cdots & \cdots & \cdots &0 & b \\
0 & a+b & \ddots  &       &      &\iddots & b & 0 \\
\vdots & \ddots & \ddots & 0  & 0 &\iddots & \iddots  & \vdots \\
\vdots &  & 0 & a+b & b & 0 & &\vdots \\
\vdots & & 0 & a+b & a & 0 & & \vdots \\
\vdots & \iddots  & \iddots & 0 & 0 & \ddots & \ddots & \vdots \\
0 & a+b & \iddots & & & \ddots & a & 0 \\
a+b &0 & \cdots & \cdots & \cdots & \cdots & 0 & a
\end{vmatrix}
\]
\[
\xrightarrow{ 
\begin{matrix}
L_{n+1} & \leftarrow & L_{n+1} -L_n \\
L_{n+2} &\leftarrow & L_{n+2} - L_{n-1} \\
& \vdots & \\
L_{2n} & \leftarrow &  L_{2n} - L_1
\end{matrix}
}  
\begin{vmatrix}
a+b & 0 & \cdots & \cdots & \cdots & \cdots &0 & b \\
0 & a+b & \ddots  &       &      &\iddots & b & 0 \\
\vdots & \ddots & \ddots & 0  & 0 &\iddots & \iddots  & \vdots \\
\vdots &  & 0 & a+b & b & 0 & &\vdots \\
\vdots & &  & 0 & a-b & 0 & & \vdots \\
\vdots &   & &  & 0 & \ddots & \ddots & \vdots \\
0 &  &  & & & \ddots & a-b & 0 \\
0 &0 & \cdots & \cdots & \cdots & \cdots & 0 & a-b
\end{vmatrix} 
=(a+b)^n(a-b)^n
\]

\bigskip
\noindent
\textbf{Exercice 4:}\\
\begin{enumerate}
\item
$P^{-1} =
\begin{pmatrix}
0 & 1 \\ 1 & -2
\end{pmatrix}$. 
\item
En faisant quelques essais, on a le sentiment que la formule doit être 
$T^n = 
\begin{pmatrix}
2^n & n2^{n-1} \\
0 & 2^n
\end{pmatrix}
$. \\
Montrons par récurrence que pour tout $n\geq 1$, 
$T^n = 
\begin{pmatrix}
2^n & n2^{n-1} \\
0 & 2^n
\end{pmatrix}
$. \\
Pour $n=1$ le résultat est vrai. \\
Soit $n\in \N^*$, et supposons que 
$T^n = 
\begin{pmatrix}
2^n & n2^{n-1} \\
0 & 2^n
\end{pmatrix}
$.
Alors 
$T^{n+1} =T^n T = 
\begin{pmatrix}
2^n & n2^{n-1} \\
0 & 2^n
\end{pmatrix}
\begin{pmatrix}
 2&1\\ 0 & 2
\end{pmatrix}
=
\begin{pmatrix}
2\cdot 2^n& 2^n +2n2^{n-1} \\
0 & 2\cdot 2^n
\end{pmatrix}
=
\begin{pmatrix}
2^{n+1} & (n+1)2^n \\
0 & 2^{n+1}
\end{pmatrix}
$, ce qui achève notre récurrence.

\item 
On montre par récurrence que pout tout $n\in \N^*$, 
$(P^{-1}MP)^n=P^{-1}M^nP$.\\
Pour $n=1$ c'est vrai. \\
Soit $n\in \N^*$. 
Supposons que 
$(P^{-1}MP)^n=P^{-1}M^nP$. Alors 
$(P^{-1}MP)^{n+1}=
(P^{-1}MP)^n (P^{-1}MP)=
P^{-1}M^nP P^{-1}MP = 
P^{-1}M^nMP =P^{-1}M^{n+1}P$, ce qui achève notre récurrence.\\
De plus 
$
P^{-1}MP=
\begin{pmatrix}
0 & 1 \\
1 & -2
\end{pmatrix}
\begin{pmatrix}
4 & -4\\1&0
\end{pmatrix}
\begin{pmatrix}
2&1\\1&0
\end{pmatrix}
=
\begin{pmatrix}
1&0\\2&-4
\end{pmatrix}
\begin{pmatrix}
2&1\\1&0
\end{pmatrix}
=\begin{pmatrix}
2&1\\0&2
\end{pmatrix}
=T$.\\
Ainsi, $P^{-1}M^nP=T^n$. En multipliant à gauche par $P$ et à droite par 
$P^{-1}$, on en déduit que 
$M^n = PT^nP^{-1}$.\\
En utilisant le résultat de la question précédente, on en déduit que \\
$M^n = 
\begin{pmatrix}
2&1\\1&0
\end{pmatrix}
\begin{pmatrix}
2^n & n2^{n-1} \\
0 & 2^n
\end{pmatrix}
\begin{pmatrix}
0 &1 \\
1 & -2
\end{pmatrix}
=
\begin{pmatrix}
2^{n+1} & (n+1)2^n \\
2^n & n2^{n-1}
\end{pmatrix}
\begin{pmatrix}
0&1\\1&-2
\end{pmatrix}
=
\begin{pmatrix}
(n+1)2^n & -n2^{n+1} \\
n2^{n-1} & -(n-1)2^n
\end{pmatrix}
$.
\item 
\begin{enumerate}
\item
On trouve 
$\begin{pmatrix}
u_{n+2} \\u_{n+1}
\end{pmatrix}
=
\begin{pmatrix}
4u_{n+1} -4u_n \\ u_{n+1}
\end{pmatrix}=
 \begin{pmatrix}
4 & -4 \\ 1 &0
\end{pmatrix}
\begin{pmatrix}
u_{n+1} \\u_n
\end{pmatrix}
$, donc 
$A= \begin{pmatrix}
4&-4\\1&0
\end{pmatrix} =M
$ marche (et c'est en fait la seule matrice qui marche).\\
On montre par récurrence que pour tout $n\in \N$,  
$\begin{pmatrix}
u_{n+1} \\u_n
\end{pmatrix} =
A^n \begin{pmatrix}
4\\3
\end{pmatrix}$.\\
Pour $n=0$, on a $A^0 = I_2$, et 
$\begin{pmatrix}
u_1\\u_0
\end{pmatrix}
=\begin{pmatrix}
4\\3
\end{pmatrix}
=I_2 \begin{pmatrix}
4\\3
\end{pmatrix}
=
A^0
\begin{pmatrix}
4\\3
\end{pmatrix}$. \\
Si le résultat est vrai pour $n$, d'après ce qu'on a montré au début de la question,
$\begin{pmatrix}
u_{n+2}\\u_{n+1}
\end{pmatrix}=
A
\begin{pmatrix}
u_{n+1} \\u_n
\end{pmatrix}
=A\cdot A^n 
\begin{pmatrix}
4\\3
\end{pmatrix}
=A^{n+1}
\begin{pmatrix}
4\\3
\end{pmatrix}$ ce qui achève notre récurrence.

\item
D'après la question précédente 
$\begin{pmatrix}
u_{n+1} \\u_n
\end{pmatrix}
=A^n \begin{pmatrix}
4\\3
\end{pmatrix}
= M^n 
\begin{pmatrix}
4\\3
\end{pmatrix}$.
On utilise maintenant le résultat de la question $(3)$ pour en déduire que 
$\begin{pmatrix}
u_{n+1} \\u_n
\end{pmatrix}
=
\begin{pmatrix}
(n+1)2^n & -n2^{n+1} \\
n2^{n-1} & -(n-1)2^n
\end{pmatrix}
\begin{pmatrix}
4\\3
\end{pmatrix}
=
\begin{pmatrix}
4(n+1)2^n - 3n2^{n+1} \\
4n2^{n-1} - 3(n-1)2^n
\end{pmatrix}
$.\\
Ainsi, 
$u_n = 4n2^{n-1} -3(n-1)2^n =
2n2^n - 3n2^n +3\cdot 2^n 
=
2^n(3-n)$.
En particulier, $u_n \neq 0$ quand $n\neq 3$, on en déduit que si 
$n\geq 4$, $u_n \neq 0$.
Donc pour $n\in \N$, $u_{n+4} \neq 0$, et 
\[\frac{u_{n+5}}{u_{n+4}}=
\frac{2^{n+5}(3-(n+5))}{2^{n+4}(3-(n+4))}
=\frac{2(-n-2)}{-n-1} \xrightarrow[n\to +\infty]{} 2.\]
\end{enumerate}
\end{enumerate}

\end{document}


