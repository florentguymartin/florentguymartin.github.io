\documentclass[a4paper, 10pt]{article}

\usepackage[utf8]{inputenc}
\usepackage[francais]{babel}
\usepackage{amsmath,amssymb,amsfonts}
\usepackage[T1]{fontenc}
\usepackage{mathrsfs}
\usepackage{theorem}

%\usepackage[cp850]{inputenc}
%\usepackage[textures]{epsfig}
%\usepackage{alltt}
%\renewcommand{\baselinestretch}{1,25}
%\psfigdriver{dvips}

\pagestyle{empty}
% my margins

\addtolength{\oddsidemargin}{-.875in}
\addtolength{\evensidemargin}{-.875in}
\addtolength{\textwidth}{1.75in}

\addtolength{\topmargin}{-.875in}
\addtolength{\textheight}{1.75in}
\theoremstyle{plain}
\theorembodyfont{\upshape}
\newtheorem{thm}{Th\'eoreme}
\newtheorem{cor}[thm]{Corollaire}
\newtheorem{lem}[thm]{Lemme}
\newtheorem{prop}[thm]{Proposition}
\newtheorem{defn}[thm]{D\'efinition}
\newtheorem{rem}[thm]{Remarque}
\newtheorem{ex}{Exercice}{\theorembodyfont{\upshape}}


\newcommand{\R}{\mathbb{R}}
\newcommand{\Z}{\mathbb{Z}}
\newcommand{\C}{\mathbb{C}}
\newcommand{\N}{\mathbb{N}}
\newcommand{\T}{\mathbb{T}}
\newcommand{\Q}{\mathbb{Q}}
\newcommand{\D}{\mathbb{D}}
\newcommand{\K}{\mathbb{K}}
\newcommand{\Li}{\mathcal{L}}
\newcommand{\M}{\mathcal{M}}
\newcommand{\F}{\mathbf{F}}
\renewcommand{\S}{\mathbb{S}}

\newcommand{\im}{\textup{Im}}
\newcommand{\BC}{\mathcal{B}}
\newcommand{\CC}{\mathcal{C}}
\newcommand{\DC}{\mathcal{D}}
\newcommand{\can}{_{\mathrm{can}}}

\newcommand{\Mat}{\textup{Mat}}
\newcommand{\Vect}{\mathrm{Vect}}
\newcommand{\rg}{\textup{rg}}
\newcommand{\cm}{\textup{Comat}}

\begin{document}
\noindent
\large
\textbf{Universit\'e Pierre et Marie Curie 
 - LM121 -
Ann\'ee 2012-2013}\\

\begin{center}
\Large
\textbf{Interro n$^o$ 3}
\end{center}
\normalsize

\medskip
\noindent
Question de cours : \emph{donner la définition d'une application linéaire de $\R^2$ dans $\R^2$.}

\medskip
\noindent
\textbf{Exercice 1:}\\
Soit $A = \begin{pmatrix} 1 & -1&2\\2&1&-2\\3&-2&1 \end{pmatrix}$.
\begin{enumerate}
 \item Calculer $\det (A)$.
\item $A$ est-elle inversible? Si oui, calculer $A^{-1}$.
\item Résoudre le système\\
$\left\{
\begin{array}{cccll}
 x&-y&+2z&=&3\\
2x&+y&-2z&=&6\\
3x&-2y&+z&=&6
\end{array}
\right.
$
\end{enumerate}

\bigskip
\noindent
\textbf{Exercice 2:}\\
Soir $r_1$ la rotation de centre 
$(1,1)$ et d'angle $\frac{\pi}{2}$, et 
$r_2$ la rotation de centre $(3,-1)$ et d'angle $\frac{\pi}{2}$.
Calculer l'image du point $(2,-1)$ par $r_2 \circ r_1$.

\bigskip
\noindent
\textbf{Exercice 3:}\\
Soit $D$ le déterminant de taille $n$ suivant : 
$D = 
\begin{vmatrix}
1 & 1 & 1 & \cdots & 1 \\
1 & 1 & 0 & \cdots & 0 \\
1 & 0 & 1 & \ddots & \vdots \\
\vdots & \vdots & \ddots & \ddots & 0 \\
1 & 0 & \cdots & 0 &1
\end{vmatrix}$.\\
Calculer $D$.

\bigskip
\noindent
\textbf{Exercice 4:}\\
Soit $C = 
\begin{pmatrix}
1 & 2 & 3 \\
-1 & 2 & 0 \\
1 & -1 & 1
\end{pmatrix}
$.\\
a) Calculer $C^3 -4C^2$. \\
b) Trouver $\lambda$ et $\mu \in \mathbb{R}$ tels que 
$C^3-4C^2 +\lambda C +\mu I_3=0$. \\
c) En déduire $C^{-1} $




\end{document}


