\documentclass{article}

\usepackage[utf8]{inputenc}
\usepackage[francais]{babel}
\usepackage{amsmath,amssymb,amsfonts}
\usepackage[T1]{fontenc}
\usepackage{mathrsfs}



\newcommand{\R}{\mathbb{R}}
\newcommand{\Z}{\mathbb{Z}}
\newcommand{\C}{\mathbb{C}}
\newcommand{\N}{\mathbb{N}}
\newcommand{\T}{\mathbb{T}}
\newcommand{\Q}{\mathbb{Q}}
\newcommand{\D}{\mathbb{D}}
\newcommand{\K}{\mathbb{K}}
\newcommand{\Li}{\mathcal{L}}
\newcommand{\M}{\mathcal{M}}
\newcommand{\F}{\mathbf{F}}
\renewcommand{\S}{\mathbb{S}}

\newcommand{\im}{\textup{Im}}
\newcommand{\BC}{\mathcal{B}}
\newcommand{\CC}{\mathcal{C}}
\newcommand{\DC}{\mathcal{D}}
\newcommand{\can}{_{\mathrm{can}}}

\newcommand{\Mat}{\textup{Mat}}
\newcommand{\Vect}{\mathrm{Vect}}
\newcommand{\rg}{\textup{rg}}
\newcommand{\cm}{\textup{Comat}}


\begin{document}
\large
\noindent 
\textbf{Université Pierre et Marie Curie - LM 121 - 2012/2013}\\
\begin{center}
\Large 
Contrôle continu n 2 MIME 11.3
\end{center}
\medskip
\noindent
\textbf{Exercice 1:}\\
Soit $u=(1,2,3) , \ v=(-1,1,2) ,\ w=(3,3,4)$ et $z=(1,1,1)$ des vecteurs de $\mathbb{R}^3$. 
\begin{enumerate}
\item $u, v$ et $w$ sont-ils liés? Si oui, donner une combinaison linéaire non-triviale de 
$u,v,w$ qui soit nulle. 
\item $z$ est-il combinaison linéaire de $u,v$ et $w$?
\end{enumerate}




\medskip
\noindent
\textbf{Exercice 2:}\\
On considère les deux droites suivantes : 
\[
D_1 : 
\begin{cases}
x+y-z &=-9 \\
x-y-2 &=0
\end{cases} \hspace{2cm}
 D_2 : \begin{cases}
 3x-y-z&=-5 \\
 x-z &=-1 \end{cases}
 \]
 \begin{enumerate}
 \item determiner $D_1\cap D_2$.
 \item Donner une équation cartésienne de l'unique plan $\mathcal{P}$ contenant ces deux droites.
 \end{enumerate}





\medskip
\noindent
\textbf{Exercice 3:}\\
Soit $u,v$ et $w\in \R^3$. 
Montrer que 
$\det(u,v,w)= (u \wedge v)\cdot w$. (Ici $\cdot$ représente le produit scalaire).





\medskip
\noindent
\textbf{Exercice 4:}\\
Soit $f$ la rotation de centre $(0,0)$ et d'angle $\frac{\pi}{2}$, et $g$ la rotation de centre $(1,2)$ et d'angle $\frac{\pi}{2}$. Identifier géométriquement $g \circ f$ (à savoir donner le centre et l'angle de cette rotation).




\end{document}

\bibliographystyle{plain}
\bibliography{bibli}
