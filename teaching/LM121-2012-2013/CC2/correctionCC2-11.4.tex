\documentclass{article}

\usepackage[utf8]{inputenc}
\usepackage[francais]{babel}
\usepackage{amsmath,amssymb,amsfonts}
\usepackage[T1]{fontenc}
\usepackage{mathrsfs}

\newcommand{\R}{\mathbb{R}}
\newcommand{\Z}{\mathbb{Z}}
\newcommand{\C}{\mathbb{C}}
\newcommand{\N}{\mathbb{N}}
\newcommand{\T}{\mathbb{T}}
\newcommand{\Q}{\mathbb{Q}}
\newcommand{\D}{\mathbb{D}}
\newcommand{\K}{\mathbb{K}}
\newcommand{\Li}{\mathcal{L}}
\newcommand{\M}{\mathcal{M}}
\newcommand{\F}{\mathbf{F}}
\renewcommand{\S}{\mathbb{S}}

\newcommand{\im}{\textup{Im}}
\newcommand{\BC}{\mathcal{B}}
\newcommand{\CC}{\mathcal{C}}
\newcommand{\DC}{\mathcal{D}}
\newcommand{\can}{_{\mathrm{can}}}

\newcommand{\Mat}{\textup{Mat}}
\newcommand{\Vect}{\mathrm{Vect}}
\newcommand{\rg}{\textup{rg}}
\newcommand{\cm}{\textup{Comat}}


\begin{document}
\large
\noindent 
\textbf{Université Pierre et Marie Curie - LM 121 - 2012/2013}\\
\begin{center}
\Large 
Correction Contrôle continu n  2 MIME 11.4
\end{center}




\medskip
\noindent
\textbf{Exercice 1 :}\\

\begin{enumerate}
\item 
$\det(u,v,w)=0$ donc les vecteurs ne sont pas libres. 
Pour trouver une combinaison linéaire nulle : 
$au+bv +cv =0$ avec $a,b,c$ non tous nuls, on résoud le système 
\[ \begin{matrix}
3a &-2b & +5c & =0\\
-2a &+b & -4c & =0 \\
4a &-b & +10c & =0
\end{matrix}\]
On trouve par exemple une solution 
$(a,b,c) =(3,2,-1)$ qui correspond au fait que 
$3u+2v-w=0$.
\item 
On cherche donc $a,b,c$ tels que $au+bv+cv=z$. Cela amène à résoudre le système suivant :
\[
\begin{matrix}
3a &-2b & +5c & =1\\
-2a &+b & -4c & =1 \\
4a &-b & +10c & =1
\end{matrix}
\ \Leftrightarrow \hspace{1cm} 
\begin{matrix}
L_1\leftarrow L_1+L_2 &a &-b & +c & =2\\
 &-2a &+b & -4c & =1 \\
&4a &-b & +10c & =1
\end{matrix} \]
$
\Leftrightarrow \hspace{1cm} 
\begin{matrix}
 &a &-b & +c & =2\\
L_2 \leftarrow L_2 +2L_1  & & -b & -2c & =5 \\
L_3 \leftarrow L_3 -4L_1 & &3b & +6c & =-7
\end{matrix}
$ \\
Mais la ligne 2 indique que 
$b+2c = \frac{-5}{2}$ alors que la ligne 3 indique que 
$b+2c = \frac{-7}{2}$, ce qui est impossible, donc $z$ n'est pas combinaison linéaire de $u,v$ et $w$.
\end{enumerate}


\medskip
\noindent
\textbf{Exercice 2:}\\

Pour trouver l'équation cartésienne de $\mathcal{D}$ comme $t=x-2$ les égalités correspondant à $y$ et $z$ donnent ce système d'équations pour $\mathcal{D}$
\begin{align*}
 2x-y-1 &=0 \\
x+z-3 &=0
\end{align*}
Ainsi un point de $\mathcal{D'}$ étant de la forme $(2t,1-t,2+2t)$ s'il appartenait aussi à $\mathcal{D}$ devrait aussi vérifier son équation cartésienne, à savoir :
\[
\begin{cases}
 2(2t) - (1-t)-1 &=0 \\
2t +2+2t -3 &=0
\end{cases}
\hspace{0.8cm} \Leftrightarrow \hspace{0.8cm} 
\begin{cases}
5t-2 &=0 \\
4t -1 & =0
\end{cases}\]
Ce qui aboutit à $t=\frac{2}{5}$ et $t=\frac{1}{4}$, ce qui est impossible, donc les deux droites n'ont pas de point commun.


\medskip
\noindent
\textbf{Exercice 3:}\\
On pose $x = \begin{pmatrix}
              x_1 \\ x_2 \\ x_3
             \end{pmatrix}$.
\begin{enumerate}
\item 
L'équation est 
$\begin{pmatrix}
  1\\1\\-1
 \end{pmatrix}
 \wedge 
\begin{pmatrix}
 x_1\\x_2\\x_3
\end{pmatrix}
 = 
\begin{pmatrix}
 x_2 +x_3 \\-x_1 -x_3 \\-x_1 +x_2
\end{pmatrix}
 = \begin{pmatrix} 
    1\\2\\3
   \end{pmatrix}
$
qui équivaut au système 
\[ 
\begin{matrix}
      & x_2   & +x_3 & =1 \\
-x_1 &  & -x_3   & =2 \\
-x_1  & +x_2 & & =3
 \end{matrix}
\hspace{1cm} \Leftrightarrow 
\]

\[\begin{matrix}
 &  & x_2  &+x_3 & = 1 \\
L_2 \leftarrow -L_2 &x_1  & & +x_3 & =-2 \\
& -x_1  &+x_2 &  &=3
 \end{matrix} 
\hspace{1cm} \Leftrightarrow 
\]

\[\begin{matrix} 
 L_1 \leftrightarrow  L_2 & x_1 & & +x_3 & =-2 \\
             &      &x_2  &+x_3 & = 1 \\
             & -x_1 &+x_2   &    & =3
 \end{matrix}
\hspace{1cm} \Leftrightarrow 
\]

\[
\begin{matrix} 
                               & x_1 & & +x_3 & =-2 \\
                                 &      &x_2  &+x_3 & = 1 \\
 L_3 \leftarrow L_3 + L_1  &  &x_2   &+x_3    & =1
 \end{matrix}
\hspace{1cm} \Leftrightarrow 
\]

\[\begin{matrix} 
                               & x_1 & & +x_3 & =-2 \\
                                 &      &x_2  &+x_3 & = 1 \\
 \end{matrix}
\]
On reconnaît l'équation d'une droite, en prenant comme paramètre $t=x_3$ par exemple, on obtient \\
$\left\{ 
\begin{matrix}
x_1& = &-2-t \\
x_2&=&1-t \\
x_3&=&t
\end{matrix} \right.$


\item 
L'équation est 
$\begin{pmatrix} 
  1 \\ 1 \\ -1 
 \end{pmatrix}
\wedge 
\begin{pmatrix}
 x_1 \\x_2 \\ x_3
\end{pmatrix}
= 
\begin{pmatrix}
 x_2+x_3 \\ -x_1 - x_3 \\ x_2-x_1 
\end{pmatrix}
 = \begin{pmatrix}
    3\\2\\1
   \end{pmatrix}$
\ qui nous amène à résoudre le système suivant :

\[\begin{matrix}
   & x_2  &+ x_3 & = 3 \\
 -x_1  & & -x_3 & =2 \\
-x_1  & +x_2 &  &=1
 \end{matrix} 
 \hspace{1cm} \Leftrightarrow 
\]
\[\begin{matrix}
 &  & x_2  & +x_3 & = 3 \\
L_2 \leftarrow -L_2 &x_1  & & +x_3 & =-2 \\
& -x_1  &+x_2 &  &=1
 \end{matrix} 
 \hspace{1cm} \Leftrightarrow 
\]

\[ 
\begin{matrix} 
 L_1 \leftrightarrow  L_2 & x_1 & & +x_3 & =-2 \\
             &      &x_2  & +x_3 & = 3 \\
             & -x_1 &+x_2   &    & =1
 \end{matrix}
\hspace{1cm} \Leftrightarrow 
\]

\[
\begin{matrix} 
                               & x_1 & & +x_3 & =-2 \\
                                 &      &x_2  &+x_3 & = 3 \\
 L_3 \leftarrow L_3 + L_1  &  &x_2   &+x_3    & =-1
 \end{matrix}
\hspace{1cm} \Leftrightarrow 
\]
Les lignes 2 et 3 sont incompatibles, donc il n'y a pas de solution.
\end{enumerate}

\medskip
\noindent
\textbf{Exercice 4:}\\
$M = \begin{pmatrix}
x\\y\\z
\end{pmatrix} \in \mathcal{P}$ si et seulement si  
$\overrightarrow{AM}$ est une combinaison linéaire de 
$\overrightarrow{AB}$ et 
$\overrightarrow{AC}$. Comme on voit que ces deux vecteurs sont libres, cela équivaut à dire que 
$\overrightarrow{AM}$, $\overrightarrow{AB}$ et 
$\overrightarrow{AC}$ sont liés, ce qui équivaut à dire que 
$\det ( \overrightarrow{AM}, \overrightarrow{AB} , \overrightarrow{AC})=0$.
On calcule ce déterminant : \\
\[\begin{vmatrix}
x-1 & 2 & 1 \\
y-1 & 1 & -2 \\
z-1 & -3 & 1
\end{vmatrix}
= -5x -5y-5z+15 \]
Une équation de $\mathcal{P}$ est donc 
\[ x+y+z =3 .\]


\end{document}

\bibliographystyle{plain}
\bibliography{bibli}
