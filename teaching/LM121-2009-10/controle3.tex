\documentclass{article}


\usepackage[latin1]{inputenc}
\usepackage[francais]{babel}
\usepackage{amsmath,amssymb,amsfonts}
\usepackage[T1]{fontenc}




\newtheorem{Theoreme}{Th{\'e}or{\`e}me}[section]
\newtheorem{Definition}{D{\'e}finition}[section]
\newtheorem{Prop}{Proposition}[section]
\newtheorem{Lemme}{Lemme}[section]

\begin{document}

\title{Contr�le 3}
\maketitle
\renewcommand{\labelitemi}{$\circ$}

\begin{enumerate}
 \item 
Soit $A = \begin{pmatrix} 1 & -1&2\\2&1&-2\\3&-2&1 \end{pmatrix}$.
\begin{enumerate}
 \item Calculer $Det (A)$.
\item $A$ est-elle inversible? Si oui, calculer $A^{-1}$.
\item R�soudre le syst�me
$$\begin{array}{cccll}
 x&-y&+2z&=&3\\
2x&+y&-2z&=&6\\
3x&-2y&+z&=&6
\end{array}
$$
\end{enumerate}

\item
\begin{enumerate}
 \item Soit 
$$\begin{array}{crcl}
\Phi :&  \mathbb{R}^2 & \to & \mathbb{R}^2 \\
 & (x,y) & \mapsto & (\frac{y^2}{1+x^2},x+2y)
\end{array}
$$
$\Phi$ est-elle une application lin�aire? Est-elle bijective?
\item
Soit 
$$\begin{array}{crcl}
\Psi :&  \mathbb{R}^3 & \to & \mathbb{R}^3 \\
 & (x,y,z) & \mapsto & (2x+y-z,x-z,x+y+z)
\end{array}
$$
$\Psi$ est-elle une application lin�aire? Est-elle bijective?
\end{enumerate}

\item
R�soudre le syst�me
$$\begin{array}{cccl}
 x&+2y&+z&=0\\
2x&-y&+z&=4\\
3x&+y&+2z&=4\\
5x&&+3z&=8
\end{array}$$

\end{enumerate}












\end{document}
