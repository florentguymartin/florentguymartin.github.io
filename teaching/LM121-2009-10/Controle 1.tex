\documentclass{article}


\usepackage[latin1]{inputenc}
\usepackage[francais]{babel}
\usepackage{amsmath,amssymb,amsfonts}
\usepackage[T1]{fontenc}




\newtheorem{Theoreme}{Th{\'e}or{\`e}me}[section]
\newtheorem{Definition}{D{\'e}finition}[section]
\newtheorem{Prop}{Proposition}[section]
\newtheorem{Lemme}{Lemme}[section]

\begin{document}

\title{Controle 1}
\maketitle
\renewcommand{\labelitemi}{$\circ$}

\begin{enumerate}
 \item[1]
Soit $a\in \mathbb{C}^*$. Montrer qu'il existe exactement deux nombres complexes $z$ tels que $z^2 =a$.


\item[2]
Donner les racines carr�es de $2i -5$.

\item[3]
soit $u=(-2,1,0)$ , \ $v=(3,0,1)$ et $w=(1,1,t)$ o� $t\in \mathbb{R}$. Pour quel(s) $t$ ces trois vecteurs sont li�s?

\item[4]
\begin{enumerate}
 \item Soit $\mathcal{D}$ la droite d'�quation parametrique
\begin{align*}
 x&=2+t \\
y & =3+2t \\
z & =1-t 
\end{align*}
Donner une �quation cart�sienne de $\mathcal{D}$.
\item
Soit $\mathcal{D'}$ la droite d'�quation parametrique
\begin{align*}
 x &=2t \\
y & =1-t \\
z & = 2+2t
\end{align*}
$\mathcal{D}$ et $\mathcal{D'}$ ont-elles des points en commun? (dit autrement, a-t-on $\mathcal{D} \cap \mathcal{D'} =\emptyset$?)

\end{enumerate}


\end{enumerate}












\end{document}
