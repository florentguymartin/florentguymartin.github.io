\documentclass{article}


\usepackage[latin1]{inputenc}
\usepackage[francais]{babel}
\usepackage{amsmath,amssymb,amsfonts}
\usepackage[T1]{fontenc}
\usepackage{enumerate}



\newtheorem{Theoreme}{Th{\'e}or{\`e}me}[section]
\newtheorem{Definition}{D{\'e}finition}[section]
\newtheorem{Prop}{Proposition}[section]
\newtheorem{Lemme}{Lemme}[section]

\date{}

\begin{document}

\title{Contr�le continu 3 LM 121 PCME 14.2}
\maketitle



\begin{enumerate}
 

\item
D�terminer $a,b \in \mathbb{R}$ tels que 
$ \begin{pmatrix} a&b\\2&-2 \end{pmatrix} . \begin{pmatrix} 1&3\\2&-4 \end{pmatrix} = 
\begin{pmatrix} 11&3\\-2 &14 \end{pmatrix}$ 

\item 
Soit $A =\begin{pmatrix}
          3&4&4 \\
         1&2&1 \\
         0&-2&4
         \end{pmatrix}$.
Calculer son d�terminant, et si c'est possible calculer $A^{-1}$.

\item
Soit $A=(1,1,1) , B= (1,2,3) , C=(0,1,2)$. Ces points sont-ils align�s? Si non, donner 
une �quation du plan $(ABC)$.

\item
Trouver une matrice $B\in M_3(\mathbb{R})$ telle que $B\neq 0$ et $B^2 =0$.


\end{enumerate}








\end{document}
