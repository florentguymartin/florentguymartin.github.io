\documentclass{article}


\usepackage[latin1]{inputenc}
\usepackage[francais]{babel}
\usepackage{amsmath,amssymb,amsfonts}
\usepackage[T1]{fontenc}
\usepackage{enumerate}



\newtheorem{Theoreme}{Th{\'e}or{\`e}me}[section]
\newtheorem{Definition}{D{\'e}finition}[section]
\newtheorem{Prop}{Proposition}[section]
\newtheorem{Lemme}{Lemme}[section]

\begin{document}

\title{Correction contr�le 1 LM121}
\maketitle

\begin{enumerate}
 \item 
$M_1$ est le point de coordonn�es $(3,2)$. \\
$M_2$ de coordonn�es $(-\frac{1}{2} , \frac{\sqrt{3}}{2})$ . C'est le point du cercle unit� qui fait un angle 
de $\frac{2\pi}{3}$ avec l'axe $(Ox)$.
$$z_{M_3} = \frac{-3+i}{1+i} = \frac{(-3+i)(1-i)}{(1+i)(1-i)} = \frac{-3+3i+i+1}{1+1}
=\frac{-2+4i}{2}=-1+2i$$
$M_3$ est donc le point de coordonn�es $(-1,2)$. \\
$z_{M_4}=2e^{\frac{-4i\pi}{3}}=2e^{\frac{2i\pi}{3}} = 2(-\frac{1}{2} +i\frac{\sqrt{3}}{2} ) 
=-1+i\sqrt{3}$.
$M_4$ a donc pour coordonn�es $(-1,\sqrt{3} )$.


\item

On pose $z=re^{i\theta}$. L'�galit� devient 
$z^4 = r^4 e^{4i\theta} = -16 = 16e^{i\pi} = 2^4e^{i\pi}$.
Une solution est donc 
$r=2$ et $\theta = \frac{\pi}{4}$, qui nous donne la solution particuli�re :\\
$z_1= 2e^{\frac{i\pi}{4} } \ \ \ (=\sqrt{2} +i\sqrt{2} )$.\\
Ainsi $z^4 = -16 = z_1^4 \Leftrightarrow \left ( \frac{z}{z_1} \right)^4 = 1 $ \\
$\Leftrightarrow \frac{z}{z_1}  \in \{ 1, e^{\frac{i\pi}{2} } , e^{i\pi} , e^{\frac{3i\pi}{2}} \}$.\\
Les solutions sont donc :
$$z_1 = 2e^{\frac{i\pi}{4}} , z_2 = 2e^{\frac{3i\pi}{4} } , z_3 =2 e^{\frac{5i\pi}{4}} , z_4 = 2e^{\frac{7i\pi}{4} }$$


\item
On pose $z=x+iy$. On a alors :

$$\frac{2z+1}{z+i} = \frac{2x+1+2iy}{x+i(y+1)}= \frac{(2x+1+2iy)(x-i(y+1) )}{(x+i(y+1) )(x-i(y+1) )}$$
$$=\frac{(2x+1)x + 2y(y+1) +i( -(2x+1)(y+1) + 2yx )}{x^2 +(y+1)^2}$$
$$= \frac{2x^2+x+2y^2+2y +i(-2xy -2x-y-1 +2yx) }{x^2+(y+1)^2}$$
$$=\frac{2x^2+x+2y^2+2y + i(-2x-y-1)}{x^2+(y+1)^2}$$
La partie imaginaire sera nulle si et seulement si
$-2x-y-1=0$ , ce qui correspond donc � la droite $\mathcal{D}$  d'�quation $y=-2x-1$, priv�e du point $(0,-1)$ , qui correspond au nombre complexe d'affixe $-i$, qui n'est pas dans le domaine de d�finition de 
$z \mapsto \frac{z+1}{z+i}$. \\


 
\item
$$\sin^3( \theta) = \left( \frac{e^{i\theta} - e^{-i\theta}}{2i} \right)^3=
\left(\frac{e^{i\theta} - e^{-i\theta}}{2i} \right)^2 \left( \frac{e^{i\theta} - e^{-i\theta} }{2i} \right) $$
$$= \left( \frac{e^{2i\theta} -2 + e^{-2i\theta} }{-4} \right) \left( \frac{e^{i\theta} - e^{-i\theta}}{2i} \right)
= \frac{-1}{4} \left( \frac{e^{3i\theta} -2e^{i\theta} + e^{-i\theta} - e^{i\theta} + 2 e^{-i\theta} - e^{-3i\theta}}{2i} \right) $$
$$= \frac{-1}{4} \left( \frac{e^{3i\theta} - e^{-3i\theta} }{2i} - 3 (\frac{e^{i\theta} - e^{-i\theta} }{2i} ) \right)$$

$$=\frac{-1}{4} \sin (3\theta ) + \frac{3}{4 } \sin (\theta) $$

\item
On pose 
$Z=z^2$.
L'�galit� devient
$$1+Z+Z^2+Z^3+Z^4+Z^5=0$$
Si cette �galit� a lieu, $Z\neq 1$ (sinon en rempla�ant, on aurait $6\times1 =0$ ).
La formule d'une somme g�om�trique nous donne alors :
$$1+Z+Z^2+Z^3+Z^4+Z^5= \frac{Z^6-1}{Z-1}=0$$
donc $Z^6-1=0$, donc $z^{12}=Z^6=1$ , donc $z$ est une racine 
$12$-i�me de l'unit�, donc $|z|=1$.

\end{enumerate}














\end{document}
