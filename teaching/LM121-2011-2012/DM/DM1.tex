\documentclass{article}

\usepackage[all]{xy}
\usepackage[latin1]{inputenc}
\usepackage[francais]{babel}
\usepackage{amsmath,amssymb,amsfonts}
\usepackage{dsfont}
\usepackage[T1]{fontenc}
\usepackage{enumerate}

\newtheorem{rem}{Remarque}
\newtheorem{defi}{D{\'e}finition}[section]
\newtheorem{prop}{Proposition}[section]
\newtheorem{lemme}{Lemme}[section]
\newtheorem{cor}{Corollaire}[section]

\date{}
\begin{document}
\noindent LM 121 2011/2012\\
PCME 14.2 
\vskip 1cm
Voici une pr�cision sur l'organisation de l'ann�e : \\
Vous aurez deux DM � rendre , et trois contr�les continus.
A priori, les dates des contr�les continus devraient �tre � peu pr�s (� chaque fois un vendredi) : 7 Octobre, 
4 Novembre et 25 Novembre. \\

\rule{\linewidth}{1pt}
\begin{center}
 {\Large DM} � rendre lundi 3 Octobre
\end{center}

\textit{   La pr\'esentation, la lisibilit\'e, l'orthographe, la
  qualit\'e de la r\'edaction, la clart\'e et la pr\'ecision des
  raisonnements entreront pour une part importante dans
  l'appr\'eciation des copies. En particulier, les r\'esultats non
  justifi\'es ne seront pas pris en compte. Vous �tes invit\'es \`a encadrer les r\'esultats de vos calculs.}



\begin{enumerate}[(a)]
\item
Mettre sous forme cart�sienne 
$z=\frac{4+i}{5i-3} $

\item
trouver la forme polaire de 
$z= -\frac{\sqrt{10}}{2} -i \frac{\sqrt{10}}{2}$ (on rappelle qu'�crire un nombre complexe 
$z$ sous forme polaire c'est l'�crire sous la forme 
$z=r e^{i\theta}$ avec $r\geq 0$ et $\theta \in \mathbb{R}$ un de ses arguments. Profitons en pour 
rappeler que $\theta$ n'est pas unique, il est seuleument \textit{ unique modulo } $2\pi$.
Par exemple $z=\frac{1}{2} + i \frac{\sqrt{3} }{2}= e^{i\frac{\pi}{3} } = e^{-i \frac{5\pi}{3} } $ , et donc 
 $\frac{\pi}{3}$ et $-\frac{5\pi}{3}$ sont tous les deux des arguments de $z$. Les arguments de $z$ sont dans cet exemple exactement les 
nombres de la forme $\frac{\pi}{3} + 2k\pi$ avec $k \in \mathbb{Z}$.)


\item
D�terminer les $z\in \mathbb{C}$ qui v�rifient : \\
$\frac{z+1}{2z-1}$ est un imaginaire pur, et 
$\left| \frac{3z+i}{z-3} \right | =3$.

\item
Soit $u$ et $v \in \mathbb{C}$.
Prouver que 
$|u+v|^2 + |u-v|^2 = 2(|u|^2 + |v|^2 ) $.

\item
Calculer \begin{displaymath}
          \int_0^{\frac{\pi}{2}} \sin (\theta) \cos^2 (\theta) d\theta
         \end{displaymath}
(indication : une m�thode pourrait consister � lin�ariser $\sin(\theta) \cos^2(\theta)$).

\item
Trouver les nombres complexes $z$ tels que 
$z^3 =-2$
\end{enumerate}









\end{document}
