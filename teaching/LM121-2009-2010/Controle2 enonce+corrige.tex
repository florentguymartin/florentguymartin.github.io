\documentclass{article}


\usepackage[latin1]{inputenc}
\usepackage[francais]{babel}
\usepackage{amsmath,amssymb,amsfonts}
\usepackage[T1]{fontenc}




\newtheorem{Theoreme}{Th{\'e}or{\`e}me}[section]
\newtheorem{Definition}{D{\'e}finition}[section]
\newtheorem{Prop}{Proposition}[section]
\newtheorem{Lemme}{Lemme}[section]

\begin{document}

\title{Controle 2}
\maketitle
\renewcommand{\labelitemi}{$\circ$}


\begin{enumerate}
 \item Calculer 
$Det \begin{pmatrix} -26 &3&-3 \\ -21&1&9\\20&4&6 \end{pmatrix}$ \par
Le r�sultat est $2010$.

\item
D�terminer $a,b \in \mathbb{R}$ tels que 
$ \begin{pmatrix} a&b\\2&-2 \end{pmatrix} . \begin{pmatrix} 1&3\\2&-4 \end{pmatrix} = 
\begin{pmatrix} 11&3\\-2 &14 \end{pmatrix}$ \par
Solution : En multipliant les deux matrices on tombe sur l'�galit� \\
$\begin{pmatrix} a+2b&3a-4b \\-2 &14 \end{pmatrix} = \begin{pmatrix}11 &3\\-2&14\end{pmatrix}$ Ce qui �quivaut donc au syst�me : \\
$\begin{array}{rl}
 a+2b=&11 \\
3a-4b=&3 
\end{array}
$
On r�sout ce syst�me et on trouve qu'il existe une unique solution : $a=5$, $b=3$.

\item
Soit $a,b,c \in \mathbb{C}$ tels que $a\neq c$. Montrer que pour tout $n\in \mathbb{N}^*$ on a 
$$\begin{pmatrix} a & b \\ 0&c \end{pmatrix}^n = \begin{pmatrix} a & b\frac{a^n-c^n}{a-c} \\ 0 & c^n \end{pmatrix}$$

Solution : On montre le r�sultat par r�currence sur $n \geq 1$. \\
Pour initialiser la r�currence, on montre le r�sultat pour $n=1$ , qui d�coule du fait que pour $n=1$ 
on a $b\frac{a^n-c^n}{a-c}=b.1=b$. \\
Si on suppose le r�sultat vrai pour $n\geq 1$ alors 
$\begin{pmatrix} a&b\\0&c \end{pmatrix}^{n+1} = 
\begin{pmatrix} a^n & b\frac{a^n-c^n}{a-c} \\ 0 &c^n\end{pmatrix} \begin{pmatrix}a&b\\0&c \end{pmatrix} =$\\
$\begin{pmatrix} a^{n+1} & a^nb + cb\frac{a^n-c^n}{a-c} \\0 & c^{n+1} \end{pmatrix} =
\begin{pmatrix} a^{n+1} & b(\frac{a^n(a-c)}{a-c} + c\frac{a^n-c^n}{a-c}) \\0 & c^{n+1} \end{pmatrix} =$\\
$\begin{pmatrix} a^{n+1} & b\frac{a^{n+1} -a^nc + ca^n-c^{n+1}}{a-c} \\0 & c^{n+1} \end{pmatrix}=
\begin{pmatrix} a^{n+1} & b\frac{a^{n+1}-c^{n+1}}{a-c} \\0 & c^{n+1} \end{pmatrix}$.
Ce qui ach�ve la r�currence.

\item
Pour $A=\begin{pmatrix}a&b\\c&d \end{pmatrix} \in M_2(\mathbb{C})$ on pose $T(A) = a+d$. Montrer que pour tout $B,C \in M_2(\mathbb{C})$ on a $T(B.C)=T(C.B)$. \\
Solution : On pose $B=\begin{pmatrix} b_{1,1} & b_{1,2} \\ b_{2,1} & b_{2,2} \end{pmatrix}$ de m�me pour $C$. Alors
$B.C= \begin{pmatrix}  b_{1,1} & b_{1,2} \\ b_{2,1} & b_{2,2} \end{pmatrix} .
\begin{pmatrix} c_{1,1} & c_{1,2} \\ c_{2,1} & c_{2,2} \end{pmatrix} = $
 
$\begin{pmatrix}
 b_{1,1}c_{1,1} + b_{1,2} c_{2,1} & * \\ * & b_{2,1}c_{1,2}+ b_{2,2}.c_{2,2}
\end{pmatrix}$\\
Ainsi $T(B.C) = b_{1,1}c_{1,1} + b_{1,2}c_{2,1} + b_{2,1}c_{1,2} + b_{2,2}c_{2,2}$.
De m�me 
$C.B = \begin{pmatrix} c_{1,1} & c_{1,2} \\ c_{2,1} & c_{2,2} \end{pmatrix}
\begin{pmatrix}  b_{1,1} & b_{1,2} \\ b_{2,1} & b_{2,2} \end{pmatrix} =
\begin{pmatrix} c_{1,1}b_{1,1} + c_{1,2}b_{2,1} & * \\ * & c_{2,1}b_{1,2} + c_{2,2}b_{2,2} \end{pmatrix}$.
On obtient donc aussi $T(C.B)= b_{1,1}c_{1,1} + b_{1,2}c_{2,1} + b_{2,1}c_{1,2} + b_{2,2}c_{2,2}$.


  

\end{enumerate}











\end{document}
