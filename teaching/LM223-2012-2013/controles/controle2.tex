\documentclass[a4paper, 11pt]{article}
\usepackage[latin1]{inputenc}
\usepackage[francais]{babel}
\usepackage[latin1]{inputenc}
\usepackage[dvips,final]{graphics}
\usepackage{amsmath,amsfonts,amssymb}
\usepackage{theorem}
\usepackage[T1]{fontenc}
%\usepackage[cp850]{inputenc}
%\usepackage[textures]{epsfig}
%\usepackage{alltt}
%\renewcommand{\baselinestretch}{1,25}
%\psfigdriver{dvips}

\pagestyle{empty}
% my margins

\addtolength{\oddsidemargin}{-.875in}
\addtolength{\evensidemargin}{-.875in}
\addtolength{\textwidth}{1.75in}

\addtolength{\topmargin}{-.875in}
\addtolength{\textheight}{1.75in}
\theoremstyle{plain}
\theorembodyfont{\upshape}
\newtheorem{thm}{Th\'eoreme}
\newtheorem{cor}[thm]{Corollaire}
\newtheorem{lem}[thm]{Lemme}
\newtheorem{prop}[thm]{Proposition}
\newtheorem{defn}[thm]{D\'efinition}
\newtheorem{rem}[thm]{Remarque}
\newtheorem{ex}{Exercice}{\theorembodyfont{\upshape}}

\title{Fiche 1\\Espaces Vectoriels, Applications lin�aires}
\newcommand{\R}{\mathbb{R}}
\newcommand{\Z}{\mathbb{Z}}
\newcommand{\C}{\mathbb{C}}
\newcommand{\N}{\mathbb{N}}
\newcommand{\T}{\mathbb{T}}
\newcommand{\Q}{\mathbb{Q}}
\newcommand{\D}{\mathbb{D}}
\newcommand{\K}{\mathbb{K}}
\newcommand{\Li}{\mathcal{L}}
\newcommand{\M}{\mathcal{M}}
\newcommand{\F}{\mathbf{F}}
\renewcommand{\S}{\mathbb{S}}

\newcommand{\im}{\textup{Im}}
\newcommand{\BC}{\mathcal{B}}
\newcommand{\CC}{\mathcal{C}}
\newcommand{\DC}{\mathcal{D}}
\newcommand{\can}{_{\mathrm{can}}}

\newcommand{\Mat}{\textup{Mat}}
\newcommand{\Vect}{\mathrm{Vect}}
\newcommand{\rg}{\textup{rg}}
\newcommand{\cm}{\textup{Comat}}

\begin{document}
\noindent
\large
\textbf{Universit\'e Pierre et Marie Curie 
 - LM223 -
Ann\'ee 2012-2013}\\

\begin{center}
\Large
\textbf{Interro n$^o$ 2}
\end{center}
\normalsize

\medskip
\noindent
\textbf{Question de cours :}\\
\begin{enumerate}
\item 
Donner la d�finition d'une forme bilin�aire sym�trique ainsi que celle d'un produit scalaire.
\item Sur $\R^3$ donner quatre formes quadratiques $q_1, q_2, q_3$ et $q_4$ diff�rentes 
telles que $q_1$ et $q_2$ soient des produits scalaires, 
$q_3$ soit de signature $(2,1)$ et $q_4$ soit d�g�n�r�e.

\end{enumerate}

\bigskip
\noindent
\textbf{Exercice 1:}\\
\begin{enumerate}
\item 
Soit $M = 
\begin{pmatrix}
-2 & 4 & -9 \\
3 & -1 & 3 \\
2 & -2 & 5
\end{pmatrix}$.
$M$ est-elle diagonalisable? Si oui la diagonaliser (i.e. donner une base de vecteurs propres). 
\item 
M�me question avec 
$N = 
\begin{pmatrix}
5 & -1 & -2 \\
-5 & 3 & 3 \\
9 & -3 & -4
\end{pmatrix}$.
\end{enumerate}



\bigskip
\noindent
\textbf{Exercice 2:}\\
Sur $\R^3$ soit $q$ la forme quadratique d�finie par 
$q(x) = x_1^2 +7x_2^2 +12x_3^2 +4x_1x_2 -2x_1x_3 -16x_2x_3$. 
\begin{enumerate}
\item Donner la matrice $M$ de $q$ dans la base canonique de $\R^3$.
\item Donner une base orthogonale pour $q$.
\item Quelle est la signature de $q$?
\item Trouver un �l�ment $x\in \Z^3$ tel que $q(x) =-1$.
\end{enumerate}
\medskip
\noindent
\textbf{Exercice 3:}\\
Sur $\R^3$ soit $q$ la forme quadratique d�finie par 
$q(x) = x_1x_2 -2x_1x_3 +4x_2x_3$. On note $M$ la matrice de $q$ dans la base canonique de 
$\R^3$.
Trouver $P \in$ GL$_3(\R)$ telle que 
$^tPMP$ soit une matrice diagonale.


\bigskip
\noindent
\textbf{Exercice 4:}\\
Soit 
\[ \begin{array}{rccc}
q:& \R_2[X] &\to & \R \\
& P & \mapsto & P(1)P(-1)
\end{array} \]
\begin{enumerate}
\item 
Justifier que $q$ est une forme quadratique, en donnant la forme polaire associ�e � $q$.
\item 
Donner la matrice de $q$ dans la base canonique de 
$\R_2[X]$, i.e. $\{1,X,X^2 \}$. 
\item D�terminer la signature de $q$.
\item Montrer que $\mathcal{B} = \{X-1,X+1,X^2-1\}$ est une base de $\R_2[X]$.
\item Calculer la matrice de $q$ dans la base $\mathcal{B}$. 
\end{enumerate}    


\bigskip
\noindent
\textbf{Exercice 5:}\\
Soit la forme quadratique de $\R^3$, 
$q(x) = x_1^2 +3x_2^2 -8x_3^2$.
Soit $F=\{(x_1,x_2,x_3) \in \R^3 \ \big| \ x_1+x_2+x_3 =0 \}$ le sous-espace vectoriel de $\R^3$.
Alors, $q_{|F}$ induit une forme quadratique sur $F$.\\
Donner la dimension de $F$, et calculer la signature de $q_{|F}$ qu'on voit comme 
une forme quadratique sur $F$.

\bigskip
\noindent
\textbf{Exercice 6:}\\
\begin{enumerate}

\item Soit $q$ une forme quadratique sur $\R^3$. D�terminer, suivant la signature de 
$q$, s'il existe $x\in \R^3 \setminus \{0\}$ tel que $q(x)=0$.
\item Soit $n\geq 2$ et $q$ une forme quadratique sur $\C^n$. Montrer qu'il existe 
$x\in \C^n \setminus \{0\}$ tel que $q(x)=0$. 
\end{enumerate}

\end{document}


