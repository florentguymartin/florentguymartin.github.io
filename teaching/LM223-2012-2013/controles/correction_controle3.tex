\documentclass[a4paper, 10pt]{article}
\usepackage[utf8]{inputenc}
\usepackage[francais]{babel}
\usepackage[dvips,final]{graphics}
\usepackage{amsmath,amsfonts,amssymb}
\usepackage{theorem}
\usepackage[T1]{fontenc}

%\usepackage[cp850]{inputenc}
%\usepackage[textures]{epsfig}
%\usepackage{alltt}
%\renewcommand{\baselinestretch}{1,25}
%\psfigdriver{dvips}

\pagestyle{empty}
% my margins

\addtolength{\oddsidemargin}{-.875in}
\addtolength{\evensidemargin}{-.875in}
\addtolength{\textwidth}{1.75in}

\addtolength{\topmargin}{-.875in}
\addtolength{\textheight}{1.75in}
\theoremstyle{plain}
\theorembodyfont{\upshape}
\newtheorem{thm}{Th\'eoreme}
\newtheorem{cor}[thm]{Corollaire}
\newtheorem{lem}[thm]{Lemme}
\newtheorem{prop}[thm]{Proposition}
\newtheorem{defn}[thm]{D\'efinition}
\newtheorem{rem}[thm]{Remarque}
\newtheorem{ex}{Exercice}{\theorembodyfont{\upshape}}


\newcommand{\R}{\mathbb{R}}
\newcommand{\Z}{\mathbb{Z}}
\newcommand{\C}{\mathbb{C}}
\newcommand{\N}{\mathbb{N}}
\newcommand{\T}{\mathbb{T}}
\newcommand{\Q}{\mathbb{Q}}
\newcommand{\D}{\mathbb{D}}
\newcommand{\K}{\mathbb{K}}
\newcommand{\Li}{\mathcal{L}}
\newcommand{\M}{\mathcal{M}}
\newcommand{\F}{\mathbf{F}}
\renewcommand{\S}{\mathbb{S}}

\newcommand{\im}{\textup{Im}}
\newcommand{\BC}{\mathcal{B}}
\newcommand{\CC}{\mathcal{C}}
\newcommand{\DC}{\mathcal{D}}
\newcommand{\can}{_{\mathrm{can}}}

\newcommand{\Mat}{\textup{Mat}}
\newcommand{\Vect}{\mathrm{Vect}}
\newcommand{\rg}{\textup{rg}}
\newcommand{\cm}{\textup{Comat}}

\begin{document}
\noindent
\large
\textbf{Universit\'e Pierre et Marie Curie 
 - LM223 -
Ann\'ee 2012-2013}\\

\begin{center}
\Large
\textbf{Correction de l'interro n$^o$ 3}
\end{center}
\normalsize

\medskip
\noindent
\textbf{Exercice 1:}\\
\begin{enumerate}
\item
Par définition, $\det(^tPP)=1$ donc 
$\det (^tP)\det(P)= \det(P)^2 =1$, et donc $\det(P) =\pm1$.


\item 
Par exemple 
$\begin{pmatrix}
1&0\\
0&1
\end{pmatrix}$,
$\begin{pmatrix}
0&-1\\
1&0
\end{pmatrix}$,
$\begin{pmatrix}
-1&0\\0&-1
\end{pmatrix}$
$\begin{pmatrix}
0&1\\-1&0
\end{pmatrix}$, qui correspondent aux rotations d'angle $0,\frac{\pi}{2}, \pi , \frac{3\pi}{2}$.

\item 
Une solution est 
$P = 
\begin{pmatrix}
\frac{2}{3} & \frac{1}{\sqrt{5}} & \frac{-4}{3\sqrt{5}} \\[3pt]
\frac{-1}{3} & \frac{2}{\sqrt{5}} & \frac{2}{3\sqrt{5}} \\[3pt]
\frac{2}{3} & 0 & \frac{5}{3\sqrt{5}} 
\end{pmatrix}$.

\end{enumerate}

\bigskip
\noindent
\textbf{Exercice 2:}\\
\begin{enumerate}
\item 
On calcule d'abord le poynôme caractéristique de $M$, \\
$\chi_M = X^3-7X^2+14X-8 = (X-1)(X-2)(X-4)$. \\
Ainsi $M$ admet 3 valeurs propres distinctes. Pour chaque valeur propre, on calcule 
un vecteur propre $v_{\lambda}$ (en calculant $\ker (M-\lambda I)$). On trouve \\
$v_1 = (1,-1,1)$, $v_2 = (1,1,0)$ et 
$v_4 = (1,-1,-2)$. Au passage, une fois qu'on connaît $v_1$ et $v_2$, on sait que 
$v_1 \wedge v_2$ sera vecteur propre pour $4$. 
Cela nous fournit une base orthogonale de $\R^3$, et il ne reste plus qu'à normaliser les vecteurs 
pour obtenir une base orthonormée. 
On pose donc \\
$P =
\begin{pmatrix}
\frac{1}{\sqrt{3}}  & \frac{1}{\sqrt{2}}   & \frac{1}{\sqrt{6}}  \\[3pt]
-\frac{1}{\sqrt{3}} & \frac{1}{\sqrt{2}}   &-\frac{1}{\sqrt{6}}  \\[3pt]
\frac{1}{\sqrt{3}}         &  0                   & - \frac{2}{\sqrt{6}}
\end{pmatrix}$, et on obtient 
$P^{-1}MP = 
\begin{pmatrix}
1&0&0\\0&2&0\\0&0&4
\end{pmatrix}$.
\item On obtient 
$q(x) = 2x_1^2 +2x_2^2 +3x_3^2  -2x_1x_3 +2x_2x_3$.
\item Au $1)$ on a trouvé $P\in O(3)$ telle que 
$P^{-1}MP = 
\begin{pmatrix}
1&0&0\\0&2&0\\0&0&4
\end{pmatrix}$. Mais comme 
$P\in O(3)$, $P^{-1} = ^tP$, et donc 
$^tPMP = 
\begin{pmatrix}
1&0&0\\0&2&0\\0&0&4
\end{pmatrix}$. Cela siginifie que dans la base 
$\mathcal{B}$ 
associée à $P$, la forme $q$ s'écrit 
$q(x') = x_1'^2 + 2x_2'^2 + 4x_3'^2$. Rappelons que par définition la base 
$\mathcal{B}$ est la base constituée des vecteurs colonnes de $P$.
Concrètement, si on pose 
$\mathcal{B} = \{ f_1,f_2,f_3\}$, pour 
$x = x_1'f_1+x_2'f_2+x_3'f_3$, on a 
$q(x) = x_1'^2 + 2x_2'^2 + 4x_3'^2$. \\
Ainsi la signature de $q$ est $(3,0)$ et $q$ est définie positive.
\item 
On se place dans la base $\mathcal{B}$, et on utilise les coordonnées de cette base qu'on note 
$x_i'$. Si $x\in \mathcal{E}$, 
$q(x) = x_1'^2 + 2x_2'^2 + 4x_3'^2 = 1$. 
Par ailleurs,\\
$1=q(x) = x_1'^2 + 2x_2'^2 + 4x_3'^2 \leq 4( x_1'^2 + x_2'^2 + x_3'^2) = 4\|x\|^2$.\\
Attention, ici on utilise le fait que la base $\mathcal{B}$ est orthonormée et donc que 
$ x_1'^2 + x_2'^2 + x_3'^2 = \|x\|^2$ .\\ 
Ainsi, pour 
$x\in \mathcal{E}$, 
$\frac{1}{4} \leq \|x\|^2$, donc 
$\frac{1}{2} \leq \|x\|$, 
et donc 
$m\geq \frac{1}{2}$.
Par ailleurs, on voit bien que ce minimum est atteint précisément sur 
les points de coordonées $(0,0,\frac{1}{2})$ et $(0,0,-\frac{1}{2})$ dans la base $\mathcal{B}$, ce qui correspond aux points 
$(\frac{2}{\sqrt{6}}, \frac{-2}{\sqrt{6}} , \frac{-4}{\sqrt{6}} )$ et 
$(\frac{-2}{\sqrt{6}}, \frac{2}{\sqrt{6}} , \frac{4}{\sqrt{6}} )$. De plus ce sont bien les seuls. 
En effet, si 
$(x_1' , x_2',x_3')$ sont les coordonnées d'un point de $\mathcal{E}$, qui atteint ce minimum, c'est à dire tel que  
$1=q(x) = x_1'^2 + 2x_2'^2 + 4x_3'^2$ et $\|x\| = \frac{1}{2}$. \\ 
Alors soit 
$x_3' = \pm \frac{1}{2}$ et alors forcément, 
$x_1'=x_2'=0$ car $q(x)=1$. Et cela nous donne bien les deux points ci-dessus. \\
Soit $x_3' \neq \pm \frac{1}{2}$. Alors comme 
$1=q(x) = x_1'^2 + 2x_2'^2 + 4x_3'^2$, on a forcément 
$|x_3'| < \frac{1}{2}$, et donc $x_1' \neq 0$ ou $x_2' \neq 0$. Dans un 
cas comme dans l'autre, on en déduit que \\
$1=q(x) = x_1'^2 + 2x_2'^2 + 4x_3'^2 < 4(x_1'^2 + x_2'^2 + x_3'^2 ) = 4\|x\|^2$, \\
et donc 
$\|x\| > \frac{1}{2}$, ce qui contredit le fait que 
$\|x\| = \frac{1}{2}$. \vspace{8pt}\\
Finalement, on a montré que le minimum est $m=\frac{1}{2}$ et qu'il est 
atteint uniquement aux deux points $(\frac{2}{\sqrt{6}}, \frac{-2}{\sqrt{6}} , \frac{-4}{\sqrt{6}} )$ et 
$(\frac{-2}{\sqrt{6}}, \frac{2}{\sqrt{6}} , \frac{4}{\sqrt{6}} )$.

\end{enumerate}
\medskip
\noindent
\textbf{Exercice 3:}\\
\begin{enumerate}
\item 
Après calcul, 
$\det(M)=-1$, donc $M$ est une symétrie orthogonale par rapport à une droite $D$ disons, et 
on a donc 
$D = \ker (M-I) = \ker 
(\begin{pmatrix}
\frac{-1}{5} & \frac{3}{5} \\[3pt]
\frac{3}{5}  & \frac{-9}{5}
\end{pmatrix}  )
 = \ker (
 \begin{pmatrix}
 -1&3\\3&-9
 \end{pmatrix})$.
 On trouve 
 $D= \{(x,y)\in \R^2 \ \big| \ y= \frac{1}{3} x \}$.

\item 
On choisit un vecteur non-nul de $\mathcal{D}$, par exemple 
$u=(1,2)$. On en déduit 
que $\frac{u}{\|u\|}$ est un vecteur unitaire de $\mathcal{D}$.
La projection sur 
$\mathcal{D}$ est ainsi donnée par 
\[
\begin{array}{lccc}
p : & \R^2 	& \to     & \R^2 \\
    & (x,y) & \mapsto & \frac{u}{\sqrt{5}}. (\frac{u}{\sqrt{5}} | (x,y) )
\end{array}
\]
soit 
$p(x,y) = ( \frac{x+2y}{5} , \frac{2x+4y}{5} )$.
Or 
si on note $s$ la symétrie recherchée, on a 
$s = id_{\R^2} -2p$, et il s'ensuit que 
$s(x,y) = ( \frac{3x-4y}{5} , \frac{-4x-3y}{5} )$
et la matrice de $s$ est donc 
$\frac{1}{5}
\begin{pmatrix}
3&-4\\-4&-3
\end{pmatrix}$.

\end{enumerate}

\bigskip
\noindent
\textbf{Exercice 4:}\\
Soit 
$u=(1,-2,-2)$,  
$v=(-4,5,2)$ et et $F$ le sous-espace vectoriel de 
$\R^3$ qu'ils engendrent. On considère $\R^3$ muni 
du produit scalaire usuel.
\begin{enumerate}
\item 
Déjà, on commence par remarquer que dim$(F)=2$, et la base en question doit donc contenir 2 
éléments. \\
On commence par définir \\
$v' :=  v - \frac{u}{\|u\|} . \left( \frac{u}{\|u\|} | v\right) = 
(-4,5,2) - \frac{ (1,-2,-2)}{3} . (\frac{-18}{3}) = (-4,5,2)+(2,-4,-4)
=(-2,1,-2)$.\\
Il ne reste plus qu'à normaliser, et on trouve qu'une base orthonormée de $F$ est \\
$\{e_1,e_2\}=\{\frac{1}{3} ( 1,-2,-2) , \frac{1}{3}(-2,1,-2) \}$.
\item 
Le plus simple est de calculer $u\wedge v = (6,6,-3)$. Ainsi, un vecteur normal 
à $F$ est $(2,2,-1)$, et une base orthonormée de $F^{\perp}$ est donc 
$\{ \frac{1}{3} (2,2,-1)\}$. 
\item 
On peut calculer la projection avec deux méthodes. \\
La première consiste à dire que l'on dispose d'une BON de $F$ à savoir $\{e_1,e_2\}$, donc 
la projection $p_F$ sur $F$ se calcule ainsi : \\
$p_F(x) = e_1 (e_1|x) + e_2(e_2|x)$.\\
Ici, cela donne : \\ 
$p_F(1,2,1) =  \frac{1}{9} (1,-2,-2) ((1,-2,-2)|(1,2,1))  + \frac{1}{9}(-2,1,-2)((-2,1,-2)|(1,2,1)) =$\\
$\frac{1}{9} (1,-2,-2) (-5)  + \frac{1}{9}(-2,1,-2)(-2) =
\frac{1}{9} (-5,10,10)  \frac{1}{9}(4,-2,4) = \frac{1}{9}  (-1,8,14)$. \\
L'autre méthode consiste à utiliser qu'on connaît une base orthonormée de 
$F^{\perp}$ à savoir $e_3 = \frac{1}{3} (2,2,-1) $. \\
On en déduit que 
que la projection sur $F^{\perp}$ 
est donnée par 
$P_{F^{\perp}} (x) =   e_3 (e_3|x)$, et que donc 
$p_F(x) = x-p_{F^{\perp} } (x)$.\\
En particulier, 
$p_F(1,2,1) = (1,2,1) - \frac{1}{9} (2,2,-1) ( (2,2,-1) | ( 1,2,1) ) 
= (1,2,1) - \frac{1}{9} ( 10,10,-5) = 
\frac{1}{9} ( -1,8,14)$. 
\item 
On rappelle que la symétrie orthogonale par rapport à $F$ a pour expression 
$s_F = p_{F} - p_{F^{\perp} }$. \\ 
Par ailleurs,  
$id_{\R^3} = p_F + p_{F^{\perp} } $, soit 
$p_F = id_{\R^3} -p_{F^{\perp} } $. \\
Il s'ensuit que 
$s_F = id_{\R^3} - 2p_{F^{\perp} }$.\\
Par ailleurs avec les notations ci-dessus, 
$p_{F^{\perp}} (x) = e_3 (e_3 | x )$, donc \\
$s_F (x) = x -2e_3(e_3 |x) = 
 (x_1,x_2,x_3) - \frac{2}{9} (2,2,-1) (2 x_1+2x_2-x_3) =$\\ 
 $(x_1,x_2,x_3) + (-\frac{4}{9}, - \frac{4}{9}, \frac{2}{9}) (2x_1+2x_2-x_3) = 
\frac{1}{9}(x_1 -8x_2 +4x_3 , -8x_1 +x_2 +4x_3, 4x_1 +4x_2 +7x_3)$.\\
La matrice de $s_F$ est donc 
 $
 \frac{1}{9}\begin{pmatrix}
 1&-8&4\\
 -8&1&4\\
 4&4&7
 \end{pmatrix}$.
 
\end{enumerate}







\end{document}


