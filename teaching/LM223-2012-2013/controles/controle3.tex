\documentclass[a4paper, 11pt]{article}
\usepackage[utf8]{inputenc}
\usepackage[francais]{babel}
\usepackage[dvips,final]{graphics}
\usepackage{amsmath,amsfonts,amssymb}
\usepackage{theorem}
\usepackage[T1]{fontenc}

%\usepackage[cp850]{inputenc}
%\usepackage[textures]{epsfig}
%\usepackage{alltt}
%\renewcommand{\baselinestretch}{1,25}
%\psfigdriver{dvips}

\pagestyle{empty}
% my margins

\addtolength{\oddsidemargin}{-.875in}
\addtolength{\evensidemargin}{-.875in}
\addtolength{\textwidth}{1.75in}

\addtolength{\topmargin}{-.875in}
\addtolength{\textheight}{1.75in}
\theoremstyle{plain}
\theorembodyfont{\upshape}
\newtheorem{thm}{Th\'eoreme}
\newtheorem{cor}[thm]{Corollaire}
\newtheorem{lem}[thm]{Lemme}
\newtheorem{prop}[thm]{Proposition}
\newtheorem{defn}[thm]{D\'efinition}
\newtheorem{rem}[thm]{Remarque}
\newtheorem{ex}{Exercice}{\theorembodyfont{\upshape}}


\newcommand{\R}{\mathbb{R}}
\newcommand{\Z}{\mathbb{Z}}
\newcommand{\C}{\mathbb{C}}
\newcommand{\N}{\mathbb{N}}
\newcommand{\T}{\mathbb{T}}
\newcommand{\Q}{\mathbb{Q}}
\newcommand{\D}{\mathbb{D}}
\newcommand{\K}{\mathbb{K}}
\newcommand{\Li}{\mathcal{L}}
\newcommand{\M}{\mathcal{M}}
\newcommand{\F}{\mathbf{F}}
\renewcommand{\S}{\mathbb{S}}

\newcommand{\im}{\textup{Im}}
\newcommand{\BC}{\mathcal{B}}
\newcommand{\CC}{\mathcal{C}}
\newcommand{\DC}{\mathcal{D}}
\newcommand{\can}{_{\mathrm{can}}}

\newcommand{\Mat}{\textup{Mat}}
\newcommand{\Vect}{\mathrm{Vect}}
\newcommand{\rg}{\textup{rg}}
\newcommand{\cm}{\textup{Comat}}

\begin{document}
\noindent
\large
\textbf{Universit\'e Pierre et Marie Curie 
 - LM223 -
Ann\'ee 2012-2013}\\

\begin{center}
\Large
\textbf{Interro n$^o$ 3}
\end{center}
\normalsize

\medskip
\noindent
\textbf{Exercice 1:}\\
\begin{enumerate}
\item
Montrer que si $P\in O(n)$, alors $\det (P) = \pm 1$. 

\item Donner quatre matrices de $SO(2)$.

\item 
Compléter la matrice suivante $P$ pour que $P \in SO(3)$ 
où
$P = 
\begin{pmatrix}
\frac{2}{3} & \cdot & \cdot \\[3pt]
\frac{-1}{3} & \cdot & \cdot \\[3pt]
\frac{2}{3} & \cdot & \cdot 
\end{pmatrix}$.

\end{enumerate}

\bigskip
\noindent
\textbf{Exercice 2:}\\
Soit 
$M = 
\begin{pmatrix}
2&0&-1\\
0&2&1\\
-1&1&3
\end{pmatrix}
$. 
\begin{enumerate}
\item 
Trouver une matrice $P\in O(3)$ telle que 
$P^{-1}MP$ soit diagonale.
\item Soit $q$ la forme quadratique associée à $M$. Donner l'expression de $q$.
\item Est-ce que $q$ est définie positive? 
\item Soit $\mathcal{E} = \{ x\in \R^3 \ \big| \ q(x)=1\}$. 
On note $\displaystyle m= \inf_{x\in \mathcal{E}} \|x\|$. Montrer que $m\in \R$, et qu'il existe exactement deux points $p_1, p_2\in \mathcal{E}$ tels que 
$\|p_1\| = \|p_2\| = m$. 
\end{enumerate}
\medskip
\noindent
\textbf{Exercice 3:}\\
\begin{enumerate}
\item 
Soit $M = 
\begin{pmatrix}
\frac{4}{5} & \frac{3}{5} \\[3pt]
\frac{3}{5} & \frac{-4}{5}
\end{pmatrix}$.
Montrer que $M\in O(2)$, puis donner les caractéristiques géométriques de $M$ 
(i.e. si $M$ est une rotation d'angle $\theta$, déterminer $\theta$, et si $M$ est une symétrie, déterminer 
l'axe de cette symétrie).
\item 
Donner la matrice de 
$M_2(\R)$ qui représente (dans la base canonique) la symétrie orthogonale d'axe 
$\mathcal{D} = \{ (x,y) \in \R^2 \ \big| \ y=2x \}$.

\end{enumerate}

\bigskip
\noindent
\textbf{Exercice 4:}\\
Soit 
$u=(1,-2,-2)$,  
$v=(-4,5,2)$ et $F$ le sous-espace vectoriel de 
$\R^3$ qu'ils engendrent. On considère $\R^3$ muni 
du produit scalaire usuel.
\begin{enumerate}
\item 
Déterminer une base orthonormée de $F$.
\item 
Déterminer une base orthonormée de $F^{\perp}$.
\item Calculer la projection orthogonale de $(1,2,1)$ sur $F$.
\item 
Donner (dans la basse canonique de $\R^3$), 
la matrice de $s_F$, la symétrie orthogonale par rapport à $F$.
\end{enumerate}







\end{document}


