\documentclass[a4paper, 11pt]{article}
\usepackage[utf8]{inputenc}
\usepackage[francais]{babel}
\usepackage[dvips,final]{graphics}
\usepackage{amsmath,amsfonts,amssymb}
\usepackage{theorem}
\usepackage[T1]{fontenc}

%\usepackage[cp850]{inputenc}
%\usepackage[textures]{epsfig}
%\usepackage{alltt}
%\renewcommand{\baselinestretch}{1,25}
%\psfigdriver{dvips}

\pagestyle{empty}
% my margins

\addtolength{\oddsidemargin}{-.875in}
\addtolength{\evensidemargin}{-.875in}
\addtolength{\textwidth}{1.75in}

\addtolength{\topmargin}{-.875in}
\addtolength{\textheight}{1.75in}
\theoremstyle{plain}
\theorembodyfont{\upshape}
\newtheorem{thm}{Th\'eoreme}
\newtheorem{cor}[thm]{Corollaire}
\newtheorem{lem}[thm]{Lemme}
\newtheorem{prop}[thm]{Proposition}
\newtheorem{defn}[thm]{D\'efinition}
\newtheorem{rem}[thm]{Remarque}
\newtheorem{ex}{Exercice}{\theorembodyfont{\upshape}}


\newcommand{\R}{\mathbb{R}}
\newcommand{\Z}{\mathbb{Z}}
\newcommand{\C}{\mathbb{C}}
\newcommand{\N}{\mathbb{N}}
\newcommand{\T}{\mathbb{T}}
\newcommand{\Q}{\mathbb{Q}}
\newcommand{\D}{\mathbb{D}}
\newcommand{\K}{\mathbb{K}}
\newcommand{\Li}{\mathcal{L}}
\newcommand{\M}{\mathcal{M}}
\newcommand{\F}{\mathbf{F}}
\renewcommand{\S}{\mathbb{S}}

\newcommand{\im}{\textup{Im}}
\newcommand{\BC}{\mathcal{B}}
\newcommand{\CC}{\mathcal{C}}
\newcommand{\DC}{\mathcal{D}}
\newcommand{\can}{_{\mathrm{can}}}

\newcommand{\Mat}{\textup{Mat}}
\newcommand{\Vect}{\mathrm{Vect}}
\newcommand{\rg}{\textup{rg}}
\newcommand{\cm}{\textup{Comat}}

\begin{document}
\noindent
\large
\textbf{Universit\'e Pierre et Marie Curie 
 - LM223 -
Ann\'ee 2012-2013}\\

\begin{center}
\Large
\textbf{Examen final, 16 janvier 2013}
\end{center}
\normalsize

\medskip
\noindent
\textbf{Exercice 1:}\\
\begin{enumerate}
\item
Donner 3 matrices de $O(2) \setminus SO(2)$ (c'est à dire des 
matrices qui sont dans $O(2)$ mais pas dans $SO(2)$). 

\item Montrer que si $P,Q \in O(3)$, alors 
$PQ \in O(3)$.

\end{enumerate}

\bigskip
\noindent
\textbf{Exercice 2:}\\
Soit $f$ l'application linéaire de  $\R^2$ dans $\R^3$ dont la matrice dans la 
base canonique est 
$\begin{pmatrix}
2&1\\
1&0\\
0&2
\end{pmatrix}
$. 
\begin{enumerate}

\item Montrer que Im$(f)$ est un plan de $\R^3$ dont on donnera une équation 
cartésienne.
\item 
Est-ce que 
$v=(1,1,1) \in$ Im$(f)$?
\item Montrer qu'il existe un unique $u \in \R^2$ tel que 
$\| f(u) - v\|$ soit minimale, et déterminer ce $u$. 
\item Donner la matrice (dans la base canonique) de la symétrie orthogonale par rapport à 
Im$(f)$.
\end{enumerate}
\medskip
\noindent
\textbf{Exercice 3:}\\
Soit $q$ la forme quadratique de 
$\R^3$ définie par 
\[ q(x,y,z) = 3x^2 +y^2 +2z^2 +4xz +4yz\]
\begin{enumerate}
\item 
Donner la matrice associée à $q$ dans la base canonique de $\R^3$.
\item 
Déterminer $\mathcal{B}$ une base orthonormée (pour le produit scalaire usuel de $\R^3$) 
qui soit aussi orthogonale pour $q$. 

\item Existe-t-il $u\in \R^3$ tel que 
$q(u)<0$ et si oui donner un tel $u$.

\item 
Montrer que si $v\in \R^3$ et $\|v\| \leq 1$, alors 
$-1 \leq q(v) \leq 5$.

\end{enumerate}

\bigskip
\noindent
\textbf{Exercice 4:}\\
Soit 
\[
\begin{array}{cccc}
f : & \R_2[X]\times \R_2[X] & \to & \R \\
    &  (P,Q)  & \mapsto & \int_{-1}^1 xP(x)Q(x)dx
    \end{array}
    \]
    
\begin{enumerate}
\item 
Montrer brièvement que $f$ est une application bilinéaire symétrique.
\item 
Soit $\mathcal{B} = \{ 1,X,X^2\}$ la base canonique de $ \R_2[X]$. Calculer la matrice $M$ de 
$f$ dans la base $\mathcal{B}$.
\item Calculer la signature de la forme quadratique $q$ associée à $f$.
\end{enumerate}







\end{document}


