\documentclass[a4paper, 11pt]{article}
\usepackage[utf8]{inputenc}
\usepackage[francais]{babel}
\usepackage[dvips,final]{graphics}
\usepackage{amsmath,amsfonts,amssymb}
\usepackage{theorem}
\usepackage[T1]{fontenc}

%\usepackage[cp850]{inputenc}
%\usepackage[textures]{epsfig}
%\usepackage{alltt}
%\renewcommand{\baselinestretch}{1,25}
%\psfigdriver{dvips}

\pagestyle{empty}
% my margins

\addtolength{\oddsidemargin}{-.875in}
\addtolength{\evensidemargin}{-.875in}
\addtolength{\textwidth}{1.75in}

\addtolength{\topmargin}{-.875in}
\addtolength{\textheight}{1.75in}
\theoremstyle{plain}
\theorembodyfont{\upshape}
\newtheorem{thm}{Th\'eoreme}
\newtheorem{cor}[thm]{Corollaire}
\newtheorem{lem}[thm]{Lemme}
\newtheorem{prop}[thm]{Proposition}
\newtheorem{defn}[thm]{D\'efinition}
\newtheorem{rem}[thm]{Remarque}
\newtheorem{ex}{Exercice}{\theorembodyfont{\upshape}}


\newcommand{\R}{\mathbb{R}}
\newcommand{\Z}{\mathbb{Z}}
\newcommand{\C}{\mathbb{C}}
\newcommand{\N}{\mathbb{N}}
\newcommand{\T}{\mathbb{T}}
\newcommand{\Q}{\mathbb{Q}}
\newcommand{\D}{\mathbb{D}}
\newcommand{\K}{\mathbb{K}}
\newcommand{\Li}{\mathcal{L}}
\newcommand{\M}{\mathcal{M}}
\newcommand{\F}{\mathbf{F}}
\renewcommand{\S}{\mathbb{S}}

\newcommand{\im}{\textup{Im}}
\newcommand{\BC}{\mathcal{B}}
\newcommand{\CC}{\mathcal{C}}
\newcommand{\DC}{\mathcal{D}}
\newcommand{\can}{_{\mathrm{can}}}

\newcommand{\Mat}{\textup{Mat}}
\newcommand{\Vect}{\mathrm{Vect}}
\newcommand{\rg}{\textup{rg}}
\newcommand{\cm}{\textup{Comat}}

\begin{document}
\noindent
\large
\textbf{Universit\'e Pierre et Marie Curie 
 - LM223 -
Ann\'ee 2012-2013}\\

\begin{center}
\Large
\textbf{Correction de l'examen final, 16 janvier 2013}
\end{center}
\normalsize

\medskip
\noindent
\textbf{Exercice 1:}\\
\begin{enumerate}
\item
On pouvait prendre par exemple
$\begin{pmatrix}
1&0\\0&-1
\end{pmatrix}
, \begin{pmatrix}
-1&0\\0&1
\end{pmatrix}, 
\begin{pmatrix}
0&1\\1&0
\end{pmatrix}$.
\item Par définition, 
$P\in O(3)$ si et seulement si 
$^tPP=I_3$ (idem pour $Q$ et $PQ$). 
Ainsi, si $P,Q \in O(3)$, 
$^t(PQ)PQ = ^tQ^tPPQ = ^tQ I_3 Q = ^tQQ=I_3$, 
donc $PQ \in O(3)$.

\end{enumerate}

\bigskip
\noindent
\textbf{Exercice 2:}\\

\begin{enumerate}
\item 
Im$(f) =$Vect$( (2,1,0) , (1,0,2))$. Comme ces deux vecteurs sont libres, il s'agit d'un plan de 
$\R^3$. 
De plus, 
$\begin{pmatrix}
2\\1\\0
\end{pmatrix}
\wedge 
\begin{pmatrix}
1\\0\\2
\end{pmatrix}
=\begin{pmatrix}
2\\-4\\-1
\end{pmatrix}$, et on en déduit que 
Im$(f) = \{ (x,y,z) \in \R^3 \ \big| \ 2x-4y-z =0 \}$.

\item Comme 
$2-4-1 \neq 0$, $v\notin $ Im$(f)$.
\item D'après le cours, il existe un unique vecteur $w \in$ Im$(f)$ qui réalise le 
minimum pour 
$\|w-v\|$, et $w$ est la projection orthogonale de $v$ sur 
Im$(f)$. Comme on connaît un vecteur orthogonal à 
Im$(f)$, à savoir 
$(2,-4,-1)$, on sait que 
\[
v = w + (2,-4,-1) \frac{(v| (2,-4,-1))}{\|(2,-4,-1)\|^2} \ \text{soit} \]
\[
(1,1,1) = w + (2,-4,-1)\frac{-1}{7} \]
Ainsi 
$w = (1,1,1) + (\frac{2}{7}, -\frac{4}{7} , -\frac{1}{7}) = (\frac{9}{7},\frac{3}{7},\frac{6}{7})$. \\
Maintenant, comme $f$ réalise une bijection de 
$\R^2$ sur son image à savoir Im$(f)$, on en déduit qu'il existe un unique 
$u\in \R^2$ tel que $f(u)=w$, et que ce $u$ est l'unique vecteur de $\R^2$ à 
atteindre le minimum pour 
$\|f(u)-v\|$, et $u$ est l'unique antécédent de $w$ pour $f$.
Si on pose $u=(u_1,u_2)$, cela nous donne 
$u_1(2,1,0) + u_2(1,0,2) = ( \frac{9}{7},\frac{3}{7},\frac{6}{7})$ soit 
\[ \begin{matrix}
2u_1 & +u_2 & =\frac{9}{7}\\
u_1 &&=\frac{3}{7}\\
 & 2u_2 & =\frac{6}{7}
 \end{matrix}\]
 et on vérifie que la seule solution de ce système est 
 $u=(\frac{3}{7},\frac{3}{7})$.\\
 Au passage, $\|w-u\| = \|( \frac{2}{7}, -\frac{4}{7}, -\frac{1}{7} \| = \frac{\sqrt{21}}{7}$.
\item 
Si on note $s$ cette symétrie, on sait que pour 
$p\in \R^3$, 
$s(p) = p-2 (2,-4,-1) \frac{(p| (2,-4,-1))}{\|(2,-4,-1)\|^2}$. 
En posant 
$p=(x,y,z)$, cela donne \\
\begin{align*}
s(x,y,z) &= (x,y,z) - \frac{(4,-8,-2)}{21}(2x-4y-z) \\
       & = \frac{1}{21} ( 
(21x,21y,21z) + (-4,8,2)(2x-4y-z) )  \\
 &=\frac{1}{21} ( 
(21x,21y,21z) +( -8x+16y+4z,16x-32y-8z, 4x-8y-2z) ) \\
&= \frac{1}{21} ( 13x +16y+4z, 16x -11y-8z, 4x-8y+19z ) 
\end{align*}
La matrice de $s$ est donc 
$\frac{1}{21}
\begin{pmatrix}
13 & 16&4\\
16&-11&-8\\
4&-8&19
\end{pmatrix}$.

\end{enumerate}
\medskip
\noindent
\textbf{Exercice 3:}\\

\begin{enumerate}
\item 
La matrice cherchée est 
$M= 
\begin{pmatrix}
3&0&2\\
0&1&2\\
2&2&2\\
\end{pmatrix}$
\item 
Comme la matrice $M$ est symétrique réelle, elle est diagonalisable dans une base orthonormée, 
et on va chercher $P \in O(3)$ telle que 
$P^{-1}MP$ soit diagonale. Comme $P^{-1} = {^tP}$, la base associée à $P$ conviendra alors. 
On a \\ 
$\chi_M (X) = 
\begin{vmatrix}
3-X & 0&2\\
0&1-X & 2 \\
2&2&2-X
\end{vmatrix} 
= (3-X) ( (1-X)(2-X) -4) +2(-(1-X)2) $ \\
$=
(3-X)(X^2-3X -2) +4X -4 
=-X^3 +6X^2 -3X -10$. \\
On constate que $-1$ est racine, et on en déduit que 
$\chi_M(X) = -(X+1)(X^2-7X+10)$. \\
Enfin après factorisation, on trouve que  
$X^2-7X+10 = (X-2)(X-5)$, ainsi, 
$\chi_M(X) =-(X+1)(X-2)(X-5)$ et $M$ a trois valeurs propres distinctes qui sont 
$-1,2,5$.\\ 
Après calcul, on trouve que les espaces propres associés sont \\
$E_{-1} = $ Vect$(1,2,-2)$ , $E_2=$ Vect$(-2,2,1)$, $E_5=$ Vect$(2,1,2)$. \\
Ce sont des vecteurs de norme $3$, et on en déduit que 
$\mathcal{B}= 
\{ ( \frac{1}{3},\frac{2}{3},-\frac{2}{3} ) , (-\frac{2}{3}, \frac{2}{3}, \frac{1}{3}) , 
(\frac{2}{3}, \frac{1}{3}, \frac{2}{3} ) \}$ est une BON de vecteurs propres de $M$. 
Ainsi si $P$ est la matrice associée à $\mathcal{B}$, i.e. 
$P= 
\begin{pmatrix}
\frac{1}{3} &-\frac{2}{3} &\frac{2}{3} \\
\frac{2}{3} & \frac{2}{3} & \frac{1}{3} \\ 
-\frac{2}{3} & \frac{1}{3} & \frac{2}{3}
\end{pmatrix}$, 
alors $^tP=P^{-1}$ et donc 
$P^{-1}MP = ^tPMP = 
\begin{pmatrix}
-1&0&0\\
0&2&0 \\
0&0&5
\end{pmatrix}$, 
et donc $\mathcal{B}$ est une base orthonormée qui est orthogonale pour $q$.

\item D'après ce qu'on a fait à la question précédente, on sait 
que 
$q( ( \frac{1}{3}, \frac{2}{3}, -\frac{2}{3}) ) =-1$, la réponse est donc oui, avec comme 
possibilité 
$u=(\frac{1}{3}, \frac{2}{3}, -\frac{2}{3})$. 

\item 
Soit 
$x',y',z'$ les coordonnées de $v$ dans la base 
$\mathcal{B}$.
Alors 
$\|v\|^2 = x'^2 +y'^2 +z'^2$ car 
$\mathcal{B}$ est une BON. 
Par ailleurs, 
$q(v) = -x'^2 +2y'^2 +5z'^2$ d'après la question $2)$. 
Ainsi si $\|v\| \leq 1$, 
$x'^2 +y'^2+z'^2 \leq 1$ et
 \[-1 \leq -(x'^2+y'^2+z'^2) \leq q(v) =  -x'^2 +2y'^2 +5y'^2 \leq 5(x'^2+y'^2+z'^2) \leq 5\]

\end{enumerate}

\bigskip
\noindent
\textbf{Exercice 4:}\\
Soit 
\[
\begin{array}{cccc}
f : & \R_2[X]\times \R_2[X] & \to & \R \\
    &  (P,Q)  & \mapsto & \int_{-1}^1 xP(x)Q(x)dx
    \end{array}
    \]
    
\begin{enumerate}
\item 
Par linéarité de l'integrale 
$f(\lambda P_1 +P_2,Q) = 
\lambda f(P_1,Q) + f(P_2,Q)$ et de plus 
$f(P,Q) = f(Q,P)$.
\item 
On calcule les valeurs prises par $f$ sur les éléments de cette base. On trouve : 
\[
\begin{matrix}
f(1,1) & = \int^1_{-1}xdx   &=0 \\
f(1,X) & = \int^1_{-1} x^2dx   &=\frac{2}{3} \\
f(1,X^2) & = \int^1_{-1}x^3dx   &=0 \\
f(X,X) & = \int^1_{-1}x^3dx   &=0 \\
f(X,X^2) & = \int^1_{-1}x^4dx   &= \frac{2}{5} \\
f(X^2,X^2) & = \int^1_{-1}x^5dx   &= 0
\end{matrix} \]
On en déduit que 
$M = 
\begin{pmatrix}
0 & \frac{2}{3} & 0 \\
\frac{2}{3} & 0 & \frac{2}{5} \\
0& \frac{2}{5} & 0
\end{pmatrix}$.
\item En posant $a,b,c$ les coordonnées dans la base 
$\mathcal{B}$, la forme quadratique s'écrit donc 
$q(a,b,c) = 
\frac{4}{3} ab + \frac{4}{5}bc$. \\
On peut calculer la signature de $q$ sans faire trop de calculs. 
Tout d'abord, on remarque facilement que $\det (M)=0$ (par exemple car la première et la troisième colonnes 
sont colinéaires). Par ailleurs les deux premières colonnes de $M$ sont libres, ainsi 
rang$(q)=$rang$(M) =2$. Ainsi la signature ne peut être que 
$(2,0), (1,1)$ ou $(0,2)$. 
Cependant, en prenant $a,b,c>0$, on voit que $q(a,b,c) >0$, ainsi la signature 
$(0,2)$ est impossible (car dans ce cas, $q$ serait négative). 
De même, si on prend $a,c>0$ et $b<0$, alors 
$q(a,b,c) <0$, ainsi la signature $(2,0)$ est également impossible. 
C'est donc que la signature de $q$ est $(1,1)$.  \\
On aurait aussi pu calculer la signature en mettant la forme $q$ sous forme de carrées linéairement indépendants. 
Par exemple en posant 
$a'=a+b, b'=a-b$ et $c'=c$, on obtient que 
$q =  \frac{1}{3} ( a' + \frac{3}{5}c')^2 - \frac{1}{3} ( b'+ \frac{3}{5}c')^2$, ce qui nous 
redonne le même résultat.
\end{enumerate}







\end{document}


