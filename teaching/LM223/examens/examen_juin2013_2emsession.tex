\documentclass[a4paper, 11pt]{article}
\usepackage[latin1]{inputenc}
\usepackage[francais]{babel}
\usepackage[latin1]{inputenc}
\usepackage[dvips,final]{graphics}
\usepackage{amsmath,amsfonts,amssymb}
\usepackage{theorem}
\usepackage[T1]{fontenc}
%\usepackage[cp850]{inputenc}
%\usepackage[textures]{epsfig}
%\usepackage{alltt}
%\renewcommand{\baselinestretch}{1,25}
%\psfigdriver{dvips}

\pagestyle{empty}
% my margins

\addtolength{\oddsidemargin}{-.875in}
\addtolength{\evensidemargin}{-.875in}
\addtolength{\textwidth}{1.75in}

\addtolength{\topmargin}{-.875in}
\addtolength{\textheight}{1.75in}
\theoremstyle{plain}
\theorembodyfont{\upshape}
\newtheorem{thm}{Th\'eoreme}
\newtheorem{cor}[thm]{Corollaire}
\newtheorem{lem}[thm]{Lemme}
\newtheorem{prop}[thm]{Proposition}
\newtheorem{defn}[thm]{D\'efinition}
\newtheorem{rem}[thm]{Remarque}
\newtheorem{ex}{Exercice}{\theorembodyfont{\upshape}}

\title{Fiche 1\\Espaces Vectoriels, Applications lin�aires}
\newcommand{\R}{\mathbb{R}}
\newcommand{\Z}{\mathbb{Z}}
\newcommand{\C}{\mathbb{C}}
\newcommand{\N}{\mathbb{N}}
\newcommand{\T}{\mathbb{T}}
\newcommand{\Q}{\mathbb{Q}}
\newcommand{\D}{\mathbb{D}}
\newcommand{\K}{\mathbb{K}}
\newcommand{\Li}{\mathcal{L}}
\newcommand{\M}{\mathcal{M}}
\newcommand{\F}{\mathbf{F}}
\renewcommand{\S}{\mathbb{S}}

\newcommand{\im}{\textup{Im}}
\newcommand{\BC}{\mathcal{B}}
\newcommand{\CC}{\mathcal{C}}
\newcommand{\DC}{\mathcal{D}}
\newcommand{\can}{_{\mathrm{can}}}

\newcommand{\Mat}{\textup{Mat}}
\newcommand{\Vect}{\mathrm{Vect}}
\newcommand{\rg}{\textup{rg}}
\newcommand{\cm}{\textup{Comat}}

\begin{document}
\noindent
\large
\textbf{Universit\'e Pierre et Marie Curie 
 - LM223 -
Ann\'ee 2012-2013}\\

\begin{center}
\Large
\textbf{Examen de 2� session, p�riode 1}
\end{center}
\normalsize

\medskip
\noindent
\textbf{Question de cours :}\\
\begin{enumerate}
\item 
Donner la d�finition d'une forme bilin�aire sym�trique ainsi que celle d'un produit scalaire.
\item Sur $\R^3$ donner quatre formes quadratiques $q_1, q_2, q_3$ et $q_4$ diff�rentes 
telles que $q_1$ et $q_2$ soient des produits scalaires, 
$q_3$ soit de signature $(2,1)$ et $q_4$ soit d�g�n�r�e.
\item Donner la d�finition d'une matrice orthogonale.
\end{enumerate}

\bigskip
\noindent

\textbf{Exercice 1:}\\
\begin{enumerate}
\item
Montrer que si $P\in O(n)$, alors $\det (P) = \pm 1$. 

\item Donner quatre matrices de $SO(2)$.

\item 
Compl�ter la matrice suivante $P$ pour que $P \in SO(3)$ 
o�
$P = 
\begin{pmatrix}
\frac{2}{3} & \cdot & \cdot \\[3pt]
\frac{-1}{3} & \cdot & \cdot \\[3pt]
\frac{2}{3} & \cdot & \cdot 
\end{pmatrix}$.
\end{enumerate}

\medskip
\noindent

\textbf{Exercice 2:}\\
Sur $\R^3$ soit $q$ la forme quadratique d�finie par 
$q(x) = x_1^2 +7x_2^2 +12x_3^2 +4x_1x_2 -2x_1x_3 -16x_2x_3$. 
\begin{enumerate}
\item Donner la matrice $M$ de $q$ dans la base canonique de $\R^3$.
\item Donner une base orthogonale pour $q$.
\item Quelle est la signature de $q$?
\item Trouver un �l�ment $x\in \Z^3$ tel que $q(x) =-1$.
\end{enumerate}


\bigskip
\noindent

\textbf{Exercice 3:}\\
Soit 
$M = 
\begin{pmatrix}
2&0&-1\\
0&2&1\\
-1&1&3
\end{pmatrix}
$. 
\begin{enumerate}
\item 
Trouver une matrice $P\in O(3)$ telle que 
$P^{-1}MP$ soit diagonale.
\item Soit $q$ la forme quadratique associ�e � $M$. Donner l'expression de $q$.
\item Est-ce que $q$ est d�finie positive? 

\end{enumerate}

\end{document}


