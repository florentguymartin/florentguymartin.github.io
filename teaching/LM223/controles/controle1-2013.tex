\documentclass[a4paper, 11pt]{article}
\usepackage[latin1]{inputenc}
\usepackage[francais]{babel}
\usepackage[latin1]{inputenc}
\usepackage[dvips,final]{graphics}
\usepackage{amsmath,amsfonts,amssymb}
\usepackage{theorem}
\usepackage[T1]{fontenc}
%\usepackage[cp850]{inputenc}
%\usepackage[textures]{epsfig}
%\usepackage{alltt}
%\renewcommand{\baselinestretch}{1,25}
%\psfigdriver{dvips}

\pagestyle{empty}
% my margins

\addtolength{\oddsidemargin}{-.875in}
\addtolength{\evensidemargin}{-.875in}
\addtolength{\textwidth}{1.75in}

\addtolength{\topmargin}{-.875in}
\addtolength{\textheight}{1.75in}
\theoremstyle{plain}
\theorembodyfont{\upshape}
\newtheorem{thm}{Th\'eoreme}
\newtheorem{cor}[thm]{Corollaire}
\newtheorem{lem}[thm]{Lemme}
\newtheorem{prop}[thm]{Proposition}
\newtheorem{defn}[thm]{D\'efinition}
\newtheorem{rem}[thm]{Remarque}
\newtheorem{ex}{Exercice}{\theorembodyfont{\upshape}}

\title{Fiche 1\\Espaces Vectoriels, Applications lin�aires}
\newcommand{\R}{\mathbb{R}}
\newcommand{\Z}{\mathbb{Z}}
\newcommand{\C}{\mathbb{C}}
\newcommand{\N}{\mathbb{N}}
\newcommand{\T}{\mathbb{T}}
\newcommand{\Q}{\mathbb{Q}}
\newcommand{\D}{\mathbb{D}}
\newcommand{\K}{\mathbb{K}}
\newcommand{\Li}{\mathcal{L}}
\newcommand{\M}{\mathcal{M}}
\newcommand{\F}{\mathbf{F}}
\renewcommand{\S}{\mathbb{S}}

\newcommand{\im}{\textup{Im}}
\newcommand{\BC}{\mathcal{B}}
\newcommand{\CC}{\mathcal{C}}
\newcommand{\DC}{\mathcal{D}}
\newcommand{\can}{_{\mathrm{can}}}

\newcommand{\Mat}{\textup{Mat}}
\newcommand{\Vect}{\mathrm{Vect}}
\newcommand{\rg}{\textup{rg}}
\newcommand{\cm}{\textup{Comat}}

\begin{document}
\noindent
\large
\textbf{Universit\'e Pierre et Marie Curie 
 - LM223 -
Ann\'ee 2012-2013}\\

\begin{center}
\Large
\textbf{Interro n$^o$ 1}
\end{center}
\normalsize

\medskip
\noindent
\textbf{Exercice 1:}\\
Calculer le d\'eterminant de la matrice suivante. 
$$
M= \left(
\begin{array}{ccc}
0&1&2\\[5pt]
1&1&2\\[5pt]
0&2&3
\end{array}
\right)
$$
En d\'eduire que $M$ est inversible et calculer son inverse.\\


\bigskip
\noindent
\textbf{Exercice 2:}\\
On consid\`ere le sous-ensemble de matrices r\'eelles $2\times 2$ suivant:
$$
\F=\left\{\begin{pmatrix} a&b\\ 
-b&a \end{pmatrix},\; a,b\in \R\right\}.
$$
\begin{enumerate}
\item Montrer que $\F$ est un sous-espace vectoriel de $M_2(\R)$. Quelle est la dimension de $\F$?\\

On  consid\`ere $\C$ comme un espace vectoriel r\'eel, et on d\'efinit une application $f:\F\rightarrow \C$ par
$$
f:\begin{pmatrix} a&b\\ 
-b&a \end{pmatrix}
\mapsto
a+ib
$$
\item Montrer que $f$ est lin\'eaire.
\item Montrer que $f$ est bijective.
\item Montrer que pour toutes $M,N \in \F$, le produit $MN $ est dans $\F$. 
\item Montrer que $f(MN) =f(M)f(N)$ pour toutes $M,N\in \F$.
\end{enumerate}


\medskip
\noindent
\textbf{Exercice 3:}\\
%On consid\`ere l'endomorphisme dont la matrice dans la base canonique est
Soit $\displaystyle M = \begin{pmatrix}-1&1&1\\ 1&-1&1\\ 1&1&-1\end{pmatrix} . $
\begin{enumerate}
\item Calculer le polyn\^ome caract\'eristique $P_M(X)$.
\item Quelles sont les valeurs propres de $M$? Donner des bases des sous-espaces propres associ\'es.
\item $M$ est-elle diagonalisable? Si oui, donner une matrice $P$ telle que $P^{-1}MP$ soit diagonale.
\end{enumerate}

\bigskip
\noindent
\textbf{Exercice 4:}\\
Soit $F= \{ (x,y,z,t) \in \C^4 \ \big| \ x+y+z+t=0 \}$.
\begin{enumerate}
\item 
Montrer que $F$ est un sous-espace vectoriel de $\C^4$ et donner, en le justifiant, 
sa dimension. 
\item 
Soit $e_1 = (1,0,0,-1)$, $e_2=(0,1,0,-1)$, $e_3=(0,0,1,-1)$ et  
$\mathcal{B} = \{e_1,e_2,e_3\}$. Montrer que $\mathcal{B}$ est une base de $F$.
\item 
Soit 
\[ \begin{array}{rrcl}
\varphi :& F & \rightarrow & F \\
         & (x,y,z,t) & \mapsto & (x+3y+t, 3x-2y , -2x+2y+z, -y+z+t) 
 \end{array} \]
Montrer que $\varphi$ est bien une application de $F$ dans $F$, et qu'elle est lin�aire.
Puis calculer $M=$Mat$_{\mathcal{B}}(\varphi)$.
\item Est-ce que $\varphi $ est diagonalisable? Si oui donner une base de vecteurs propres de $\varphi$.
 
\end{enumerate}

\bigskip
\noindent
\textbf{Exercice 5:}\\

Soit $A \in M_3(\C)$ une matrice diagonalisable.
$\mathcal{C}_A= \{M \in M_3(\mathbb{C}) \ \big| \ 
MA=AM \}$
\begin{enumerate}
\item 
Montrer que $\mathcal{C}_A$ est un sous-espace vectoriel de $M_3(\mathbb{C})$. 
\item 
Soit $\lambda$ une valeur propre de $A$, $E_{\lambda}$ l'espace propre associ�.
Si $u \in E_{\lambda}$, montrer que $Mu \in E_{\lambda}$.
\item 
En d�duire la dimension de $\mathcal{C}_A$.\\
\emph{Attention, la dimension de $\mathcal{C}_A$ d�pend de la dimension des espaces propres de $A$, 
et n'est pas la m�me pour toutes les matrices diagonalisables. Dans votre r�ponse il faut donc distinguer plusieurs cas.}
\item
Soit $D = \begin{pmatrix}
a &0 &0 \\ 0 & b & 0 \\ 0 & 0 &c
\end{pmatrix} \in M_3(\mathbb{C})$, avec $a,b,c$ trois nombres complexes distincts.\\
Soit $\mathcal{C}_D= \{M \in M_3(\mathbb{C}) \ \big| \ 
MD=DM \}$. 
Donner une base de $\mathcal{C}_D$.

\end{enumerate}





\end{document}


