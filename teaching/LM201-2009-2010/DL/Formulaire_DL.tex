\documentclass[a4paper, 10pt]{article}

\usepackage[french]{babel}
\usepackage[T1]{fontenc}
\usepackage{textcomp}
\usepackage{amsmath, amsfonts, amsthm, amssymb}
\usepackage{enumerate}



\begin{document}

\begin{flushleft}
Universit� Paris 6 Pierre et Marie Curie\\
Ann�e 2009-2010\\
LM110 - Fonctions\\
\vspace{10mm}
C. Dupont
\end{flushleft}

\vspace{10mm}

\begin{center}
\textbf{\Large{D�veloppements limit�s usuels}}
\end{center}
~~\par
Tous les d�veloppements limit�s sont donn�s en $0$. 

Dans la derni�re �galit�, $\alpha$ est un r�el quelconque.
\vspace{5mm}

$$\exp(x)=1+x+\frac{x^2}{2!}+\frac{x^3}{3!}+...+\frac{x^n}{n!}+\circ (x^n)$$\\
$$\sin(x)=x-\frac{x^3}{3!}+\frac{x^5}{5!}-\frac{x^7}{7!}+...+(-1)^n \frac{x^{2n+1}}{(2n+1)!}+\circ (x^{2n+2})$$\\
$$\cos(x)=1-\frac{x^2}{2!}+\frac{x^4}{4!}-\frac{x^6}{6!}+...+(-1)^n \frac{x^{2n}}{(2n)!}+\circ (x^{2n+1})$$\\
$$\frac{1}{1-x}=1+x+x^2+x^3+...+x^n+\circ (x^n)$$\\
$$\frac{1}{1+x}=1-x+x^2-x^3+...+(-1)^n x^n+\circ (x^n)$$\\
$$\ln(1-x)=-x-\frac{x^2}{2}-\frac{x^3}{3}-\frac{x^4}{4}-...-\frac{x^n}{n}+\circ (x^n)$$\\
$$\ln(1+x)=x-\frac{x^2}{2}+\frac{x^3}{3}-\frac{x^4}{4}+...+(-1)^{n+1}\frac{x^n}{n}+\circ (x^n)$$\\
$$(1+x)^\alpha=1+\alpha x +\frac{\alpha(\alpha-1)}{2!}x^2+...+\frac{\alpha(\alpha-1)...(\alpha-n+1)}{n!}x^n +\circ (x^n)$$
\end{document}
