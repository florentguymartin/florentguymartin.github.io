\documentclass{article}


\usepackage[latin1]{inputenc}
\usepackage[francais]{babel}
\usepackage{amsmath,amssymb,amsfonts}
\usepackage[T1]{fontenc}




\newtheorem{Theoreme}{Th{\'e}or{\`e}me}[section]
\newtheorem{Definition}{D{\'e}finition}[section]
\newtheorem{Prop}{Proposition}[section]
\newtheorem{Lemme}{Lemme}[section]

\begin{document}



\renewcommand{\labelitemi}{$\circ$}

\large{Exercice 1} : \\
Donner le DL {\`a} l'ordre 4 en 0 de \\
$$\frac{\cos (x)}{1+x+x^2}$$

\vspace{5mm}

\large{Exercice 2} : \\
On rappelle que $f : \mathbb{R} \rightarrow \mathbb{R}$ est d{\'e}rivable
si et seulement si $f$ a un DL {\`a} l'ordre 1 en 0 : $f(x) = a +xb +\circ
(x)$ , et dans ce cas $f'(0)=b$.\\
Soit $f : \mathbb{R} \rightarrow \mathbb{R}$ d{\'e}finie par 
$$f(x) = \left\{ 
\begin{array}{ll}
\frac{x}{e^x-1} & \text{ si} \ x \neq 0 \\
1 & \text{si} \ x=0
\end{array}\right. $$
Montrer que $f$ est d{\'e}rivable sur $\mathbb{R}$ et calculer $f'(0)$.


\vspace{5mm}

\large{Exercice 3} : \\
Calculer 
$$\int_5^6 \frac{x\sqrt{x} +1}{x+1}dx$$



\vspace{5mm}

\large{Exercice 4} : \\
D{\'e}composer en {\'e}l{\'e}ment simple la fraction rationelle :
$$\frac{X+2}{X^3-2X^2+X-2}$$
(indication : on pourra d{\'e}j{\`a} essayer de trouver une racine pas trop
compliqu{\'e}e du d{\'e}nominateur)

\vspace{5mm}

\large{Exercice 5} :\\
Soit $f : \mathbb{R}_+ \rightarrow \mathbb{R}$, une fonction
continue telle que $f(0)=0$ et
$f(x) \xrightarrow[x \to +  \infty]{} 0$. Montrer que $f$ est major{\'e}e sur
$\mathbb{R}_+$ et y atteint son maximum (c'est {\`a} dire qu'il existe
$x_1 \in \mathbb{R}_+$ tel que $\forall x \in \mathbb{R}_+ \ f(x)
\leq f(x_1)$).

\vspace{5mm}

\large{Exercice 6} : \\
Trouver la solution de l'{\'e}quation diff{\'e}rentielle :\\
$y'(x) - y(x) =e^x+e^{2x}$ \ telle que $y(0)=2$.

\bibliographystyle{alpha}
\bibliography{bibli}





\end{document}
