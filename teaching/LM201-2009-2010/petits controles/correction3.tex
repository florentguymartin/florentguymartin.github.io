\documentclass{article}


\usepackage[latin1]{inputenc}
\usepackage[francais]{babel}
\usepackage{amsmath,amssymb}
\usepackage[T1]{fontenc}


\begin{document}




\title{correction}
\maketitle

\renewcommand{\labelitemi}{*}


\textbf{Exercice 1} : \\
Donner le DL � l'ordre $3$ en $0$ de :
$$\text{e}^{(\text{e}^x\text{sin}(x))}$$
A l'ordre 3 en $0$ on a : \\
$\text{e}^x\text{sin}(x) = (1+x+\frac{x^2}{2}+\frac{x^3}{6})(x-\frac{x^3}{6})+\circ (x^3) = \\
x + x^2 + \frac{x^3}{2} - \frac{x^3}{6} + \circ (x^3) = x + x^2 +\frac{x^3}{3} + \circ (x^3)$\\
Comme cette expression tend vers $0$ on peut r�utiliser le DL de $\text{e}^x$ en $0$ : \\
$\text{e}^{(\text{e}^x\text{sin}(x))}=1 + x + x^2 +\frac{x^3}{3} + \frac{(x + x^2 +\frac{x^3}{3})^2}{2} + \frac{(x + x^2 +\frac{x^3}{3})^3}{6} + \circ (x^3)=$\\
$1+x+x^2+ \frac{x^3}{3} + \frac{x^2}{2} + x^3 + \frac{x^3}{6} + \circ (x^3) = 1+x+\frac{3x^2}{2} + \frac{3x^3}{2} + \circ (x^3) $


\vspace{10mm}

\textbf{Exercice 2} : \\
Calculer l'int�grale suivante :

$$\int_1^2 \frac{1}{x^2-2x+2}dx$$

On a � integrer l'inverse d'un polynome du second degr�. Il y a donc trois possibilit�s :
\begin{itemize}

\item le polynome a deux racines r�elles distinctes $a$ et $b$, auquel cas l'expression est (� une constante pr�s)
de la forme $\frac{1}{(x-a)(x-b)}$, et on doit pouvoir d�composer cette fraction rationelle sous la forme
$\frac{\alpha}{x-a}+\frac{\beta}{x-b}$ et on sait trouver une primitive maintenant.

\item ou bien le polynome a une racine multiple, disons $a$, et donc � une constante pr�s, l'expression qu'on cherche � integrer est de la forme
$\frac{1}{(x-a)^2}$. Mais l� on a une primitive pas dure � trouver : $-\frac{1}{x-a}$.

\item dernier cas, le polynome n'a pas de racines r�elles (i.e. a un discriminant $<0$). Dans ce cas, � une constante multiplicative pr�s, 
on a une expression de la forme
$\frac{1}{x^2+cx+d}=\frac{1}{(x+\frac{c}{2})^2+d-\frac{c^2}{4}}$.\\
Le terme $d-\frac{c^2}{4}$ est en fait $\frac{-\Delta}{4}$ donc est strictement positif. \\
 Puis on fait le changement de variable $u=x+\frac{c}{2}$, qui revient donc � 
$x=u-\frac{c}{2}$, donc donne $dx=du$, et on est ramen� � integrer $\frac{1}{u^2+r}$ o� $r=\frac{-\Delta}{4}$ est $>0$. L� on fait tout pour se retrouver avec une expression de la forme 
$\frac{1}{1+w^2}$ dont on connait une primitive : Arctan. Or il suffit de faire :\\
$\frac{1}{u^2+r} = \frac{1}{r} \frac{1}{\frac{u^2}{r}+1} = \frac{1}{r} \frac{1}{(\frac{u}{\sqrt{r}})^2+1}$. \\
Donc on fait le changement de variable $w=\frac{u}{\sqrt{r}}$ qui donne $du=\sqrt{r}dw$. 
\end{itemize}
\vspace{8mm}
Dans le cas qui nous concerne, $I=\int_1^2 \frac{dx}{x^2-2x+2}$ on est dans le troisi�me cas, i.e le polynome n'a pas de racines r�elles, et on arrive � :\\
$x^2-2x+2=(x-1)^2+1$ d'o� $I=\int_1^2\frac{1}{(x-1)^2+1} dx$ .\\
On fait donc le changement de variables $x-1=u$ , i.e $x=1+u$, qui donne $dx=du$ 
et comme dans l'integrale $I=\int_1^2 ...$ $x$ variait entre $1$ et $2$, $u$ (qui est �gal � $x-1$) doit varier entre 0 et 1, d'o� \\
$I=\int_0^1\frac{1}{u^2+1}du = [\text{Arctan}]_0^1 = \frac{\pi}{4}-0=\frac{\pi}{4}$.





\bibliographystyle{alpha}
\bibliography{bibli}





\end{document}
