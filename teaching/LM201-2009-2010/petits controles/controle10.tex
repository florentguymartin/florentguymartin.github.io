\documentclass{article}


\usepackage[latin1]{inputenc}
\usepackage[francais]{babel}
\usepackage{amsmath,amssymb,amsfonts}
\usepackage[T1]{fontenc}




\newtheorem{Theoreme}{Th{\'e}or{\`e}me}[section]
\newtheorem{Definition}{D{\'e}finition}[section]
\newtheorem{Prop}{Proposition}[section]
\newtheorem{Lemme}{Lemme}[section]

\begin{document}

\title{Controle 9}
\maketitle
\renewcommand{\labelitemi}{$\circ$}
\subsection*{Exercice 1}
Soit $n\in \mathbb{N}$. Montrer que $\sum_{k=0}^n \binom{n}{k}^2 =
\binom{2n}{n}$. (indication : consid{\'e}rer le polynome $(1+X)^n(1+X)^n=(1+X)^{2n}$.)

\subsection*{Exercice 2}
Soit $P$ un polynome r{\'e}el paire, c'est {\`a} dire que pour tout $x\in
\mathbb{R}$ \ $P(x) = P(-x)$. Alors montrer qu'il existe une polynome
$Q$ tel que $P(X)= Q(X^2)$.






\bibliographystyle{alpha}
\bibliography{bibli}





\end{document}
