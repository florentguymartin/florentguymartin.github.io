\documentclass{article}


\usepackage[latin1]{inputenc}
\usepackage[francais]{babel}
\usepackage{amsmath}
\usepackage{amssymb}
\usepackage[T1]{fontenc}


\begin{document}


\textbf{Exercice1} : Donner un d�veloppement limit� en $0$ � l'ordre $3$ de : 
$$\frac{1}{e^x + x} - ln(1+sin(x))$$


\par

\textbf{Exercice 2} : Donner une expression simplifi�e de : 
$$\prod_{k=0}^n e^{ \frac{2ik \pi}{n}}$$
O� $n\in \mathbb{N}^* $ et par simplifi�e, on veut dire, une expression o� on n'a pas de symbole $\sum$ ou $\prod$.



\textbf{Exercice1} : Donner un d�veloppement limit� en $0$ � l'ordre $3$ de : 
$$\frac{1}{e^x + x} - ln(1+sin(x))$$


\par

\textbf{Exercice 2} : Donner une expression simplifi�e de : 
$$\prod_{k=0}^n e^{ \frac{2ik \pi}{n}}$$
O� $n\in \mathbb{N}^* $ et par simplifi�e, on veut dire, une expression o� on n'a pas de symbole $\sum$ ou $\prod$. 


\par


\textbf{Exercice1} : Donner un d�veloppement limit� en $0$ � l'ordre $3$ de : 
$$\frac{1}{e^x + x} - ln(1+sin(x))$$


\par
\textbf{Exercice 2} : Donner une expression simplifi�e de : 
$$\prod_{k=0}^n e^{ \frac{2ik \pi}{n}}$$
O� $n\in \mathbb{N}^* $ et par simplifi�e, on veut dire, une expression o� on n'a pas de symbole $\sum$ ou $\prod$.

\par



\textbf{Exercice1} : Donner un d�veloppement limit� en $0$ � l'ordre $3$ de : 
$$\frac{1}{e^x + x} - ln(1+sin(x))$$


\par
\textbf{Exercice 2} : Donner une expression simplifi�e de : 
$$\prod_{k=0}^n e^{ \frac{2ik \pi}{n}}$$
O� $n\in \mathbb{N}^* $ et par simplifi�e, on veut dire, une expression o� on n'a pas de symbole $\sum$ ou $\prod$.






\bibliographystyle{alpha}
\bibliography{bibli}





\end{document}
