\documentclass{article}


\usepackage[latin1]{inputenc}
\usepackage[francais]{babel}
\usepackage{amsmath,amssymb,amsfonts}
\usepackage[T1]{fontenc}




\newtheorem{Theoreme}{Th{\'e}or{\`e}me}[section]
\newtheorem{Definition}{D{\'e}finition}[section]
\newtheorem{Prop}{Proposition}[section]
\newtheorem{Lemme}{Lemme}[section]

\begin{document}

\title{Controle 4}
\maketitle
\renewcommand{\labelitemi}{$\circ$}

\Large{Exercice 1} :\\
Calculer 
$$I = \int_4^5 \frac{1}{u^4-1} du$$

\vspace{8mm}

\Large{Exercice 2} : \\
Soit $(u_n)_{n\in \mathbb{N}}$ une suite telle que $u_0>0$ et $\forall
n \in \mathbb{N} \ u_{n+1} \geq 2 u_n$. Montrer que $\frac{u_n}{n} \xrightarrow[n
\to \infty]{} +  \infty$.

\vspace{15mm}

Controle 5 \\
\Large{Exercice 1} : \\
1) Soit $(u_n)_{n \in \mathbb{N} }$ la suite d{\'e}finie par $u_0=1$ et
$u_1=4$ et v{\'e}rifiant pour tout $n \in \mathbb{N}$
$u_{n+2}=5u_{n+1}-6u_n$. Calculer une expression pour $u_n$ et montrer
que $\forall n\in \mathbb{N} \ u_n>0$.
\vspace{8mm}
2)Montrer que $\frac{\ln (u_n)}{n}$ a une limite en $+ \infty$ et la
calculer.


\vspace{15mm}
Controle 6\\
\Large{Exercice 1} : \\
Soit $u_n$ la suite d{\'e}finie par, $u_0=-5$, $u_1=-8$ et $\forall n\in
\mathbb{N} \ u_{n+2} = 4(u_{n+1}-u_n)$. Montrer que $\forall n\geq 4$
on a  $u_{n+1}>u_n$.
\vspace{8mm}
 
\Large{Exercice 2 } : \\
Soit (E) l'{\'e}quation diff{\'e}rentielle $y'+2y=2x^2-1$. Trouver la solution
de (E) telle que $y(0)=1$.





\bibliographystyle{alpha}
\bibliography{bibli}





\end{document}
