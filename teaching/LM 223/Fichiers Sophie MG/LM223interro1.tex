\documentclass[a4paper, 11pt]{article}
\usepackage[francais]{babel}
\usepackage[latin1]{inputenc}
\usepackage[dvips,final]{graphics}
\usepackage{amsmath,amsfonts,amssymb}
\usepackage{theorem}
\usepackage[T1]{fontenc}
%\usepackage[cp850]{inputenc}
%\usepackage[textures]{epsfig}
%\usepackage{alltt}
%\renewcommand{\baselinestretch}{1,25}
%\psfigdriver{dvips}

\pagestyle{empty}
% my margins

\addtolength{\oddsidemargin}{-.875in}
\addtolength{\evensidemargin}{-.875in}
\addtolength{\textwidth}{1.75in}

\addtolength{\topmargin}{-.875in}
\addtolength{\textheight}{1.75in}
\theoremstyle{plain}
\theorembodyfont{\upshape}
\newtheorem{thm}{Th\'eoreme}
\newtheorem{cor}[thm]{Corollaire}
\newtheorem{lem}[thm]{Lemme}
\newtheorem{prop}[thm]{Proposition}
\newtheorem{defn}[thm]{D\'efinition}
\newtheorem{rem}[thm]{Remarque}
\newtheorem{ex}{Exercice}{\theorembodyfont{\upshape}}

\title{Fiche 1\\Espaces Vectoriels, Applications lin�aires}
\newcommand{\R}{\mathbb{R}}
\newcommand{\Z}{\mathbb{Z}}
\newcommand{\C}{\mathbb{C}}
\newcommand{\N}{\mathbb{N}}
\newcommand{\T}{\mathbb{T}}
\newcommand{\Q}{\mathbb{Q}}
\newcommand{\D}{\mathbb{D}}
\newcommand{\K}{\mathbb{K}}
\newcommand{\Li}{\mathcal{L}}
\newcommand{\M}{\mathcal{M}}
\newcommand{\F}{\mathbf{F}}
\renewcommand{\S}{\mathbb{S}}

\newcommand{\im}{\textup{Im}}
\newcommand{\BC}{\mathcal{B}}
\newcommand{\CC}{\mathcal{C}}
\newcommand{\DC}{\mathcal{D}}
\newcommand{\can}{_{\mathrm{can}}}

\newcommand{\Mat}{\textup{Mat}}
\newcommand{\Vect}{\mathrm{Vect}}
\newcommand{\rg}{\textup{rg}}
\newcommand{\cm}{\textup{Comat}}

\begin{document}
\noindent
\large
\textbf{Universit\'e Pierre et Marie Curie 
 - LM223 -
Ann\'ee 2012-2013}\\

\begin{center}
\Large
\textbf{Interro n$^o$ 1}
\end{center}
\normalsize

\medskip
\noindent
\textbf{Exercice 1:}\\
On consid\`ere l'espace vectoriel $\R_2[X]$ des  polyn\^omes r\'eels de degr\'e inf\'erieur ou
\'egal \`a 2. 
On note $\BC=\{1,X,X^2\}$ la base canonique de $\R_2[X]$. 
On admet que la famille $\BC'=\{ X, X+1, X(X+1)\}$ est aussi une base de  $\R_2[X]$.

Tout \'el\'ement $\Pi(X)$ de  $\R_2[X]$ peut s'\'ecrire sous les formes suivantes:
$$
\Pi(X)=A+BX+CX^2=aX+b(X+1)+cX(X+1).
$$
\begin{enumerate}
\item Exprimer $A,B,C$ en fonction de $a,b,c$. 

\item Exprimer $a,b,c$ en fonction de $A,B,C$.
\item \'Ecrire les expressions pr\'ec\'edentes sous forme matricielle. C'est \`a dire, donner les matrices $P$ et $Q$ telles que 
$$
\left(\begin{array}{c}A\\B\\C\end{array}\right)=P\left(\begin{array}{c}a\\b\\c\end{array}\right)
\quad 
\text{ et }
\left(\begin{array}{c}a\\b\\c\end{array}\right)=Q\left(\begin{array}{c}A\\B\\C\end{array}\right)
$$

\item Quelle est la matrice de passage de $\BC$ \`a $\BC'$?\\

\indent
On consid\`ere l'endomorphisme de d\'erivation $D$ de $\R_2[X]$ d\'efini par $D : \Pi(X) \mapsto \Pi'(X)$.
On note $M$ la matrice de $D$ dans $\BC$, c'est \`a dire $M=\Mat_{\BC,\BC}(D)$ et
 $M'$ la matrice de $D$ dans $\BC'$, c'est \`a dire $M'=\Mat_{\BC',\BC'}(D)$. \\

\item \'Ecrire $M$.

\item Donner une relation entre $M$, $M'$, $P$ et $Q$.

%\item Utiliser la relation ci-dessus pour calculer $M'$.
\end{enumerate}


\bigskip
\noindent
\textbf{Exercice 2:}\\
On consid\`ere le sous-ensemble de matrices r\'eelles $2\times 2$ suivant:
$$
\F=\left\{\begin{pmatrix} a&b\\ 
-b&a \end{pmatrix},\; a,b\in \R\right\}.
$$
\begin{enumerate}
\item Montrer que $\F$ est un sous-espace vectoriel de $M_2(\R)$. Quelle est la dimension de $\F$?\\

On  consid\`ere $\C$ comme un espace vectoriel r\'eel, et on d\'efinit une application $f:\F\rightarrow \C$ par
$$
f:\begin{pmatrix} a&b\\ 
-b&a \end{pmatrix}
\mapsto
a+ib
$$
\item Montrer que $f$ est lin\'eaire.
\item Montrer que $f$ est bijective.
\item Montrer que pour toutes $M,N\in \F$ le produit $MN$ est dans $\F$.
\item Montrer que $f(MN) =f(M)f(N)$ pour toutes $M,N\in \F$.
\end{enumerate}

\newpage 
\medskip
\noindent
\textbf{Exercice 3:}\\
Calculer le d\'eterminant de la matrice suivante. 
$$
M= \left(
\begin{array}{ccc}
0&1&2\\[5pt]
1&1&2\\[5pt]
0&2&3
\end{array}
\right)
$$
En d\'eduire que $M$ est inversible et calculer son inverse.\\




\medskip
\noindent
\textbf{Exercice 4, \`a faire en dernier:}\\
Soient $E$ un espace vectoriel r\'eel de dimension $4$ de base $\BC$, 
et $F$ un espace vectoriel r\'eel de dimension~$3$ de base $\CC$.

On consid\`ere l'application lin\'eaire $\varphi:E\rightarrow F$ dont la matrice 
$A=\Mat_{\BC,\CC}(\varphi)$ est donn\'ee par:
$$
A=\begin{pmatrix} 1 & -2 &-1 &2 \\ 
1&-5&-1&5\\
-2&1&2&-1 \end{pmatrix}
$$

\begin{enumerate}
\item Calculer le rang de $A$.
\item On suppose ici que $E=\R^4$, $F=\R^3$ et que les bases $\BC$ et $\CC$ sont les bases canoniques.
\begin{enumerate}
\item Que vaut $\varphi\big((x,y,z,t)\big)$?
\item Montrer que $\ker(\varphi) =\Vect\{ (1,0,1,0),(0,1,0,1)\}$.
\end{enumerate}
\item On suppose ici que $E=\R_3[X]$, $F=\R_2[X]$ et que les bases $\BC$ et $\CC$ sont les bases canoniques,
c'est \`a dire $\BC=\{1,X,X^2,X^3\}$ et $\CC=\{1,X,X^2\}$.
\begin{enumerate}
\item Que vaut $\varphi(X^2-X)$?
\item Donner une base de $\ker(\varphi)$.
\end{enumerate}
\item La matrice $A$ peut-elle repr\'esenter une application lin\'eaire entre deux espaces de matrices?
\end{enumerate}



\end{document}


