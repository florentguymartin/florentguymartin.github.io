\documentclass[a4paper]{article}
\usepackage[francais]{babel}
\usepackage[latin1]{inputenc}
\usepackage[dvips,final]{graphics}
\usepackage{amsmath,amsfonts,amssymb}
\usepackage{theorem}
\usepackage[T1]{fontenc}
%\usepackage[cp850]{inputenc}
%\usepackage[textures]{epsfig}
%\usepackage{alltt}
%\renewcommand{\baselinestretch}{1,25}
%\psfigdriver{dvips}

\pagestyle{empty}
% my margins

\addtolength{\oddsidemargin}{-.875in}
\addtolength{\evensidemargin}{-.875in}
\addtolength{\textwidth}{1.75in}

\addtolength{\topmargin}{-.875in}
\addtolength{\textheight}{1.75in}
\theoremstyle{plain}
\theorembodyfont{\upshape}
\newtheorem{thm}{Th\'eoreme}[section]
\newtheorem{cor}[thm]{Corollaire}
\newtheorem{lem}[thm]{Lemme}
\newtheorem{prop}[thm]{Proposition}
\newtheorem{defn}[thm]{D\'efinition}
\newtheorem{rem}[thm]{Remarque}
\newtheorem{ex}[thm]{Ex}{\theorembodyfont{\upshape}}

\title{Fiche 1\\Espaces Vectoriels, Applications lin�aires}
\newcommand{\R}{\mathbb{R}}
\newcommand{\Z}{\mathbb{Z}}
\newcommand{\C}{\mathbb{C}}
\newcommand{\N}{\mathbb{N}}
\newcommand{\T}{\mathbb{T}}
\newcommand{\Q}{\mathbb{Q}}
\newcommand{\D}{\mathbb{D}}
\newcommand{\K}{\mathbb{K}}
\newcommand{\Li}{\mathcal{L}}
\newcommand{\M}{\mathcal{M}}
\renewcommand{\S}{\mathbb{S}}

\newcommand{\im}{\textup{Im}}
\newcommand{\BC}{\mathcal{B}}
\newcommand{\CC}{\mathcal{C}}
\newcommand{\DC}{\mathcal{D}}
\newcommand{\can}{_{\mathrm{can}}}

\newcommand{\rg}{\textup{rg}}
\newcommand{\cm}{\textup{Comat}}
\newcommand{\Mat}{\textup{Mat}}
\newcommand{\Vect}{\mathrm{Vect}}
\begin{document}
\noindent
\large
\textbf{Universit\'e Pierre et Marie Curie 
 - LM223 -
Ann\'ee 2012-2013}\\

\begin{center}
\Large
\textbf{Quelques corrections d'exercices de la fiche n$^o$ 1}
\end{center}
\normalsize
%%%%%%%%%%%%%%%%%%%%%%%%%%%%%%%
 \section{Espaces Vectoriels}
 %%%%%%%%%%%%%%%%%%%%%%%%%%%%%%%
 
\textbf{Corrig\'e Exo 1.3:}
\begin{enumerate}
\item 
Un sous-espace vectoriel de $\R^3$ contient n\'ecessairement le vecteur nul $(0,0,0)$.
Ici, on v\'erifie facilement que  $(0,0,0)$ appartient \`a $E_k$ si et seulement si $k=0$.
Il reste alors \`a v\'erifier que $E_0$ est bel et bien un sous-espace vectoriel de $\R^3$.

\textbullet Soient $(x,y,z), (x',y',z')\in E_0$, c'est \`a dire deus points de$\R^3$ satisfaisant  $x+y-2z=0$ et $x'+y'-2z'=0$.
Montrons que la somme $(x,y,z)+(x',y',z')=(x+x',y+y',z+z')$ est encore dans $E_0$.
\begin{eqnarray*}
(x+x')+(y+y')-2(z+z')&=&(x+y-2z)+(x'+y'-2z')\\
&=&0
\end{eqnarray*}
\textbullet Soient $(x,y,z)\in E_0$ et $\lambda \in \R$. 
Montrons que $\lambda (x,y,z)=(\lambda x, \lambda y, \lambda z)$ est encore dans $E_0$
\begin{eqnarray*}
(\lambda x)+(\lambda y)-2(\lambda z)&=& \lambda(x+y-2z)\\
&=&0
\end{eqnarray*}
Conclusion $E_0$ n'est pas vide, est stable par addition et multiplication scalaire, donc c'est un s.e.v de~$\R^3$.

\item $E_0\not= \R^3$ donc c'est un sous espace de dimension 1 ou 2 (nous pouvons anticiper que $\dim (E_0)=2$
car nous reconnaissons l'equation d'un plan vectoriel!). Il s'agit alors de trouver de 2 vecteurs libres satisfaisant l'equation de $E_0$...
Par exemple, $\{(2,0,1), (0,2,1) \}$ est une base de $E_0$.
\item Similaire \`a Question 1.
\item $E_0\cap F$ est l'intersection de deux plans vectoriels, c'est donc une droite vectorielle, autrement dit un s.e.v de dimension 1.
Il s'agit alors de trouver un vecteur directeur, $i.e.$ un vecteur staisfaisant les deux equations. Par exemple, on trouve $(1,-1, 0)$ d'o\`u
$E_0\cap F= \Vect \{(1,-1,0)\}$.
\end{enumerate}

\medskip
\textbf{Corrig\'e Exo 1.4:}

\noindent
Etant donn\'e $P(X)=AX^2+BX+C$ un polyn\^ome r\'eel de degr\'e au plus 2, montrons qu'on peut trouver $a,b,c$ r\'eels tels que 
$$P(X)=aX(X-1)+b(X-1)(X-2)+cX(X-2).$$
On peut, soit d\'evelopper l'expression ci-dessus, identifier avec la 1\`ere expression pour avoir $A,B,C$ en fonction de $a,b,c$ puis r\'esoudre pour avoir $a,b,c$ en fonction de $A,B,C$, ou alors, \'evaluer le polyn\^ome en des valeurs bien choisies pour avoir directement 
$a,b,c$ en fonction de $A,B,C$.
En \'evaluant en 0,1,2 on obtient:
$$
\left\{
\begin{array}{rcl}
2b&=&C\\
-c&=&A+B+C\\
2a&=&4A+2B+C
\end{array}
\right.
\quad
\text{ d'o\`u }\quad
\left\{
\begin{array}{rcl}
a&=&2A+B+\frac{1}{2}C\\
b&=&\frac{1}{2}C\\
c&=&-(A+B+C)
\end{array}
\right.
$$
On a donc montr\'e que la famille $\{ X(X-1), (X-1)(X-2), X(X-2)\}$ est g\'en\'eratrice dans $\R_2[X]$. 
Comme $\R_2[X]$ est de dimension 3, on conclut que cette famille est une base.

\medskip
\textbf{Corrig\'e Exo 1.5:}

\begin{enumerate}
\item 
Soient $\lambda, \mu$ r\'eels tels que $\lambda x+ \mu y =0$. 
Montrons que $\lambda= \mu=0$.
\begin{eqnarray*}
\lambda x+ \mu y &=&\lambda(2,3,-1)+ \mu (1,-1,-2)\\
&=& (2\lambda +\mu, 3\lambda -\mu, -\lambda -2\mu)
\end{eqnarray*}
\begin{eqnarray*}
\lambda x+ \mu y = 0&\Longrightarrow&(2\lambda +\mu, 3\lambda -\mu, -\lambda -2\mu)=(0,0,0)\\[6pt]
&\Longrightarrow&
\left\{
\begin{array}{lcl}
2\lambda +\mu&=&0\\
3\lambda -\mu&=&0\\
 -\lambda -2\mu&=&0
\end{array}
\right.
\end{eqnarray*}
On additionnant les deux premi\`eres equations, on trouve $5\lambda =0$ donc $\lambda =0$. Puis en substituant on trouve aussi $\mu =0$. Donc la famille $\{ x,y\}$ est libre. De m\^eme pour $\{ u,v \}$
\item On veut montrer que $\Vect \{ x,y\}=\Vect\{ u,v \}$. Commen\c{c}ons par montrer $\Vect \{ x,y\}\subset \Vect\{ u,v \}$. 
Soit $ax+by\in \Vect \{ x,y\}$ montrons qu'il existe $c,d \in \R$ tels que 
$$ax+by=cu+dv.$$
Ceci m\`ene au syst\`eme
$$
\left\{
\begin{array}{lcl}
2a +b&=&3c+5d\\
3a -b&=&7c\\
 -a -2b&=&-7d
\end{array}
\right.
$$
d'inconnues $c,d$. On trouve que le syst\`eme admet $c=\frac{1}{7}(3a-b), d=\frac{1}{7}(a+2b)$ comme solution.
On peut ensuite, de la m\^eme mani\`ere, montrer l'inclusion inverse  $\Vect \{ u,v\}\subset \Vect\{ x,y \}$
(qui conduit au m\^eme syst\`eme mais d'inconnues $a,b$). Ou alors, on peut utiliser un argument de dimension.
Puisque les familles sont libres, ce sont des bases respectives de $\Vect \{ x,y\}$ et $\Vect\{ u,v \}$: 
ces deux espaces sont de dimension 2.
L'inclusion $\Vect \{ x,y\}\subset \Vect\{ u,v \}$ plus m\^eme dimension, implique l'\'egalit\'e entre ces deux espaces.

\end{enumerate}
%%%%%%%%%%%%%%%%%%%%%%%%%%%%%%%
 \section{Applications lin\'eaires}
 %%%%%%%%%%%%%%%%%%%%%%%%%%%%%%%%

\medskip
\textbf{Corrig\'e Exo 2.3:}

\begin{enumerate}
\item Par propri\'et\'es de la d\'erivation, on a $D(\lambda P+\mu Q)=(\lambda P+\mu Q)'=\lambda P'+\mu Q'=\lambda D(P)+\mu D(Q)$ donc $D$ est lin\'eaire.\\
\textbullet Noyau: $D(P)=0\Longrightarrow P=c=\text{constante}$ donc $\ker (D)\simeq \R$.\\
\textbullet  Image: Il est clair que $D(\R_3[X])\subset \R_2[X]$. Montrons que $\im D =\R_2[X]$. 
Soit $Q=aX^2+bX+c \in \R_2[X]$. En posant $P=\frac{a}{3}X^3+\frac{b}{2}X^2+cX$, on obtient un ant\'ec\'edent, $D(P)=Q$.
Donc on a aussi l'inclusion $\R_2[X] \subset \im D$.
\item Le th\'eor\`eme du rang dit $\dim (\ker D)+\dim (\im D)=\dim(\R_3[X])$. Ici, on trouve  $\dim (\ker D)=1$, $\dim (\im D)=3$
et $\dim(\R_3[X])=4$: le th\'eor\`eme du rang est bien satisfait.
\item[3,4.] Les matrices de $D$ et $D^2$ dans la base canonique $(1, X, X^2, X^3)$ sont
$$
M=\Mat_{\BC_{\can}}(D)=
\left(
\begin{array}{cccc}
0&1&0&0\\
0&0&2&0\\
0&0&0&3\\
0&0&0&0
\end{array}
\right)
,\quad
M^2=\Mat_{\BC_{\can}}(D^2)=
\left(
\begin{array}{cccc}
0&0&2&0\\
0&0&0&6\\
0&0&0&0\\
0&0&0&0
\end{array}
\right)
$$
\item[5.] $M^4$ est la matrice associ\'ee \`a $D^4$. On sait qu'en d\'erivant 4 fois un polyn\^ome de degr\'e au plus 3, on obtiendra toujours 0. Donc $M^4$ est la matrice nulle.
\end{enumerate}

\medskip
\textbf{Corrig\'e Exo 2.4:}

\begin{enumerate}
\item  Par propri\'et\'es de la multiplication de matrice on a $f(\lambda M + \mu N)=A(\lambda M + \mu N)=\lambda AM + \mu AN=
\lambda f(M) + \mu f(N)$.

On calcule successivement $f(E_{11}), f(E_{12})\ldots$, on trouve
$$
\begin{array}{ccccccc}
f(E_{11})&=&\begin{pmatrix} 2&5\\1&3 \end{pmatrix}\begin{pmatrix} 1 & 0 \\ 0 & 0 \end{pmatrix}&=&
\begin{pmatrix} 2 & 0 \\ 1 & 0 \end{pmatrix}&=&2E_{11}+E_{21}\\[12pt]
f(E_{12})&=&\begin{pmatrix} 2&5\\1&3 \end{pmatrix}\begin{pmatrix} 0 & 1 \\ 0 & 0 \end{pmatrix}&=&
\begin{pmatrix} 0 & 2 \\ 0 & 1 \end{pmatrix}&=&2E_{12}+E_{22}\\[12pt]
f(E_{21})&=&\begin{pmatrix} 2&5\\1&3 \end{pmatrix}\begin{pmatrix} 0 & 0 \\ 1 & 0 \end{pmatrix}&=&
\begin{pmatrix} 5 & 0 \\ 3 & 0 \end{pmatrix}&=&5E_{11}+3E_{21}\\[12pt]
f(E_{22})&=&\begin{pmatrix} 2&5\\1&3 \end{pmatrix}\begin{pmatrix} 0 & 0 \\ 0 & 1 \end{pmatrix}&=&
\begin{pmatrix} 0 & 5 \\ 0 & 3 \end{pmatrix}&=&5E_{12}+3E_{22}\\
\end{array}
$$
d'o\`u
$$
\Mat_{\BC_{\can}}(f)=
\left(
\begin{array}{cccc}
2&0&5&0\\
0&2&0&5\\
1&0&3&0\\
0&1&0&3
\end{array}
\right)
$$


\item On transforme la matrice ci-dessus par pivot de Gauss sur les lignes:
$$
\left(
\begin{array}{cccc}
2&0&5&0\\
0&2&0&5\\
1&0&3&0\\
0&1&0&3
\end{array}
\right)
\sim
\left(
\begin{array}{cccc}
2&0&5&0\\
0&2&0&5\\
0&0&\frac{1}{2}&0\\
0&1&0&3
\end{array}
\right)
\sim
\left(
\begin{array}{cccc}
2&0&5&0\\
0&2&0&5\\
0&0&\frac{1}{2}&0\\
0&0&0&\frac{1}{2}
\end{array}
\right)
$$
On d\'eduit dans un premier temps que la matrice est inversible, car de d\'eterminant non nul ($\det f=1$).
En poursuivant les transformations, et en les faisant simultan\'ement sur la matrice identit\'e on trouve
l'inverse:
$$
\Mat_{\BC_{\can}}(f)^{-1}=
\left(
\begin{array}{cccc}
3&0&-5&0\\
0&3&0&-5\\
-1&0&2&0\\
0&-1&0&2
\end{array}
\right)
$$
\textit{Remarque: pour cette derni\`ere question, on peut \'egalement montrer que $f$ est inversible, en montrant l'existence de 
l'application r\'eciproque. En effet, on voit facilement que $A$ est inversible et que $ M \mapsto A^{-1}M$  donne $f^{-1}$. 
On peut ensuite calculer $\Mat_{\BC_{\can}}(f^{-1}) $ en faisant la m\^eme chose qu'\`a la question~1.}
\end{enumerate}

\section{D\'efinitions abstraites}

\section{Changement de base}


\medskip
\textbf{Corrig\'e Exo 3.1:}

Notons $\BC$ la base canonique et $\BC'$ une nouvelle base telle que 
$$
\Mat_{\BC', \BC'}(f)=\begin{pmatrix} 2&0\\0&3 \end{pmatrix}=:B
$$
Notons \'egalement $P=\Mat_{\BC', \BC}(id)$ la matrice de passage de $\BC$ \`a $\BC'$. 
Les matrices sont reli\'ees par
$$
B=P^{-1}AP.
$$
On cherche donc une matrice
$$
P=\left(\begin{array}{cc}a&b\\c&d\end{array}\right),
\quad \text{ telle que } \quad PB=AP.
$$
Ceci se traduit par un syst\`eme \`a r\'esoudre, on trouve $P=\begin{pmatrix} 2&1\\1&1 \end{pmatrix}$. 
Les colonnes de cette matrice donnent les coordonn\'ees de la base  $\BC'$ dans  $\BC$.
On trouve  $\BC'=\{(2,1), (1,1)\}$.

\textit{Remarque: on peut remplacer chacun des vecteurs de $\BC'$ par un multiple, \c{c}a ne changera pas la propri\'et\'e de la base.}
 
 
 \section{Matrices}
 
 
\medskip
\textbf{Corrig\'e Exo 5.2:}
$$ \left(
\begin{array}{cccc}
a&a &a &a\\
 a&b&b&b \\
  a&b  &c&c\\
a &b &c&d \\
\end{array}
\right)
\sim
 \left(
\begin{array}{cccc}
a&a &a &a\\
0 &b-a&b-a&b-a \\
 0&b-a  &c-a&c-a\\
0&b-a &c-a&d-a \\
\end{array}
\right) 
\sim
 \left(
\begin{array}{cccc}
a&a &a &a\\
0 &b-a&b-a&b-a \\
 0&0 &c-b&c-b\\
0&0 &c-b&d-b \\
\end{array}
\right) 
\sim
 \left(
\begin{array}{cccc}
a&a &a &a\\
0 &b-a&b-a&b-a \\
 0&0 &c-b&c-b\\
0&0 &0&d-c\\
\end{array}
\right) 
$$

d'o\`u $\det =a(b-a)(c-b)(d-c)$.

Pour la deuxi\`eme matrice, on suppose dans un 1er temps que $a\not=0$. 
La matrice est de taille paire, disons $2n$. 
On fait successivement les op\'erations suivantes sur les lignes:
$$L_{2n}\leftarrow L_{2n}-\frac{b}{a}L_{1}; \; L_{2n-1}\leftarrow L_{2n-1}-\frac{b}{a}L_{2};\;L_{2n-2}\leftarrow L_{2n-2}-\frac{b}{a}L_{3}\ldots
$$
On obtient

$$
\hspace{0.5cm}
 \left(
\begin{array}{cccccc}
a& & && & b\\
 &\diagdown &&&\diagup & \\
  &  & a&b& &\\
& &0&a-\frac{b^2}{a}& & \\
 &\diagup &&&\diagdown & \\
0  &  & & & &a-\frac{b^2}{a}
\end{array}
\right),
 $$
d'o\`u $\det =a^n(a-\frac{b^2}{a})^n=(a^2-b^2)^n$.

Si $a=0$ la matrice initiale est une matrice antidiagonale. Pour la mettre sous forme diagonale on peut \'echanger les lignes $L_1$ et $L_{2n}$ puis  $L_2$ et $L_{2n-1}$, etc jusqu'\`a $L_n$ et $L_{n+1}$. 
Ce qui fait au total $n$ \'echanges, donc $n$ changements de signe pour le d\'eterminant. D'o\`u $\det =(-1)^nb^{2n}$. 

\end{document}