\documentclass[a4paper, 11pt]{article}
\usepackage[francais]{babel}
\usepackage[latin1]{inputenc}
\usepackage[dvips,final]{graphics}
\usepackage{amsmath,amsfonts,amssymb}
\usepackage{theorem}
\usepackage[T1]{fontenc}
%\usepackage[cp850]{inputenc}
%\usepackage[textures]{epsfig}
%\usepackage{alltt}
%\renewcommand{\baselinestretch}{1,25}
%\psfigdriver{dvips}

\pagestyle{empty}
% my margins

\addtolength{\oddsidemargin}{-.875in}
\addtolength{\evensidemargin}{-.875in}
\addtolength{\textwidth}{1.75in}

\addtolength{\topmargin}{-.875in}
\addtolength{\textheight}{1.75in}
\theoremstyle{plain}
\theorembodyfont{\upshape}
\newtheorem{thm}{Th\'eoreme}
\newtheorem{cor}[thm]{Corollaire}
\newtheorem{lem}[thm]{Lemme}
\newtheorem{prop}[thm]{Proposition}
\newtheorem{defn}[thm]{D\'efinition}
\newtheorem{rem}[thm]{Remarque}
\newtheorem{ex}{Exercice}{\theorembodyfont{\upshape}}

\title{Fiche 1\\Espaces Vectoriels, Applications lin�aires}
\newcommand{\R}{\mathbb{R}}
\newcommand{\Z}{\mathbb{Z}}
\newcommand{\C}{\mathbb{C}}
\newcommand{\N}{\mathbb{N}}
\newcommand{\T}{\mathbb{T}}
\newcommand{\Q}{\mathbb{Q}}
\newcommand{\D}{\mathbb{D}}
\newcommand{\K}{\mathbb{K}}
\newcommand{\Li}{\mathcal{L}}
\newcommand{\M}{\mathcal{M}}
\newcommand{\F}{\mathbf{F}}
\renewcommand{\S}{\mathbb{S}}

\newcommand{\im}{\textup{Im}}
\newcommand{\BC}{\mathcal{B}}
\newcommand{\CC}{\mathcal{C}}
\newcommand{\DC}{\mathcal{D}}
\newcommand{\can}{_{\mathrm{can}}}

\newcommand{\Mat}{\textup{Mat}}
\newcommand{\Vect}{\mathrm{Vect}}
\newcommand{\rg}{\textup{rg}}
\newcommand{\cm}{\textup{Comat}}

\begin{document}
\noindent
\large
\textbf{Universit\'e Pierre et Marie Curie 
 - LM223 -
Ann\'ee 2012-2013}\\

\begin{center}
\Large
\textbf{Interro n$^o$ 2}
\end{center}
\normalsize


\bigskip
\noindent
\textbf{Exercice 1:}
\begin{enumerate}
\item Donner les d\'efinitions de ``endomorphisme diagonalisable'' et ``matrice diagonalisable''.
\item \'Enoncer le premier th\'eor\`eme de diagonalisation.
\item Donner un exemple de matrice $A$ telle que son polyn\^ome caract\'eristique $P_A(X)=(1-X)^3$ et $A$ \textbf{diagonalisable}.
\item Donner un exemple de matrice $B$ telle que son polyn\^ome caract\'eristique $P_B(X)=(1-X)^3$ et $B$ \textbf{non-diagonalisable}.
\end{enumerate}

\bigskip
\noindent
\textbf{Exercice 2:}\\
%On consid\`ere l'endomorphisme dont la matrice dans la base canonique est
Soit $\displaystyle M = \begin{pmatrix}-1&1&1\\ 1&-1&1\\ 1&1&-1\end{pmatrix} . $
\begin{enumerate}
\item Calculer le polyn\^ome caract\'eristique $P_M(X)$.
\item Quelles sont les valeurs propres de $M$? Donner des bases des sous-espaces propres associ\'es.
\item $M$ est-elle diagonalisable? Si oui, donner une matrice $P$ telle que $P^{-1}MP$ soit diagonale.
\end{enumerate}


\bigskip
\noindent
\textbf{Exercice 3:}\\
On consid\`ere l'espace vectoriel $E=\R_2[X]$ des  polyn\^omes r\'eels de degr\'e inf\'erieur ou
\'egal \`a 2. 
On note $p_0=1, p_1=1+X, p_2=(1+X)^2$.
\begin{enumerate}
\item Montrer que la famille $\BC=\{p_0,p_1,p_2\}$ est une base de $E$.
\item On note $\BC^*=\{p_0^*,p_1^*,p_2^*\}$  la base duale de $\BC$.
Que vaut $p_2^*(aX^2+bX+c)$?

\item Soit $p\in E$. Montrer que l'application $\varphi_p: E\rightarrow \R, q\mapsto \int_0^1p(x)q(x)dx$, est un \'el\'ement de $E^*$.

\item On choisit $p=X$. Calculer les valeurs de $\varphi_X$ sur $\BC$.

\item Montrer que l'application $\Phi:E\rightarrow E^*, p\mapsto \varphi_p$, est un isomorphisme.
\end{enumerate}



\bigskip
\noindent
\textbf{Exercice 4:}\\
On consid\`ere 
la forme bilin\'eaire $b:\R^3\times \R^3\rightarrow \R$
dont la matrice dans la base canonique est
$$
A=
\begin{pmatrix}
3&1&-5\\
1&0&-2\\
-5&-2&6
\end{pmatrix}
$$
\begin{enumerate}
  \item La forme $b$ est-elle sym\'etrique? Que vaut $b\big((1,1,0),(0,-1,-1)\big)$, $b\big((x_1,x_2,x_3),(x_1,x_2,x_3)\big)$?
  \end{enumerate}  
On donne une nouvelle base  $\BC'=\{e'_1,e'_2,e'_3\}$ de $\R^3$ (on ne demande pas de montrer que c'est une base) o\`u
\begin{eqnarray*}
e'_1&=&(1,0,1), \\
e'_2&=&(-1,2,0),\\
e'_3&=&(0,1,1).
\end{eqnarray*}
  \begin{enumerate}
  \item[2.] \'Ecrire la matrice de $b$ dans $\BC'$.
  \item[3.] Soient $u,v\in \R^3$ de coordonn\'ees  $(1,1,0)$, respectivement $(0,-1,-1)$, dans $\BC'$.
  Que vaut $b(u,v)$?
 \item[4.]  Soit $w=e'_1+e'_2+e'_3$. Que vaut $b(w,w)$? Donner une base de $\{w\}^\bot$.
  \item[5.] La forme $b$ est-elle d\'eg\'en\'er\'ee? Est-elle un produit scalaire?\end{enumerate}

\end{document}