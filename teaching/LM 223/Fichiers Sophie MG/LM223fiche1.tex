\documentclass[a4paper]{article}
\usepackage[francais]{babel}
\usepackage[latin1]{inputenc}
\usepackage[dvips,final]{graphics}
\usepackage{amsmath,amsfonts,amssymb}
\usepackage{theorem}
\usepackage[T1]{fontenc}
%\usepackage[cp850]{inputenc}
%\usepackage[textures]{epsfig}
%\usepackage{alltt}
%\renewcommand{\baselinestretch}{1,25}
%\psfigdriver{dvips}

\pagestyle{empty}
% my margins

\addtolength{\oddsidemargin}{-.875in}
\addtolength{\evensidemargin}{-.875in}
\addtolength{\textwidth}{1.75in}

\addtolength{\topmargin}{-.875in}
\addtolength{\textheight}{1.75in}
\theoremstyle{plain}
\theorembodyfont{\upshape}
\newtheorem{thm}{Th\'eoreme}[section]
\newtheorem{cor}[thm]{Corollaire}
\newtheorem{lem}[thm]{Lemme}
\newtheorem{prop}[thm]{Proposition}
\newtheorem{defn}[thm]{D\'efinition}
\newtheorem{rem}[thm]{Remarque}
\newtheorem{ex}[thm]{Ex}{\theorembodyfont{\upshape}}

\title{Fiche 1\\Espaces Vectoriels, Applications lin�aires}
\newcommand{\R}{\mathbb{R}}
\newcommand{\Z}{\mathbb{Z}}
\newcommand{\C}{\mathbb{C}}
\newcommand{\N}{\mathbb{N}}
\newcommand{\T}{\mathbb{T}}
\newcommand{\Q}{\mathbb{Q}}
\newcommand{\D}{\mathbb{D}}
\newcommand{\K}{\mathbb{K}}
\newcommand{\Li}{\mathcal{L}}
\newcommand{\M}{\mathcal{M}}
\renewcommand{\S}{\mathbb{S}}

\newcommand{\im}{\textup{Im}}
\newcommand{\BC}{\mathcal{B}}
\newcommand{\CC}{\mathcal{C}}
\newcommand{\DC}{\mathcal{D}}
\newcommand{\can}{_{\mathrm{can}}}

\newcommand{\rg}{\textup{rg}}
\newcommand{\cm}{\textup{Comat}}

\begin{document}
\noindent
\large
\textbf{Universit\'e Pierre et Marie Curie 
 - LM223 -
Ann\'ee 2012-2013}\\

\begin{center}
\Large
\textbf{Fiche d'exercices n$^o$ 1}
\end{center}
\normalsize
%%%%%%%%%%%%%%%%%%%%%%%%%%%%%%%
 \section{Espaces Vectoriels}
 %%%%%%%%%%%%%%%%%%%%%%%%%%%%%%%
 \begin{ex}
\begin{enumerate}
\item Rappeler pourquoi l'ensemble $\R^\R$, \textit{ie} l'ensemble des
applications de $\R$ dans $\R$ est un espace vectoriel r\'eel.

\item Parmi les parties suivantes de $\R^\R$, lesquelles sont des sous-espaces
vectoriels de $\R^\R$:
\begin{enumerate}
\item
L'ensemble des applications $f:\R\to \R$ telles que :
$f(0)=0,$
\item
L'ensemble des applications $f:\R\to \R$ telles que :
$f(1)=0,$
\item
L'ensemble des applications $f:\R\to \R$ telles que :
$f(0)=1,$
\item
L'ensemble des applications $f:\R\to \R$ d\'erivables en 1 et telles que :
$f'(1)=0,$
\item
L'ensemble des applications $f:\R\to \R$ solutions de l'equation
diff\'erentielle :
$f'(x)=\cos(x)f(x),$
\item
L'ensemble des applications $f:\R\to \R$ solutions de l'equation
diff\'erentielle :
$f'(x)+\cos(x)f(x)=\sin(x),$
\item  L'ensemble des polynomes r\'eels,
\item L'ensemble des polynomes r\'eels de degr\'e inf\'erieur ou \'egal \`a 2.
\end{enumerate}
\end{enumerate}
\end{ex}


\begin{ex}
\begin{enumerate}
\item On se place dans l'espace vectoriel $\R^3$. Parmi les familles de
vecteurs suivantes, dire lesquelles sont g\'en\'eratrices, lesquelles sont
libres, lesquelles sont des bases. Compl\'eter les familles libres en
des bases. Extraire des bases des familles g\'en\'eratrices.
\begin{enumerate}
\item $F_1=\{(1,0,1),(0,1,0),(1,0,0)\}$
\item $F_2=\{(0,1,2),(2,1,0)\}$
\item $F_3=\{(1,0,2),(0,1,2),(1,2,0)\}$
\item $F_4=\{(2,2,1),(1,1,1),(1,0,0),(0,1,0)\}$
\end{enumerate}
\item Dans $\R^\R$, la famille $\{\cos, \sin\}$ est-elle libre? g�n�ratrice?
\item Dans $\R$ consid�r� comme un espace vectoriel sur $\Q$, les familles $\{1, \sqrt{2}\}$ et $\{1, \sqrt{2}, \sqrt{3}\}$ sont-elles libres? g�n�ratrices?
\end{enumerate}
\end{ex}


\begin{ex}
Soit $E_k:=\lbrace (x,y,z)\in \R^3 / x+y-2z=k \rbrace$ pour $k\in \R$.
\begin{enumerate}
\item D\'eterminer les r\'eels $k$ pour que $E_k$ soit un sous espace vectoriel de $\R^3$
\item Donner une base de $E_0$.
\item Soit $F:=\lbrace (x,y,z)\in \R^3 / x+y-3z=0 \rbrace$. Montrer que $F$ est un sous espace vectoriel de   $\R^3$.
Donner une base de $F$.
\item D\'eterminer $E_0\cap F$.
\end{enumerate}
\end{ex}

\begin{ex}
Montrer que si $P$ est un polynome r\'eel de degr\'e inf\'erieur ou
\'egal \`a 2 alors $P$ peut s'\'ecrire sous la forme:
$$P(X)=aX(X-1)+b(X-1)(X-2)+cX(X-2)$$ o\`u $a,b,c$ sont des constantes r\'eelles
que l'on d\'eterminera. 

Que peut-on dire de la famille $\{ X(X-1), (X-1)(X-2), X(X-2)\}$?
\end{ex}

\begin{ex}
On consid\`ere les vecteurs de $\R^3$ suivants: $x=(2,3,-1)$, $y=(1,-1,-2)$, $u=(3,7,0)$ et $v=(5,0,-7)$. 
\begin{enumerate}
\item Montrer que $\{ x,y\}$ et $\{ u,v \}$ sont deux familles libres.
\item Montrer que $\{ x,y\}$ et $\{ u,v \}$ engendrent le m\^eme sous espace vectoriel de $\R^3$.

\end{enumerate}
\end{ex}

%%%%%%%%%%%%%%%%%%%%%%%%%%%%%%%
 \section{Applications lin\'eaires}
 %%%%%%%%%%%%%%%%%%%%%%%%%%%%%%%%
 \begin{ex}
\begin{enumerate}
\item Parmi les applications suivantes de $\R^3$ dans $\R^3$,
lesquelles sont lin\'eaires. D\'eterminer le noyau de celles qui sont
lin\'eaires.
\begin{enumerate}
\item $\psi_1: (x,y,z) \mapsto (x^2,2y,x+y)$

\item $\psi_2: (x,y,z) \mapsto (3z+y,x+y+z,x+y)$

\item $\psi_3: (x,y,z) \mapsto (y+z,x.z,x+y)$

\item $\psi_4: (x,y,z) \mapsto (x+3y,2y,4x)$  
\end{enumerate}
\item Parmi les applications suivantes de $\R^\R$ dans $\R$,
lesquelles sont lin\'eaires. D\'eterminer le noyau de celles qui sont
lin\'eaires.
\begin{enumerate}
\item $\phi_1 : f \mapsto (f(1))^2$
\item $\phi_2 : f \mapsto f(1)+f(2)$
\item $\phi_3 : f \mapsto 2+f(1)$
\end{enumerate}
\end{enumerate}
\end{ex}

\begin{ex}
Soit $a\in \C$, on d\'efinit $f:\C\longrightarrow \C$ par $z\mapsto z+a\bar z$.\\
Suivant les valeurs de $a$, dire si $f$ est $\C$-lin\'eaire ou $\R$-lin\'eaire.  Quand $f$ est
$\R$ lin\'eaire donner son noyau, son image et sa matrice dans la base $(1,i)$.

\end{ex}

\begin{ex}
On consid\`ere $E = \R_3[X]$, muni de l'op\'eration de d\'erivation $D : P \mapsto P'$.
\begin{enumerate}
\item V\'erifier que D est lin\'eaire, calculer son noyau et son image.
\item \'Enoncer le th\'eor\`eme du rang et le v\'erifier sur cet exemple.
\item \'Ecrire la matrice $M$ de $D$ dans la base canonique $(1, X, X^2, X^3)$.
\item On consid\`ere $D^2: E \to E,\ P \mapsto P''$; \'Ecrire sa matrice dans la m\^eme base.
\item Donner \emph{sans calculs} la valeur de $M^4$.
\end{enumerate}
\end{ex}


\begin{ex}
Soient $E_{11} = \begin{pmatrix} 1 & 0 \\ 0 & 0 \end{pmatrix}$, $E_{12} = \begin{pmatrix} 0 & 1 \\ 0 & 0 \end{pmatrix}$, $E_{21} = \begin{pmatrix} 0 & 0 \\ 1 & 0 \end{pmatrix}$, $E_{22} = \begin{pmatrix} 0 & 0 \\ 0 & 1 \end{pmatrix}$; on rappelle $(E_{11}, E_{12}, E_{21}, E_{22})$ est une base, dite canonique, de $M_2(\C)$. On pose $A = \begin{pmatrix} 2&5\\1&3 \end{pmatrix}$ et on consid\`ere l'application $f : M_2(\C) \to M_2(\C),\ M \mapsto AM$.
\begin{enumerate}
\item Montrer que $f$ est lin\'eaire et calculer sa matrice dans la base canonique.
\item Prouver que $f$ est inversible et calculer son inverse.
\end{enumerate}
\end{ex}

\section{Changement de base}
\begin{ex}
Soit $A= \begin{pmatrix} 1&2\\-1&4 \end{pmatrix}$, et $f$ l'endomorphisme de $\R^2$ qui a pour matrice $A$ dans la base canonique. Trouver une base dans laquelle $f$ admet pour matrice $\begin{pmatrix} 2&0\\0&3 \end{pmatrix}$. Calculer la matrice de passage.
\end{ex}


\begin{ex}
On note $E=\R^3$, $F=\R^2$, et on les munit de leurs bases
canoniques respectives $\BC\can$ et $\CC\can$. On note aussi $\BC$ la
famille~:
\[\BC=\left(\left(\begin{array}{c}1\\0\\0\end{array}\right),
\left(\begin{array}{c}-1\\1\\0\end{array}\right),
\left(\begin{array}{c}1\\-3\\2\end{array}\right)\right),\]
et $\phi:E\to F$ l'application d\'efinie par~:
\[\phi\left(\begin{array}{c}x_1\\x_2\\x_3\end{array}\right)=
\left(\begin{array}{rrr}x_1&-x_2&+3x_3\\&2x_2&+x_3\end{array}\right).\]

\begin{enumerate}
\item Montrer que $\BC$ est une base de $E$.

\item Ecrire les matrices $M_{\BC\can,\CC\can}(\phi)$ et  $M_{\BC,\CC\can}(\phi)$.

\item D\'eterminer des \'equations, puis une base du noyau de $\phi$.

\item D\'eterminer une famille g\'en\'eratrice, puis des \'equations de l'image
de $\phi$.
\end{enumerate}
\end{ex}


 %%%%%%%%%%%%%%%%%%%%%%%%%%%%%%%
 \section{Pour s'entrainer avec les d\'efinitions abstraites}
 %%%%%%%%%%%%%%%%%%%%%%%%%%%%%%%
\begin{ex}
On rappelle qu'un espace vectoriel sur un corps $\K$, est la donn\'ee d'un groupe ab\'elien $E$, et d'une
multiplication externe $(\lambda ,x)\rightarrow \lambda x$ de $\K\times E$ dans $E$, v\'erifiant pour tout 
$(x,y)\in E\times E$ et tout $(\lambda ,\mu )\in \K\times \K$:
\begin{enumerate}
\item[(i)] $\lambda (x+y)=\lambda x+\lambda y$
\hspace{2cm} \item[(ii)] $(\lambda +\mu )x=\lambda x+\mu x$
\item[(iii)] $\lambda (\mu x)=(\lambda \mu )x$
\item[(iv)] $1x=x$
\end{enumerate}

\begin{enumerate}
\item
Montrer que $0x=0_E$.
Puis montrer que $(-1)x $ est l'inverse de $x$ dans le groupe ab\'elien $E$
\item
En d\'eduire qu'il est inutile de supposer $E$ ab\'elien, que cette propri\'et\'e peut se d\'eduire des axiomes
(i) \`a (iv).
\end{enumerate}
\end{ex}

\begin{ex}

Soit $E$ un espace vectoriel r\'eel. On d\'efinit une multiplication externe  de $\C\times E^2$ dans $E^2$ par:\\
$$ (a+ib)\cdot (u,v)=(au-bv,av+bu)$$

Montrer que $E^2$ est un espace vectoriel sur $\C$ pour cette multiplication et pour son addition usuelle.

\end{ex}





\begin{ex}

Soit $E$ un $\K$-espace vectoriel. Soit $F$ et $G$ deux sous espaces vectoriels de $E$.
\begin{enumerate}
\item Montrer que $F+G$ est un sous espace vectoriel de $E$.
\item Montrer que $F\cap G$ est un sous espace vectoriel de $E$.
\item
\begin{enumerate}
\item $F\cup G$ est-il un sous espace vectoriel de $E$? Donner un contre exemple.
\item Montrer que si $F\cup G$ est un sous espace vectoriel de $E$, alors $F\subset G$ ou $G\subset F$.
\end{enumerate}
\end{enumerate}

\end{ex}







\begin{ex}
Soit $E$ un espace vectoriel, on note $v_i$ des \'el\'ements de $E$.
\begin{enumerate}

\item Montrer que si $\mathcal{G} =(v_1,\dots,v_n)$ est g\'en\'eratrice, alors pour
tout vecteur $v\in E$, la famille $(v_1,\dots,v_n,v)$ est g\'en\'eratrice.

\item Plus g\'en\'eralement, si $\mathcal{G} =(v_1,\dots,v_n)$ est g\'en\'eratrice, pour toute famille $\mathcal{F}$, la famille
$\mathcal{G}\cup \mathcal{F}$ est g\'en\'eratrice.

\item Montrer que si $\mathcal{L}=(v_1,\dots,v_n)$ est libre, alors
$(v_1,\dots,v_{n-1})$ est libre.

\item Montrer que si $\mathcal{L}=(v_1,\dots,v_n)$ est libre, et si $v\not\in <(v_1,\dots,v_n)>$ alors
$(v_1,\dots,v_n,v)$  est libre.


\item Si $\mathcal{F}=(v_1,\dots,v_n)$ est li\'ee, alors pour tout $v\in E$,
$(v_1,\dots,v_n,v)$ est li\'ee.

\end{enumerate}
\end{ex}





\begin{ex}
Soit $E:=\lbrace f\in \R^\R / \exists (a,\varphi )\in \R^2, \forall x\in \R f(x)=a\cos (x-\varphi ) \rbrace$. 
\begin{enumerate}
\item V\'erifier que $E$ est un espace vectoriel r\'eel.
\item V\'erifier que les applications $\cos$ et $\sin$ appartiennent \`a $E$.
\item Donner une base de $E$.
\end{enumerate}
\end{ex}




\begin{ex}
Soit $E$ un espace vectoriel de dimension finie. $F$ et $G$ des sous espaces vectoriels de $E$.
\begin{enumerate}
\item Montrer que les assertions suivantes sont equivalentes.
\begin{enumerate}
\item Pour tout $x\in E$ il existe un unique couple $(u,v)\in F\times G$ tel que 
$x=u+v$.
\item $F+G=E$ et $F\cap G=\lbrace 0_E \rbrace$
\item $F+G=E$ et $\dim F+\dim G=\dim E$
\item $\dim F+\dim G=\dim E$ et $F\cap G=\lbrace 0_E \rbrace$
\end{enumerate}
Dans ces cas on note $E=F\oplus G$, et on dit que $E$ est la somme directe de $F$ et $G$.

\item Montrer qu'il existe $H$ un sous espace vectoriel de $E$ tel que $E=F\oplus H$.
\end{enumerate}
\end{ex}

\begin{ex}
Soit  $E$ un espace vectoriel, $F$ et $G$ deux sous espaces vectoriels de $E$ .
\begin{enumerate}
\item Montrer que le sous espace engendr\'e par $F\cup G$ est le sous espace $F+G$.
\item Si $A$ et $B$ sont deux parties quelconques  de $E$, quel est le sous espace engendr\'e par $A\cup B$?
\item Quel est le sous espace engendr\'e par $\complement _E F$ le compl\'ementaire de $F$ dans $E$? \\
(distinguer les cas $E=F$ et $E\not=F$)
\end{enumerate}
\end{ex}

\begin{ex}
Soit $f\in \Li(E,F)$, et soit $(v_1,...,v_p)$ une famille de vecteurs de $E$ .
\begin{enumerate}
\item On suppose que $(v_1,...,v_p)$ est une famille  libre et que $f$ est injective. Montrer que 
$(f(v_1),...,f(v_p))$ est une famille  libre dans $F$.

\item On suppose que $(v_1,...,v_p)$ est une famille g\'en\'eratrice et que $f$ est surjective. Montrer que 
$(f(v_1),...,f(v_p))$ est une famille  g\'en\'eratrice dans $F$.

\end{enumerate}
\end{ex}

\begin{ex}
Soit $E$, $F$ et $G$ trois espaces vectoriels.
Soit $f\in \Li(E,F)$ surjective et $g$ une application quelconque de $F$ dans $G$ .
On suppose que $g\circ f$ est lin\'eaire, montrer que $g$ est lin\'eaire.



\end{ex}

\begin{ex}
Soit $E$ un espace vectoriel de dimension finie.
Soit $f$ et $g$ deux endomorphismes de $E$, tels que $f\circ g=g\circ f$
\begin{enumerate}
\item Montrer que $f(\ker g)\subset \ker g$ et $f($Im$ g)\subset $ Im$ g$.
\item On suppose que $f+g=id_E$\\
Montrer que $\ker (f\circ g)=\ker f\oplus \ker g$.
\end{enumerate}
\end{ex}


\begin{ex}
Dans l'espace vectoriel r\'eel des fonctions de $\R$ dans $\R$, soit $E$
le sous-espace vectoriel engendr\'e par les fonctions
\[ \begin{array}{cccl}f:&\R&\longrightarrow&\R\\
&x&\longmapsto&\cos(x)\end{array},\quad
\begin{array}{cccl}g:&\R&\longrightarrow&\R\\
&x&\longmapsto&\sin(x)\end{array},\]
\[\begin{array}{cccl}h:&\R&\longrightarrow&\R\\
&x&\longmapsto&\cos(x-1)\end{array},\quad
\begin{array}{cccl}k:&\R&\longrightarrow&\R\\
&x&\longmapsto&\sin(x-1)\end{array},\quad
\]

\begin{enumerate}
\item Est-ce que la famille $(f,g,h,k)$ est libre~?

\item Montrer que $\BC=(f,g)$ est une base de $E$.
Quelle est la dimension de $E$~?

\item Compl\'eter la famille \`a un vecteur $(h)$ en une base $\CC$ de $E$.

\item Montrer que la d\'erivation des fonctions $D:u\mapsto u'$ est bien
un endomorphisme de $E$ (i.e., que la d\'eriv\'ee d'un \'el\'ement de $E$
appartient \`a $E$ et que $D$ est lin\'eaire).

\item Ecrire la matrice $M_{\BC,\BC}(D)$.  Ecrire la matrice $M_{\CC,\CC}(D)$. 
Donner les matrices de passage de $\BC$ \`a $\CC$ et de  $\CC$ \`a $\BC$.


\end{enumerate}
\end{ex}

\newpage

 %%%%%%%%%%%%%%%%%%%%%%%%%%%%%%%
 \section{Matrices: d\'eterminant, inversion, rang}
 %%%%%%%%%%%%%%%%%%%%%%%%%%%%%%%
 
\begin{ex}
Calculer de deux manieres diff�rentes le d�terminant des matrices suivantes~:
$$\left(\begin{array}{ccc}1&0&2\\-3&1&4\\2&-1&3\end{array}\right),\quad
\left(\begin{array}{ccc}2&-1&3\\-1&3&1\\1&-1&1\end{array}\right),\quad
\left (\begin {array}{ccc} 1&i&2\\1&0&1
\\2&i&2\end {array}\right),\quad
\left (\begin {array}{ccc}1&2&0\\i&i&-1\\0&2&i\end{array}\right)
$$

%\sss Pour quelle(s) valeur(s) de $\a\in\RM$ la matrice
%$\left(\begin{array}{ccc}1&2&\a\\1&2&1\\2&4&2\end{array}\right)$
%est-elle inversible~?
\end{ex}
 \begin{ex}
Calculer les d\'eterminants des matrices suivantes:
$$ \left(
\begin{array}{cccc}
a&a &a &a\\
 a&b&b&b \\
  a&b  &c&c\\
a &b &c&d \\
\end{array}
\right),
\hspace{0.5cm}
 \left(
\begin{array}{cccccc}
a& & && & b\\
 &\diagdown &&&\diagup & \\
  &  & a&b& &\\
& &b &a& & \\
 &\diagup &&&\diagdown & \\
b  &  & & & &a
\end{array}
\right),
\hspace{0.5cm}
 \left(
\begin{array}{cccc}
1+a^2&a & & \\
 a&\ddots&\ddots&\\
&\ddots&\ddots& a\\
 & &a&1+a^2 \\
\end{array}
\right)
 $$
\end{ex}

\begin{ex}
Soit $A\in \M_{2n}(\R)$ telle que 
$$ A=\left(
\begin{array}{ccccc}
A_1&& &(0)\\
&A_2&& \\
  &  & \ddots& \\
(0) &     & &A_n
\end{array}
\right) \hspace{.3cm}\text{avec } A_i\in \M_2(\R)$$

\begin{enumerate}
\item Montrer que $\det A=\displaystyle  \prod_ {i=1} ^n \det A_i$.
\item Calculer 
$$ \det\left(
\begin{array}{cccccc}
a_1& & &b_1 & & \\
 &\ddots&&&\ddots & \\
  &  & a_n& & &b_n\\
c_1& & &d_1 & & \\
 &\ddots&&&\ddots & \\
  &  & c_n& & &d_n
\end{array}
\right)$$

\end{enumerate}

\end{ex}

 
 
 \begin{ex} 
Calculer le rang des matrices suivantes:
$$ A=\left(
\begin{array}{ccc}
1&5&1\\
0&2&-2\\
1&0&6
\end{array}
\right)
\text{, } B=\left(
\begin{array}{cccc}
1&1&1&1\\
0&1&2&-1\\
1&0&-2&3\\
2&1&0&-1
\end{array}
\right)
\text{et } C=\left(
\begin{array}{ccccc}
2&3&-3&4&2\\
3&6&-2&5&9\\
7&18&-2&7&7\\
2&4&-2&3&1
\end{array}
\right)
$$

\end{ex}

\begin{ex}
Calculer quand c'est possible l'inverse des matrices suivantes~:
%\vspace{-.3cm}
\[A=\left(\begin{array}{cc}1&2\\3&4\end{array}\right),\quad
B=\left(\begin{array}{cc}2&1\\-1&2\end{array}\right),\quad
C=\left(\begin{array}{cc}1&1\\1&1\end{array}\right),\quad
D=\left(\begin{array}{cc}a&b\\c&d\end{array}\right),\]
\vspace{-.2cm}
\[%E=\left(\begin{array}{ccc}1&0&0\\0&a&b\\0&c&d\end{array}\right),\quad
E=\left (\begin {array}{ccc} 1&i&2\\1&0&1
\\2&i&2\end {array}\right),\quad
F=\left (\begin {array}{ccc}1&2&0\\i&i&-1\\0&2&i\end{array}\right),\quad
G=
\left(
\begin{array}{ccc}
t&-3&9\\
1& 0& -1\\
1&1&t
\end{array}
\right)
.\]


%\sss Pour quelle(s) valeur(s) de $\a\in\RM$ la matrice
%$\left(\begin{array}{ccc}1&2&\a\\1&2&1\\2&4&2\end{array}\right)$
%est-elle inversible~?
\end{ex}

\begin{ex}
\begin{enumerate}
\item
 Soit $A$ une matrice 2x2. Calculer $P_A(X)=\det(A-XI)$.
Que reconnaissez-vous dans les coefficients de $P$?
\item
 Calculer le polynome $P_A$ pour 
$$
A= \left(
\begin{array}{cc}
1&2\\[5pt]
3&4
\end{array}
\right).
$$
\item Calculer $P_A(A)$. Obtiendra-t-on toujours ce resultat?
\end{enumerate}
\end{ex}

\begin{ex}
Soit $A\in \M_n(\R)$, montrer que:
\begin{center} $ \rg (A)=n\Longrightarrow \rg (\cm A)=n$\\
 $ \rg (A)=n-1\Longrightarrow \rg (\cm A)=1$\\
$ \rg (A)\leq n-2\Longrightarrow \rg (\cm A)=0$\\
\end{center}
Ces implications sont-elles des equivalences?
\end{ex}

 %%%%%%%%%%%%%%%%%%%%%%%%%%%%%%%
 \section{Encore des matrices de passage}
 %%%%%%%%%%%%%%%%%%%%%%%%%%%%%%%
\begin{ex}

Dans $\R^3$, on note $\BC=(e_1,e_2,e_3)$ la base canonique et
$f_1=(1,2,3)$, $f_2=(-1,-2,3)$, $f_3=(0,1,0)$.
\begin{enumerate}
\item Montrer que $\CC=(f_1,f_2,f_4)$ est une base de $\R^3$.

\item Quelles sont les coordonn\'ees du vecteur $(1,1,0)$ dans les bases $\BC$ et $\CC$?

\item M\^emes questions avec $(x_1,x_2,x_3)\in\R^3$.
Interpr\'eter avec la matrice de passage.

\end{enumerate}
\end{ex}



\begin{ex}
Soit $E=\C^3$, vu comme espace vectoriel sur $\C$, et
$\BC=(e_1,e_2,e_3)$ sa base canonique.
On consid\`ere $f_1=e_1-e_2$, $f_2=-ie_2+e_3$, $f_3=e_1-e_2+ie_3$.
\begin{enumerate}
\item 
Montrer que $\CC=(f_1,f_2,f_3)$ est une base de $E$.

\item D\'eterminer la matrice de passage $P$ de $\BC$ \`a $\CC$ ainsi que son
inverse $P^{-1}$.

\item Soit
\[
A=\left(\begin{array}{ccc}-2i&-2i&1\\2(1+i)&1+2i&-1+i\\1+2i&1+i&-1\end{array}\right).\]

Est-ce que $A$ est la matrice de passage de $\BC$ \`a une base $\DC$ de
$E$~? Si oui, quelle est la matrice de passage de $\CC$ \`a $\DC$~?
\end{enumerate}
\end{ex}




\begin{ex}

Soit $ P=\left(
\begin{array}{cc}
1&1\\
1&1
\end{array}
\right)$
 et soit $f:\M_2(\R) \longrightarrow \M_2(\R)$ d\'efinie par $f(M)=PM$ pour $M\in \M_2(\R)$.
\vspace{.1cm}
\begin{enumerate}
\item Montrer que $f$ est un endomorphisme de $\M_2(\R)$.\vspace{.15cm}
\item Soit $ e_1=\left(
\begin{array}{cc}
1&0\\
0&0
\end{array}
\right), 
e_2=\left(
\begin{array}{cc}
0&1\\
0&0
\end{array}
\right), 
e_3=\left(
\begin{array}{cc}
0&0\\
1&0
\end{array}
\right), e_4=\left(
\begin{array}{cc}
0&0\\
0&1
\end{array}
\right)$, et $\BC:=(e_1,e_2,e_3,e_4).$\vspace{.25cm}\\
Donner la matrice de $f$ dans la base $\BC$.
\item D\'eterminer et donner des bases de $\ker f$ et $\im f$. Montrer $\M_2(\R)=\ker f\oplus \im f$.
%\item En d\'eduire des bases $\CC$ et $\CC'$ telles que la matrice de $f$ dans les bases $\CC$ et $\CC'$ soit diagonale.
%\item Donner les matrices de passage de $\BC$ \`a $\CC$ et de $\BC$ \`a $\CC'$  ainsi que leurs inverses.
\vspace{.15cm}
\item Soit $\DC:=(f_1,f_2,f_3,f_4)$ o\`u 
$ f_1=\left(
\begin{array}{cc}
1&0\\
0&1
\end{array}
\right), 
f_2=\left(
\begin{array}{cc}
0&1\\
1&0
\end{array}
\right), 
f_3=\left(
\begin{array}{cc}
1&0\\
1&0
\end{array}
\right), f_4=\left(
\begin{array}{cc}
0&-1\\
0&1
\end{array}
\right)$\vspace{.25cm}\\
Montrer que $\DC$ est une base, donner la matrice de passage de $\DC$ \`a $\BC$ et la matrice de $f$ dans la base $\DC$.
%\item Donner les matrices de passage de $\DC$ \`a $\CC$ et de $\DC$ \`a $\CC'$  ainsi que leurs inverses.
\end{enumerate}
\end{ex}


\end{document}


