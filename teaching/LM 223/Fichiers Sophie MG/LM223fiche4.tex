\documentclass[a4paper,10pt]{article}
\usepackage[francais]{babel}
\usepackage[latin1]{inputenc}
\usepackage[dvips,final]{graphics}
\usepackage{amsmath,amsfonts,amssymb}
\usepackage{theorem}
\usepackage[T1]{fontenc}
%\usepackage[cp850]{inputenc}
%\usepackage[textures]{epsfig}
%\usepackage{alltt}
%\renewcommand{\baselinestretch}{1,25}
%\psfigdriver{dvips}

\pagestyle{empty}
% my margins

\addtolength{\oddsidemargin}{-.875in}
\addtolength{\evensidemargin}{-.875in}
\addtolength{\textwidth}{1.75in}

\addtolength{\topmargin}{-.875in}
\addtolength{\textheight}{1.75in}
\theoremstyle{plain}
\theorembodyfont{\upshape}
\newtheorem{thm}{Th\'eoreme}[section]
\newtheorem{cor}[thm]{Corollaire}
\newtheorem{lem}[thm]{Lemme}
\newtheorem{prop}[thm]{Proposition}
\newtheorem{defn}[thm]{D\'efinition}
\newtheorem{rem}[thm]{Remarque}
\newtheorem{exercice}[thm]{Ex}{\theorembodyfont{\upshape}}


\newcommand{\R}{\mathbb{R}}
\newcommand{\Z}{\mathbb{Z}}
\newcommand{\C}{\mathbb{C}}
\newcommand{\N}{\mathbb{N}}
\newcommand{\T}{\mathbb{T}}
\newcommand{\Q}{\mathbb{Q}}
\newcommand{\D}{\mathbb{D}}
\newcommand{\K}{\mathbb{K}}
\newcommand{\Li}{\mathcal{L}}
\newcommand{\M}{\mathcal{M}}
\renewcommand{\S}{\mathbb{S}}

\newcommand{\im}{\textup{Im}}
\newcommand{\BC}{\mathcal{B}}
\newcommand{\CC}{\mathcal{C}}
\newcommand{\DC}{\mathcal{D}}
\newcommand{\can}{_{\mathrm{can}}}

\newcommand{\rg}{\textup{rg}}
\newcommand{\cm}{\textup{Comat}}


\newcommand{\pmm}[1]{\ \text{#1}} %pour ins�rer la ponctuation dans les �quations hors-texte avec tjrs le m�me espace
\newcommand{\std}[1]{\mathbf{#1}}
\newcommand{\I}[1][]{\std{I}_{#1}} \newcommand{\ind}[1]{\std{1}_{#1}}
\renewcommand{\ge}{\geqslant} \renewcommand{\le}{\leqslant}
\newcommand{\eps}{\varepsilon} \newcommand{\truc}{\,\cdot\,}

\DeclareMathOperator{\tr}{tr}

\begin{document}
\noindent
\large
\textbf{Universit\'e Pierre et Marie Curie 
 - LM223 -
Ann\'ee 2012-2013}\\

\begin{center}
\Large
\textbf{Fiche d'exercices n$^o$ 4}
\end{center}
\normalsize
\section{Formules sur les formes quadratiques g\'en\'erales}

\begin{exercice}
Soit $E$ un espace vectoriel sur $\K=\R$ ou $\C$ et une application $f:E\rightarrow \K$ telle que
\begin{enumerate}
\item[(i)] $f(\lambda x)=\lambda^2 f(x)$, pour tous $x\in E$, $\lambda\in \K$,
\item[(ii)] $(x,y)\mapsto \frac{1}{2}\big(f(x+y)-f(x)-f(y)\big)$ de $E\times E$ dans $\K$, est lin\'eaire par rapport \`a la premi\`ere variable.
\end{enumerate}
Montrer que $f$ est une forme quadratique.
\end{exercice} 

\begin{exercice}
Soit $E$ un espace vectoriel sur $\K=\R$ ou $\C$  et  $q$ une forme quadratique sur $E$.
\begin{enumerate}
\item Montrer la \textit{formule du parall\'elogramme}:
$$
 \forall \; x,y\in E, \quad q(x+y)+q(x-y)=2\big(q(x)+q(y)\big) \qquad \text{(PA)}.
$$
\item Montrer la formule de polarisation d'ordre 3:
$$
 \forall \; x,y,z\in E, \quad q(x+y)+q(y+z)+q(x+z)=q(x)+q(y)+q(z)+q(x+y+z)  \qquad \text{(PO)}.
$$
\item Expliquer pourquoi (PO) implique (PA).
\end{enumerate}
\end{exercice}

\begin{exercice}
Soit $q$ une forme quadratique positive (i.e. $q(x)\geq 0$, $\forall\; x$) sur un $\R$-e.v $E$ et $b$ sa forme polaire.
\begin{enumerate}
\item Montrer  l'\textit{in\'egalit\'e de Cauchy-Schwarz}
$$
 \forall \; x,y\in E, \quad b(x,y)^2\leq q(x)q(y) \qquad \text{(CS)} .
$$
(Indication: consid\'erer l'application $P:\R \rightarrow \R, t\mapsto q(x+ty)$ avec $x$ et $y$ fix\'es.)
\item Montrer que si $x$ et y sont li\'es alors (CS) est une \'egalit\'e. La r\'eciproque est-elle vraie?
\item On suppose de plus que $q$ est d\'efinie. Montrer que si pour $x,y\in E$ (CS) est une \'egalit\'e,
alors $x$ et $y$ sont li\'es. (Indication: utiliser \`a nouveau $P$.)
\end{enumerate}

\end{exercice}

\begin{exercice}
Soit $q$ une forme quadratique positive (i.e. $q(x)\geq 0$, $\forall\; x$).
\begin{enumerate}
\item Montrer  l'\textit{in\'egalit\'e de Minkowski} (appel\'ee aussi i\textit{in\'egalit\'e triangulaire})
$$
 \forall \; x,y\in E, \quad \sqrt{q(x+y)}\leq \sqrt{q(x)}+\sqrt{q(y)} \qquad \text{(M)} .
$$
(Indication: utiliser Cauchy-Schwarz.)
\item On suppose de plus que $q$ est d\'efinie. Montrer que pour $x,y\in E$ (M) est une \'egalit\'e si et seulement si 
$x=\varrho y$ ou $y=\varrho x$ avec $\varrho \geq 0$.
\end{enumerate}
 \end{exercice}
 
 
\section{Formes quadratiques/ Repr\'esentations matricielles}
 \begin{exercice} \label{fqr4}
Trouver les formes polaires et le rang des formes quadratiques suivantes sur $\mathbb{R}^4$, donner les matrices dans la base canonique de $\R^4$.
\begin{enumerate}
\item $q(x,y,z,t)=xy+y^2$ 
\item $q(x,y,z,t)=xy+zt+t^2$ 
\item $q(x,y,z,t)=x^2-y^2+z^2-t^2$
\end{enumerate}
\end{exercice}

\begin{exercice}
Montrer que le d\'eterminant est une forme quadratique sur $M_2(\R)$. Donner sa forme polaire, donner sa matrice
dans la base canonique $\{E_{11}, E_{12}, E_{21}, E_{22}\}$. Cette forme est-elle d\'eg\'en\'er\'ee, d\'efinie, positive?
\end{exercice}

\begin{exercice}
Soient $A$ et $B$ deux matrices sym\'etriques de $M_n(\K)$ telles que $\forall$ X $\in \mathbb{R}^n$, $^tXAX=$ $^tXBX$ (*)
\begin{enumerate}
\item Soient X, Y $\in \mathbb{R}^n$, en appliquant (*) \`a X, Y et X+Y, montrer que $^tXAY=$ $^tXBY$
\item En rempla\c{c}ant X et Y par des vecteurs de la base canonique, en d\'eduire que $A=B$.
 \end{enumerate}
 Que peut-on d\'eduire pour les formes quadratiques ?
\end{exercice}

\section{Orthogonalit\'e-Isotropie}

\begin{exercice}
Soit $q$ une forme quadratique sur $\R^n$, $b$ sa forme polaire et $Q$ sa matrice dans la base canonique.
\begin{enumerate}
\item Si $Q=\mathrm{Id}$, existe-t-il un vecteur $x\in \R^n$ tel que $b(x,y)=0$ quel que soit $y\in \R^n$? 
Que vaut le noyau $N(q)$, le c\^one $C(q)$?
\item On suppose ici que $\rg(Q)=n$. 
\begin{enumerate}
\item Expliquer pourquoi pour tout $x\in \R^n$ il existe $y\in \R^n$ tel que $x=Qy$.
\item Que vaut le noyau $N(q)$, le c\^one $C(q)$?
 \end{enumerate}
\item On suppose ici que $\rg(Q)<n$. 
\begin{enumerate}
\item Expliquer pourquoi il existe $x\in \R^n$ tel que $Qx=0$.
\item Montrer que $N(q)\not=\{0\}$.
 \end{enumerate}
 \item En d\'eduire que $N(q)=\{0\}$ si et seulement si $\rg(Q)=n$. 
 \end{enumerate}
 \end{exercice}


\begin{exercice}
 Soit $q: E \rightarrow K$ une forme quadratique sur un espace vectoriel de dimension finie. 
\begin{enumerate} 
\item Pour tout sous-espace vectoriel $F \subset E$, montrer que l'on a ($F^\perp)^\perp=F+N(q)$.
\item  Pour tous sous-espaces vectoriels $F,G \subset E$, montrer que $(F+G)^{\perp}=F^{\perp} \cap G^{\perp}$. Si $q$ est non-d\'eg\'en\'er\'ee, montrer que l'on a $( F \cap G)^{\perp} =F^{\perp}+G^{\perp}$. 
\end{enumerate}
\end{exercice}

 
 \begin{exercice}
 Les formes quadratiques de l'exercice \ref{fqr4} sont-elles d\'eg\'en\'er\'ees? Sont-elles d\'efinies?
 \end{exercice}


 \begin{exercice}
 On consid\`ere la forme quadratique $q$ de $\R^4$ dont la matrice dans la base canonique est:
 $$
 \begin{pmatrix}
 1&0&\sqrt{3}&0\\
 0&-9&0&0\\
 \sqrt{3}&0 &4&\sqrt{3}\\
 0&0&\sqrt{3}&1
 \end{pmatrix}.
 $$
 \begin{enumerate} 
\item La forme $q$ est-elle d\'eg\'en\'er\'ee?
\item Existe-t-il des vecteurs $q$-isotropes non nuls?
\item On note $u=(0,-2,3,0)$. Que vaut $\dim \{u\}^\bot $? Donner une base de $ \{u\}^\bot $.
\end{enumerate}
 \end{exercice}
 
\begin{exercice}
Construire une base orthogonale pour chacune des formes de $\R^3$ ou $\R^4$ suivantes: 
 \begin{enumerate} 
\item $q(x,y,z)=x^2+y^2+xz$,
\item $q(x,y,z)=8xy-16xz-8yz$,
\item $q(x,y,z)=2x^2+5y^2+19z^2-8xy+12xz-18yz$,
\item $q(x,y,z,t)=x^2+2xy-4xt+3y^2+8yz+6y^2-2t^2$.
\end{enumerate}
 \end{exercice}
 
\begin{exercice}
Diagonaliser et donner la signature des formes de $\R^3$ ou $\R^4$ suivantes: 
 \begin{enumerate} 
\item $q(x_1,x_2,x_3)=x_1^2+3x_2^2+8x_3^2-4x_1x_2+6x_1x_3-10x_2x_3$,
\item $q(x_1,x_2,x_3)=4x_1^2+2x_2^2-3x_1x_2+2x_1x_3-4x_2x_3$,
\item $q(x_1,x_2,x_3, x_4)=4x_1x_2+4x_2x_3-2x_3x_4$.
\end{enumerate}
\end{exercice}

\begin{exercice}
Soit $E=\R_n[X]$ et $q$ la forme quadratique sur $E$ donn\'ee par $P\mapsto \int_{-1}^1P^2(t)dt$.
\begin{enumerate} 
\item Donner la forme polaire $b$ associ\'ee \`a $q$.
\item On pose $p_0=1$, et $p_k=\frac{d^k}{dX^k}((X^2-1)^k)$. Montrer que $p_k\in \{1, X, X^2,\ldots, X^{k-1}\}^\bot$.
\item En d\'eduire une base $q$-orthogonale de $E$.
\end{enumerate}
\end{exercice}

\end{document}


