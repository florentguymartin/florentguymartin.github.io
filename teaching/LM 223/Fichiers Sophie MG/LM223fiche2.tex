\documentclass[a4paper,11pt]{article}
\usepackage[francais]{babel}
\usepackage[latin1]{inputenc}
\usepackage[dvips,final]{graphics}
\usepackage{amsmath,amsfonts,amssymb}
\usepackage{theorem}
\usepackage[T1]{fontenc}
%\usepackage[cp850]{inputenc}
%\usepackage[textures]{epsfig}
%\usepackage{alltt}
%\renewcommand{\baselinestretch}{1,25}
%\psfigdriver{dvips}

\pagestyle{empty}
% my margins

\addtolength{\oddsidemargin}{-.875in}
\addtolength{\evensidemargin}{-.875in}
\addtolength{\textwidth}{1.75in}

\addtolength{\topmargin}{-.875in}
\addtolength{\textheight}{1.75in}
\theoremstyle{plain}
\theorembodyfont{\upshape}
\newtheorem{thm}{Th\'eoreme}[section]
\newtheorem{cor}[thm]{Corollaire}
\newtheorem{lem}[thm]{Lemme}
\newtheorem{prop}[thm]{Proposition}
\newtheorem{defn}[thm]{D\'efinition}
\newtheorem{rem}[thm]{Remarque}
\newtheorem{ex}[thm]{Ex}{\theorembodyfont{\upshape}}


\newcommand{\R}{\mathbb{R}}
\newcommand{\Z}{\mathbb{Z}}
\newcommand{\C}{\mathbb{C}}
\newcommand{\N}{\mathbb{N}}
\newcommand{\T}{\mathbb{T}}
\newcommand{\Q}{\mathbb{Q}}
\newcommand{\D}{\mathbb{D}}
\newcommand{\K}{\mathbb{K}}
\newcommand{\Li}{\mathcal{L}}
\newcommand{\M}{\mathcal{M}}
\renewcommand{\S}{\mathbb{S}}

\newcommand{\im}{\textup{Im}}
\newcommand{\BC}{\mathcal{B}}
\newcommand{\CC}{\mathcal{C}}
\newcommand{\DC}{\mathcal{D}}
\newcommand{\can}{_{\mathrm{can}}}

\newcommand{\rg}{\textup{rg}}
\newcommand{\cm}{\textup{Comat}}

\begin{document}
\noindent
\large
\textbf{Universit\'e Pierre et Marie Curie 
 - LM223 -
Ann\'ee 2012-2013}\\

\begin{center}
\Large
\textbf{Fiche d'exercices n$^o$ 2}
\end{center}
\normalsize
 \section{Valeurs propres.}



\begin{ex}
Soit $E$ un espace vectoriel de dimension $n$, $n\geq 1$, et soient $u$, $v$ deux endomorphismes de $E$. 
\begin{enumerate}
\item Montrer que si 0 est valeur propre de $u\circ v$ alors 0 est aussi valeur propre de $v\circ u$.
\item Montrer que $u\circ v$  et $v\circ u$ ont le m\^eme ensemble de valeur propre.

\end{enumerate}
\end{ex}

\begin{ex}
Soit $E$ un espace vectoriel de dimension $n$, $n\geq 1$, et soit $u$ un endomorphisme nilpotent (i.e $\exists k\in \N$,
tel que $u^k=0$) non nul de $E$. 
\begin{enumerate}
\item On note $Sp_u$ l'ensemble des valeurs propres de $u$.
\begin{enumerate}
\item Montrer que $0\in Sp_u $, puis que $Sp_u =\lbrace 0\rbrace$.
\item $u$ est-il diagonalisable? 
\end{enumerate}
\item On consid\`ere maintenant l'endomorphisme $id-u$.
\begin{enumerate}
\item En utilisant le fait que $u^k=0$ et une identit\'e remarquable, montrer que $id-u$ est inversible et donner son inverse.
\item Quelles sont les valeurs propres de $id-u$?
\item $id-u$ est-il diagonalisable? 

\end{enumerate}

\end{enumerate}
\end{ex}




\begin{ex}
Soit $E$ un $\C$-espace vectoriel de dimension $n$, $n\geq 1$.
\begin{enumerate}
\item Montrer que tout endomorphisme de $E$ poss\`ede au moins une valeur propre.
\item Soient $u$, $v$ deux endomorphismes de $E$ tels que $u\circ v=v\circ u$.

\begin{enumerate}
\item Soit $\lambda $ une valeur propre de $u$ et soit $E^u_\lambda $ le sous espace propre de $u$
associ\'e \`a la valeur propre $\lambda $. Montrer que $E^u_\lambda $ est stable par $v$.
\item En d\'eduire qu'il existe dans $E^u_\lambda $ un vecteur propre de $v$.
\item On suppose que $u$ poss\`ede $n$ valeurs propres distinctes. Montrer que $u$ et $v$ sont tous deux diagonalisables,
et qu'il existe une base $\BC$ de $E$ telle que Mat$_\BC(u)$ et Mat$_\BC(v)$ sont diagonales.
\end{enumerate}
\end{enumerate}
\end{ex}


 \section{Diagonalisation.}

\begin{ex}
Pour les matrices r\'eelles suivantes, trouver les valeurs propres, d\'eterminer si la matrice est diagonalisable et dans ce cas donner 
une base de vecteurs propres:
$$\left(
\begin{array}{ccc}
0&1 &2\\
 -1&2&2 \\
  1&0  &0
\end{array}
\right),
\left(
\begin{array}{ccc}
2&0 &0\\
 -3&-1&3 \\
  3&3  &-1
\end{array}
\right),
\left(
\begin{array}{ccc}
3&1 &0\\
 -4&-1&0 \\
  4&-8  &2
\end{array}
\right),
\left(
\begin{array}{ccc}
1&0 &0\\
 0&0&-1 \\
  0&1&2
\end{array}
\right),
\left(
\begin{array}{ccc}
0&0&1\\
 1&0&-3 \\
  0&1  &3
\end{array}
\right)$$


\end{ex}

\begin{ex}
On consid\`ere:
$$A=\left(
\begin{array}{ccc}
3&2&2\\
 2&3&2 \\
 2&2  &3
\end{array}
\right)$$
Calculer $A^n$ pour tout $n\in \N$.

\end{ex}

\newpage
\begin{ex}

Soit $A$ la matrice $n\times n$ dont tous les coefficients valent
$1$. 
\begin{enumerate}
\item Calculer $A^2$. En d\'eduire que si $\lambda $ est une valeur propre de
$A$, alors $\lambda $ vaut $0$ ou $n$. 
\item D\'eterminer la dimension des espaces
propres de $A$~: $A$ est-elle diagonalisable~?
\end{enumerate}

\end{ex}


\begin{ex}
Soit $\Phi :\R_2[X]\rightarrow \R_2[X]$, l'application qui \`a tout polyn\^ome $P\in \R_2[X]$ associe le reste de la division 
euclidienne de $(X+1)P(X)$ par $X^3+1$.
\begin{enumerate}
\item Justifier que $\Phi $ est bien d\'efinie et montrer que $\Phi $ est lin\'eaire.
\item Ecrire la matrice $M$ de $\Phi $ dans la base canonique $(1,X,X^2)$.
\item Quelles sont les valeurs propres de $M$? $M$ est-elle diagonalisable?
\end{enumerate}
\end{ex}





\begin{ex}
On consid\`ere l'application lin\'eaire suivante:
 $$\begin{array}{cccl}\Phi:&\M_2(\R)&\longrightarrow&\M_2(\R)\\
&M&\longmapsto&AM\end{array}\hspace{1cm}\text{o\`u }A=\left(\begin{array}{cc}2&1\\1&2\end{array}\right)$$

\begin{enumerate}
\item Ecrire la matrice de $\Phi$ dans la base canonique de $\M_2(\R)$.

\item Est-ce que $\Phi$ est diagonalisable~? Si oui donner la base de $\M_2(\R)$ dans laquelle la matrice de $\Phi$ est diagonale. 
\end{enumerate}
\end{ex}


\end{document}


