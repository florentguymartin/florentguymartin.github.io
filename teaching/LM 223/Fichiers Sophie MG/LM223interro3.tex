\documentclass[a4paper, 11pt]{article}
\usepackage[francais]{babel}
\usepackage[latin1]{inputenc}
\usepackage[dvips,final]{graphics}
\usepackage{amsmath,amsfonts,amssymb}
\usepackage{theorem}
\usepackage[T1]{fontenc}
%\usepackage[cp850]{inputenc}
%\usepackage[textures]{epsfig}
%\usepackage{alltt}
%\renewcommand{\baselinestretch}{1,25}
%\psfigdriver{dvips}

\pagestyle{empty}
% my margins

\addtolength{\oddsidemargin}{-.875in}
\addtolength{\evensidemargin}{-.875in}
\addtolength{\textwidth}{1.75in}

\addtolength{\topmargin}{-.875in}
\addtolength{\textheight}{1.75in}
\theoremstyle{plain}
\theorembodyfont{\upshape}
\newtheorem{thm}{Th\'eoreme}
\newtheorem{cor}[thm]{Corollaire}
\newtheorem{lem}[thm]{Lemme}
\newtheorem{prop}[thm]{Proposition}
\newtheorem{defn}[thm]{D\'efinition}
\newtheorem{rem}[thm]{Remarque}
\newtheorem{ex}{Exercice}{\theorembodyfont{\upshape}}

\title{Fiche 1\\Espaces Vectoriels, Applications lin�aires}
\newcommand{\R}{\mathbb{R}}
\newcommand{\Z}{\mathbb{Z}}
\newcommand{\C}{\mathbb{C}}
\newcommand{\N}{\mathbb{N}}
\newcommand{\T}{\mathbb{T}}
\newcommand{\Q}{\mathbb{Q}}
\newcommand{\D}{\mathbb{D}}
\newcommand{\K}{\mathbb{K}}
\newcommand{\Li}{\mathcal{L}}
\newcommand{\M}{\mathcal{M}}
\newcommand{\F}{\mathbf{F}}
\renewcommand{\S}{\mathbb{S}}

\newcommand{\im}{\textup{Im}}
\newcommand{\BC}{\mathcal{B}}
\newcommand{\CC}{\mathcal{C}}
\newcommand{\DC}{\mathcal{D}}
\newcommand{\can}{_{\mathrm{can}}}

\newcommand{\Mat}{\textup{Mat}}
\newcommand{\Vect}{\mathrm{Vect}}
\newcommand{\rg}{\textup{rg}}
\newcommand{\cm}{\textup{Comat}}

\begin{document}
\noindent
\large
\textbf{Universit\'e Pierre et Marie Curie 
 - LM223 -
Ann\'ee 2011-2012}\\

\begin{center}
\Large
\textbf{Interro n$^o$ 3}
\end{center}
\normalsize


\bigskip
\noindent
\textbf{Exercice 1:}
\begin{enumerate}
\item Donner la d\'efinition de ``forme quadratique''.
\item Donner la d\'efinition de ``produit scalaire''.
\item Donner un exemple (sans d\'emonstration) de produit scalaire $b:E\times E\rightarrow \R$ dans chacun des cas suivants:
\begin{enumerate}
\item $E=\R^n$,
\item $E=M_n(\R)$,
\item $E=\R_n[X]$.
\end{enumerate}
\end{enumerate}

\bigskip
\noindent
\textbf{Exercice 2:}\\
On consid\`ere la forme quadratique $q: \R^3\rightarrow \R$ donn\'ee par 
$$
q(x)=x_1^2+x_2^2+4x_3^2-2x_1x_2-4x_1x_3, \quad \forall x=(x_1,x_2,x_3)\in\R^3.
$$
On note $b$ la forme polaire de $q$.
\begin{enumerate}
\item \'Ecrire la matrice $M$ de $q$ dans la base canonique.
\item Donner l'expression de $b\big((x_1,x_2,x_3),(y_1,y_2,y_3)\big)$.
\item \'Ecrire $q$ comme combinaison de carr\'es de formes lin\'eairement ind\'ependantes.
\item Quel est le rang de $q$? Quelle est sa signature? Justifier les r\'eponses.
\item La forme $q$ est-elle d\'eg\'en\'er\'ee? Est-elle d\'efinie? Justifier les r\'eponses.
\item La forme $b$ est-elle un produit scalaire? Justifier la r\'eponse.
\item Donner une base $q$-orthogonale de $\R^3$.
\item Donner une matrice inversible $P$ telle que $^tPMP$ soit diagonale. Pr\'eciser alors la matrice diagonale obtenue.
\end{enumerate}


\bigskip
\noindent
\textbf{Exercice 3:}\\
On consid\`ere une forme quadratique $q:E\rightarrow \R$, o\`u $E$ est un espace vectoriel r\'eel de dimension 3, de base $\BC=\{e_1,e_2,e_3\}$.
La matrice de $q$ dans la base $\BC$ est
$$
Q=
\begin{pmatrix}
3&1&-5\\
1&0&-2\\
-5&-2&6
\end{pmatrix}
$$
\begin{enumerate}
  \item Montrer que le vecteur $u=3e_2+2e_3$ est un vecteur $q$-isotrope.
  \item D\'eterminer $\{u\}^\bot$.
  \item Si $E=\R_2[X]$ et $\BC=\{1,X,X^2\}$, trouver deux polyn\^omes, non proportionnels, $q$-orthogonaux \`a $P=3X+2X^2$.
  \end{enumerate}  

\end{document}