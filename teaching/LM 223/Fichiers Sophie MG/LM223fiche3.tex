\documentclass[a4paper,10pt]{article}
\usepackage[francais]{babel}
\usepackage[latin1]{inputenc}
\usepackage[dvips,final]{graphics}
\usepackage{amsmath,amsfonts,amssymb}
\usepackage{theorem}
\usepackage[T1]{fontenc}
%\usepackage[cp850]{inputenc}
%\usepackage[textures]{epsfig}
%\usepackage{alltt}
%\renewcommand{\baselinestretch}{1,25}
%\psfigdriver{dvips}

\pagestyle{empty}
% my margins

\addtolength{\oddsidemargin}{-.875in}
\addtolength{\evensidemargin}{-.875in}
\addtolength{\textwidth}{1.75in}

\addtolength{\topmargin}{-.875in}
\addtolength{\textheight}{1.75in}
\theoremstyle{plain}
\theorembodyfont{\upshape}
\newtheorem{thm}{Th\'eoreme}[section]
\newtheorem{cor}[thm]{Corollaire}
\newtheorem{lem}[thm]{Lemme}
\newtheorem{prop}[thm]{Proposition}
\newtheorem{defn}[thm]{D\'efinition}
\newtheorem{rem}[thm]{Remarque}
\newtheorem{exercice}[thm]{Ex}{\theorembodyfont{\upshape}}


\newcommand{\R}{\mathbb{R}}
\newcommand{\Z}{\mathbb{Z}}
\newcommand{\C}{\mathbb{C}}
\newcommand{\N}{\mathbb{N}}
\newcommand{\T}{\mathbb{T}}
\newcommand{\Q}{\mathbb{Q}}
\newcommand{\D}{\mathbb{D}}
\newcommand{\K}{\mathbb{K}}
\newcommand{\Li}{\mathcal{L}}
\newcommand{\M}{\mathcal{M}}
\renewcommand{\S}{\mathbb{S}}

\newcommand{\im}{\textup{Im}}
\newcommand{\BC}{\mathcal{B}}
\newcommand{\CC}{\mathcal{C}}
\newcommand{\DC}{\mathcal{D}}
\newcommand{\can}{_{\mathrm{can}}}

\newcommand{\rg}{\textup{rg}}
\newcommand{\cm}{\textup{Comat}}


\newcommand{\pmm}[1]{\ \text{#1}} %pour ins�rer la ponctuation dans les �quations hors-texte avec tjrs le m�me espace
\newcommand{\std}[1]{\mathbf{#1}}
\newcommand{\I}[1][]{\std{I}_{#1}} \newcommand{\ind}[1]{\std{1}_{#1}}
\renewcommand{\ge}{\geqslant} \renewcommand{\le}{\leqslant}
\newcommand{\eps}{\varepsilon} \newcommand{\truc}{\,\cdot\,}

\DeclareMathOperator{\tr}{tr}

\begin{document}
\noindent
\large
\textbf{Universit\'e Pierre et Marie Curie 
 - LM223 -
Ann\'ee 2012-2013}\\

\begin{center}
\Large
\textbf{Fiche d'exercices n$^o$ 3}
\end{center}
\normalsize
\section{Formes lin\'eaires et espace dual}

\begin{exercice}
Dans $\R^2$, on consid\`ere la base $v_1= \begin{pmatrix} 2\\1 \end{pmatrix}$, $v_2= \begin{pmatrix} 5\\3 \end{pmatrix}$. Calculer la base duale.
\end{exercice}

\begin{exercice}
Donner une base de l'espace des matrices $M_2(\R)$ et sa duale.
\end{exercice}

\begin{exercice}
On consid\`ere l'espace vectoriel des polyn\^omes r\'eels $E=\R_2[X]$. 
On d\'efinit trois fonctions $f_0,f_1,f_2$ de $E$ vers $\R$ par $f_i(P)=P(i)$ pour tout $P$ dans $E$.
\begin{enumerate}
\item Montrer que les $f_i$ sont des applications lin\'eaires.
\item Montrer que $\{ f_0,f_1,f_2\}$ est une base de $E^*$.
\item Trouver la base pr\'eduale $\{ e_0,e_1,e_2\}$ de E, c'est-\`a-dire la base telle que $e_i^*=f_i$, $i=0,1,2$.
\end{enumerate}
\end{exercice}

\begin{exercice}
Si $H$ est un sous-espace vectoriel d'un espace vectoriel $E$, on pose 
$H^{\bot} := \{ \varphi \in E^* \ \big| \ \varphi_{|H} =0 \}$. \\
Soient $F$ et $G$ deux sous-espaces d'un espace vectoriel $E$. Montrer que :
\begin{enumerate}
\item $F \subset G \Rightarrow G^{\bot} \subset F^{\bot}$,
\item $(F + G)^\bot = F^\bot \cap G^\bot$,
\item $(F\cap G)^\bot = F^\bot + G^\bot$,
\item $E=F\oplus G \Rightarrow E^*=F^\bot\oplus G^\bot.$
\end{enumerate}
\end{exercice}

\begin{exercice} Calcul du dual de $M_n(\C)$.
\begin{enumerate}
\item Soit $A \in M_n(\C)$. Montrer que l'application $\varphi_A : M_n(\C) \to \C,\ M \mapsto \tr(AM)$ est un \'el\'ement de $M_n(\C)^*$.
\item Montrer alors que l'application $\varphi : M_n(\C) \to M_n(\C)^*,\ A \mapsto \varphi_A$ est lin\'eaire et injective.
\item En d\'eduire que tout \'el\'ement de $M_n(\C)^*$ s'\'ecrit $\varphi_A$ pour un unique $A \in M_n(\C)$.
\end{enumerate}
\end{exercice}

\section{Formes bilin\'eaires}

\begin{exercice}
Soit $E$ un $\std{K}$-espace vectoriel ($\std{K}=\R$ ou $\C$). On
note $\mathcal{S}_2(E)$ (resp. $\mathcal{A}_2(E)$) l'espace des
formes bilin\'eaires sym\'etriques (resp. antisym\'etriques) de $E$.
Montrer que~:
\begin{equation*}
\mathcal{L}_2(E)=\mathcal{S}_2(E)\oplus\mathcal{A}_2(E).
\end{equation*}
\end{exercice}

\begin{exercice}\label{exercice:ps}
Dans chacun des exemples suivants, montrer que $\varphi$ est un
produit scalaire sur $E$.
\begin{enumerate}
  \item $E=\R^n$, $n\ge2$ et
  $\varphi(x,y)=\sum\limits_{i=1}^{n}x_iy_i$,
  \item $E=M_n(\R)$, $n\ge2$ et $\varphi(A,B)=\tr(^{t}\!AB)$,
  \item\label{ps:poly} $E=\R_n\lbrack X\rbrack$, $n\ge2$ et
  $\varphi(P,Q)=\int_0^1P(t)Q(t)dt$,
  \item $E=\ell^2(\N)=\big\{u=(u_n)_n\in\R^\N\, \big\lvert\, \sum\limits_{n\ge
  0}u_n^2<+\infty\big\}$ et
  $\varphi(u,v)=\sum\limits_{n=0}^{\infty}u_nv_n$. \\
  \lbrack Dans le 4., commencer par montrer que $\varphi$ est bien
  d\'efinie.\rbrack
\end{enumerate}
\end{exercice}

\newpage
\begin{exercice}
Soient $A$ et $B$ deux matrices de $M_n(\R)$, on pose
\begin{equation*}
    \varphi(A,B)=\tr(AB).
\end{equation*}
\begin{enumerate}
  \item Si $A=((a_{ij}))_{1\le i,j\le n}$ et $B=((b_{ij}))_{1\le i,j\le
  n}$, montrer que~:
  \begin{equation*}
    \tr(AB)=\sum_{1\le i,j\le n}a_{ij}b_{ji}.
  \end{equation*}
  \item La forme $\varphi$ est-elle bilin\'eaire? sym\'etrique?
  \item Supposons \`a pr\'esent, $A$ sym\'etrique et $B$ antisym\'etrique.
  Montrer alors~:
  \begin{itemize}
    \item $\varphi(A,A)\ge 0$,
    \item $\varphi(B,B)\le 0$,
    \item $\varphi(A,B)=0$.
  \end{itemize}
  \item La forme $\varphi$ est-elle d\'eg\'en\'er\'ee? Est-elle un produit scalaire?
  \item On note $S_n$, resp. $AS_n$, le s.e.v des matrices sym\'etriques, resp. anti-sym\'etriques. 
  Donner $S_n^\bot$ et $AS_n^\bot$.
\end{enumerate}
\end{exercice}

\begin{exercice}
On consid\`ere l'application suivante d\'efinie sur $\R^3\times\R^3$~:
\begin{equation*}
    \big((x,y,z),(x',y',z')\big)\mapsto \Bigg(\begin{vmatrix}y & z \\
    y' & z'\\ \end{vmatrix},\begin{vmatrix}z & x \\
    z' & x'\\ \end{vmatrix},\begin{vmatrix}x & y \\
    x' & y'\\ \end{vmatrix}\Bigg)=(x,y,z)\wedge(x',y',z').
\end{equation*}
\begin{enumerate}
  \item Montrer qu'elle est bilin\'eaire altern\'ee.
  \item Montrer que si $
{e_1}$ et $
{e_2}$ sont
  deux vecteurs de $\R^3$, alors~:
$$
    (
{e_1}\wedge
{e_2})\cdot
{e_1} = 0, \qquad
    (
{e_1}\wedge
{e_2})\cdot
{e_2} = 0.
$$
  \item En d\'eduire que, si $e_1$ et $e_2$ sont lin\'eairement ind\'ependants, \[\textrm{Vect}(
{e_1},
{e_2})=
  \textrm{Vect}(
{e_1}\wedge
{e_2})^\bot=\big\{
{x}\in\R^3\, \big\lvert\,
  (
{e_1}\wedge
{e_2})\cdot
{x}=0\big\}\]
\end{enumerate}
\noindent\emph{Application.} D\'eterminer une \'equation du plan
engendr\'e par les vecteurs ${e_1}=(1,2,-3)$ et
$
{e_2}=(-2,0,1)$.
\end{exercice}

\begin{exercice}
On reprend le produit scalaire \ref{ps:poly}. de l'exercice
\ref{exercice:ps} avec $n=2$~:
\begin{equation*}
    \forall\, P,\, Q\in\R_2\lbrack X\rbrack,\quad
    \varphi(P,Q)=\int_0^1P(t)Q(t)dt.
    \end{equation*}
\'Ecrire la matrice de $\varphi$ dans la base canonique de
$\R_2\lbrack X\rbrack$. Faire de m\^eme dans la base
$\mathcal{B}=\big(1,X,X^2-X+\frac{1}{6}\big)$.
\end{exercice}


\begin{exercice}
On consid\`ere la forme bilin\'eaire $b:\R^3\times \R^3\rightarrow \R$
dont la matrice dans la base canonique est
$$
B=\dfrac{1}{9}
\begin{pmatrix}
2&-2&10\\
-2&11&8\\
10&8&5
\end{pmatrix}
$$
\begin{enumerate}
  \item La forme $b$ est-elle sym\'etrique? Que vaut $b\big((1,1,1),(-1,-1,-1)\big)$, $b\big((x_1,x_2,x_3),(x_1,x_2,x_3)\big)$?
  \item Montrer que $1$ et $-1$ sont des valeurs propres pour $B$. Pouvez-vous trouver une autre valeur propre pour $B$?
  \item Montrer que la matrice $B$ est diagonalisable et donner une matrice $P$ telle que 
  $P^{-1}BP$ soit diagonale.
  \item V\'erifier que pour la matrice $P$ trouv\'ee $^{t}\!PP$ est diagonale \`a coefficients strictement positifs.
  \item Construire \`a partir de $P$ une matice $P'$ telle que $^{t}\!P'P'=I$.
  \item En d\'eduire une base de $\R^3$ dans laquelle la matrice de $b$ est diagonale.
\end{enumerate}
\end{exercice}

\end{document}


