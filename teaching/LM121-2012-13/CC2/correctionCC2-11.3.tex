\documentclass{article}

\usepackage[utf8]{inputenc}
\usepackage[francais]{babel}
\usepackage{amsmath,amssymb,amsfonts}
\usepackage[T1]{fontenc}
\usepackage{mathrsfs}



\newcommand{\R}{\mathbb{R}}
\newcommand{\Z}{\mathbb{Z}}
\newcommand{\C}{\mathbb{C}}
\newcommand{\N}{\mathbb{N}}
\newcommand{\T}{\mathbb{T}}
\newcommand{\Q}{\mathbb{Q}}
\newcommand{\D}{\mathbb{D}}
\newcommand{\K}{\mathbb{K}}
\newcommand{\Li}{\mathcal{L}}
\newcommand{\M}{\mathcal{M}}
\newcommand{\F}{\mathbf{F}}
\renewcommand{\S}{\mathbb{S}}

\newcommand{\im}{\textup{Im}}
\newcommand{\BC}{\mathcal{B}}
\newcommand{\CC}{\mathcal{C}}
\newcommand{\DC}{\mathcal{D}}
\newcommand{\can}{_{\mathrm{can}}}

\newcommand{\Mat}{\textup{Mat}}
\newcommand{\Vect}{\mathrm{Vect}}
\newcommand{\rg}{\textup{rg}}
\newcommand{\cm}{\textup{Comat}}


\begin{document}
\large
\noindent 
\textbf{Université Pierre et Marie Curie - LM 121 - 2012/2013}\\
\begin{center}
\Large 
Correction du Contrôle continu n 2 MMIME 11.3
\end{center}
\medskip
\noindent
\textbf{Exercice 1:}\\

\begin{enumerate}
\item 
On sait qu'ils sont liés si et seulement si 
$\det(u,v,w) = 0$.
Or $\det(u,v,w)=0$ donc ils sont liés. 
On cherche alors a résoudre le 
système donné par 
$au+bv+cw=0$. On trouve par exemple la solution suivante : \\
$-2u +v+w=0$.
\item 
Dire que $z$ est combinaison linéaire de $u,v$ et $w$ équivaut à dire qu'il existe 
$a,b,c \in \R $ tels que 
$au+bv+cw=z$. Cela nous donne le système suivant : 
\[ 
\begin{array}{cccc}
a&-b&+3c &=1\\
2a & +b &+3c &=1 \\
3a & +2b & +4c & =1
\end{array}
\ \Leftrightarrow \ \ 
\begin{array}{lcccc}
 & a&-b&+3c &=1\\
 L_2 \leftarrow L_2 -2L_1 &  & 3b & -3c & =-1 \\
 L_3 \leftarrow L_3-3L_1 & & 5b & -5c &=-2
\end{array}\]
Mais la deuxième ligne donne 
$b-c= \frac{-1}{3}$ et la troisième 
$b-c = \frac{-2}{5}$, donc le système n'a pas de solutions, donc $z$ n'est pas combinaison linéaire de 
$u,v,w$.
\end{enumerate}



\medskip
\noindent
\textbf{Exercice 2:}\\
\begin{enumerate}
\item 
On peut par exemple paramétrer $D_1$ par $y$, et $D_2$ par $z$ et on en déduit les équations paramétriques suivantes :
\[
D_1 :  
\begin{cases}
x &= 2+t \\
y&=t \\
z &=11+2t
\end{cases}
\hspace{2cm}
D_2 : 
\begin{cases}
x &= \lambda-1 \\
y & =2\lambda+2\\
z & = \lambda
\end{cases}\]
Dire qu'un point $M$ de l'espace est à la fois dans $D_1$ et $D_2$, équivaut à dire qu'il existe un 
paramètre $t$ tel que $M = (2+t,t,11+2t)$ (i.e. $M\in D_1$), et qu'il existe un parametre 
$\lambda$ tel que $M=(\lambda-1 , 2\lambda +2,\lambda)$. Cela amène à résoudre les solutions en 
$\lambda$ et $t$ du système 
\[ 
\begin{matrix}
2&+t&=&\lambda-1 \\
&t &= &2\lambda +2 \\
11&+2t &=&\lambda
\end{matrix} 
\hspace{10pt} \Leftrightarrow \hspace{10pt}
\begin{matrix}
t&-\lambda &=&- 3\\
t&-2\lambda &=&2 \\
2t&-\lambda &=&-11
\end{matrix}
\hspace{10pt} \Leftrightarrow \hspace{10pt}
\begin{matrix}
 &  t&-\lambda & =& -3 \\
L_2\leftarrow L_2-L_1 & &-\lambda &=&5 \\
L_3 \leftarrow L_3-2L_1 & & \lambda &=&-5
\end{matrix}
\]
Les deux dernières équations étant les mêmes, le sytème équivaut au suivant :
\[
\begin{cases}
  t-\lambda & = -3 \\
 -\lambda &=5 
\end{cases}\]
qui admet une unique solution 
$\lambda =-5$ et $t=-8$. En remplaçant cette valeur de $\lambda$ (resp. $t$) dans la 
paramétrisation de $D_1$ (resp. $D_2$), 
on obtient le même point, et on en déduit que 
$D_1\cap D_2$ est réduit à un point qui est 
$A=(-6,-8,-5)$. (En particulier, les droites sont concourantes).
\item D'après la question précédente, $D_1$ passe par $A$ et a pour vecteur directeur 
$u=\begin{pmatrix}
1\\1\\2
\end{pmatrix}$. De même $D_2$ passe par $A$ et a pour vecteur directeur 
$v=\begin{pmatrix}
1\\2\\1
\end{pmatrix} $.
Ainsi, comme ces deux droites sont concourantes, et non confondues (car leurs vecteurs directeurs 
ne sont pas colinéaires) il existe bien un unique plan qui les contient, 
$\mathcal{P}$, qui est le plan passant par $A$ de vecteurs directeurs $u,v$. 
\[u \wedge v = 
\begin{pmatrix}
1\\1\\2
\end{pmatrix}
\wedge
\begin{pmatrix}
1\\2\\1
\end{pmatrix}
=
\begin{pmatrix}
-3\\1\\1
\end{pmatrix}\]
On en déduit donc qu'une équation cartésienne de 
$\mathcal{P}$ est donnée par 
\[-3x+y+z=a\] 
pour une certaine constante $a$. En utilisant le fait que $A$ est dans $\mathcal{P}$, 
on trouve $a=5$ soit l'équation 
\[ \mathcal{P} \ : \ -3x+y+z=5\]
\end{enumerate}




\medskip
\noindent
\textbf{Exercice 3:}\\ 
Il faut juste calculer correctement les deux membres de l'égalité.


\medskip
\noindent
\textbf{Exercice 4:}\\

En complexe, on obtient 
$f(z) = iz$ et  
$g(z) = 3+i+iz$. 
Et finalement $g\circ f(z) = 3+i-z$.
Le point fixe $z_0$ de $g\circ f $ doit être la solution de l'équation $g\circ f(z_0)=z_0$.
Et après un calcul on trouve  
$z_0= \frac{3+i}{2}$. Ainsi, 
$g\circ f$ est une rotation de centre $(\frac{3}{2}, \frac{1}{2})$, et d'angle $-\pi$ (c'est en fait une symétrie centrale). 


\end{document}

\bibliographystyle{plain}
\bibliography{bibli}
