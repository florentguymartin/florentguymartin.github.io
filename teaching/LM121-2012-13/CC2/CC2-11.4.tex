\documentclass{article}

\usepackage[utf8]{inputenc}
\usepackage[francais]{babel}
\usepackage{amsmath,amssymb,amsfonts}
\usepackage[T1]{fontenc}
\usepackage{mathrsfs}

\newcommand{\R}{\mathbb{R}}
\newcommand{\Z}{\mathbb{Z}}
\newcommand{\C}{\mathbb{C}}
\newcommand{\N}{\mathbb{N}}
\newcommand{\T}{\mathbb{T}}
\newcommand{\Q}{\mathbb{Q}}
\newcommand{\D}{\mathbb{D}}
\newcommand{\K}{\mathbb{K}}
\newcommand{\Li}{\mathcal{L}}
\newcommand{\M}{\mathcal{M}}
\newcommand{\F}{\mathbf{F}}
\renewcommand{\S}{\mathbb{S}}

\newcommand{\im}{\textup{Im}}
\newcommand{\BC}{\mathcal{B}}
\newcommand{\CC}{\mathcal{C}}
\newcommand{\DC}{\mathcal{D}}
\newcommand{\can}{_{\mathrm{can}}}

\newcommand{\Mat}{\textup{Mat}}
\newcommand{\Vect}{\mathrm{Vect}}
\newcommand{\rg}{\textup{rg}}
\newcommand{\cm}{\textup{Comat}}


\begin{document}
\large
\noindent 
\textbf{Université Pierre et Marie Curie - LM 121 - 2012/2013}\\
\begin{center}
\Large 
 Contrôle continu n 2 MIME 11.4
\end{center}




\medskip
\noindent
\textbf{Exercice 1 :}\\
Soit $u=(3,-2,4)$ , $v=(-2,1,-1)$ , $w=(5,-4,10)$ et $z=(1,1,1)$. 
\begin{enumerate}
\item$u,v$ et $w$ son-ils libres? Si non, donner une combinaison linéaire non-triviale de 
$u,v,w$ qui soit nulle. 
\item $z$ est-il  combinaison linéaire de $u,v$ et $w$?
\end{enumerate}


\medskip
\noindent
\textbf{Exercice 2:}\\

Soit $\mathcal{D}$ la droite d'équation paramétrique
\[\begin{array}{ccc}
 x&=2&+t \\
y & =3&+2t \\
z & =1&-t 
\end{array} \]
et $\mathcal{D'}$ la droite d'équation paramétrique
\[ \begin{array}{ccc}
 x &=&2t \\
y & =1&-t \\
z & = 2&+2t
\end{array}\]

$\mathcal{D}$ et $\mathcal{D'}$ ont-elles des points en commun? (dit autrement, a-t-on $\mathcal{D} \cap \mathcal{D'} \ne \emptyset$?)



\medskip
\noindent
\textbf{Exercice 3:}\\

Soit $a$ et $b$ 2 vecteurs non nuls de $\mathbb{R}^3$. Pour $x\in \R^3$, on s'intéresse à l'équation $a \wedge x = b$.

\begin{enumerate}


\item
Si 
$a= \begin{pmatrix} 
     1 \\ 1 \\ -1
    \end{pmatrix}$
 et $b= \begin{pmatrix} 1 \\ 2 \\ 3 \end{pmatrix}$, montrer que l'ensemble des $x\in \mathbb{R}^3$ tels que $a\wedge x = b$ est une droite dont on donnera une paramétrisation.
\item
Si $a=\begin{pmatrix}
       1 \\ 1 \\-1
      \end{pmatrix} $ 
et $b= \begin{pmatrix} 3 \\2 \\ 1 \end{pmatrix} $, décrire l'ensemble des $x\in \mathbb{R}^3$ tels que $a\wedge x =b$.
\end{enumerate}

\medskip
\noindent
\textbf{Exercice 4:}\\
Soit 
$A=\begin{pmatrix}
1 \\1 \\1
\end{pmatrix}$, 
$B=\begin{pmatrix}
 3\\ 2\\ -2
\end{pmatrix}$
et 
$C=\begin{pmatrix}
 2\\ -1 \\ 2
\end{pmatrix}$ 3 points de $\R^3$.
Déterminer une équation cartésienne de l'unique plan $\mathcal{P}$ passant par $A,B$ et $C$.
\end{document}

\bibliographystyle{plain}
\bibliography{bibli}
