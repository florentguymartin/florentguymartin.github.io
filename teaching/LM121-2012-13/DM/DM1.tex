\documentclass{article}


\usepackage[latin1]{inputenc}
\usepackage[francais]{babel}
\usepackage{amsmath,amssymb,amsfonts}
\usepackage[T1]{fontenc}
\usepackage{enumerate}



\newtheorem{Theoreme}{Th{\'e}or{\`e}me}[section]
\newtheorem{Definition}{D{\'e}finition}[section]
\newtheorem{Prop}{Proposition}[section]
\newtheorem{Lemme}{Lemme}[section]
\title{DM  LM 121 }
\date{}
\begin{document}


\maketitle
\noindent \emph{Encadrez bien vos r�sultats.}\\
A rendre le 8 Novembre.

\begin{enumerate}
 

\item
Pour quelle(s) valeur(s) de $x\in \mathbb{R}$ les vecteurs
$u=\begin{pmatrix}
  1 \\ x \\ 1
 \end{pmatrix}$
, 
$v = \begin{pmatrix}
      1 \\ 2 \\ 3
     \end{pmatrix}
$
et 
$w= \begin{pmatrix}
     x\\
1\\
-1 
    \end{pmatrix}$
sont-ils libres?

\item

Soit 
$\mathcal{P}$ le plan passant par 
$A =(1,2,-1)$ et engendr� par les vecteurs 
$u = \begin{pmatrix}
      1 \\1 \\ -1
     \end{pmatrix} $
 et $v= \begin{pmatrix}
         2 \\ -3 \\ -1
        \end{pmatrix}$.
Soit $\mathcal{D}$ la droite passant par 
$B=(1,1,1)$ et de vecteur directeur $w=\begin{pmatrix}
                                        1 \\ 2 \\ 1
                                       \end{pmatrix}$.
D�terminer $\mathcal{P} \cap \mathcal{D}$.
 

\item
Trouver $u,v$ et $w$ tels que
$\begin{pmatrix}
1 & -1 & 1 \\
2 & 1 & 3 \\
1 & -1 & 2\end{pmatrix}
\begin{pmatrix} u & 3 & 0 \\
v & -1 & -1 \\
w & 1 & 2 \end{pmatrix}
=
\begin{pmatrix} 
-3 & 5 & 3 \\
-2 & 8 & 5 \\
-5 & 6 & 5 \end{pmatrix}
$




\item
R�soudre le syst�me
$$\begin{array}{cccl}
 x&+2y&+z&=0\\
2x&-y&+z&=4\\
3x&+y&+2z&=4\\
5x&&+3z&=8
\end{array}$$

\item

\begin{enumerate}
\item
Soit 
$M = \begin{pmatrix}
      3 & -30 \\
      0 & 2    
     \end{pmatrix}$.
Pour $n\in \mathbb{N}^*$ , on note 
$M^n = M.M \ldots M$ (n fois). \\
Trouver une formule simple pour $M^n$. \\
(indication : $ \displaystyle 3^n + 2.3^{n-1} + 2^2.3^{n-2} + \ldots + 2^{n-1}.3 + 2^n
= \sum_{i=0}^n 3^{n-i}2^i = 3^n( \sum _{i=0}^n(\frac{2}{3})^i )$ et reconna�tre alors une 
s�rie g�om�trique).

\item
On consid�re une population de poules et de renards.
Au temps $n\in \mathbb{N}$ on a $p_n$ poules et $r_n$ renards. Leur 
population �volue ainsi :\\
(1) Au temps $n+1$ on a trois fois plus de poules qu'au temps $n$ , 
mais entre temps chaque renard a mang� $30$ poules.\\
(2) Au temps $n+1$ on a deux fois plus de renards qu'au temps $n$. \\
Cela se traduit par les relations : \\
(1) $p_{n+1} = 3p_n -30r_n$ \\
(2) $r_{n+1} = 2r_n$\\
Initialement il y a 59 poules et 2 renards ($p_0 = 59$ et $r_0=2$).\\
Question : existe-t-il un moment o� les renards auront mang� toutes les poules? 
(dit autrement, existe-t-il un $n\in \mathbb{N}$ tel que $p_n \leq 0$ ). \\
indication : consid�rer la multiplication matricielle 
$M. \begin{pmatrix} p_n \\ r_n \end{pmatrix}$ et la question pr�c�dente.\\
Pour info :

\begin{tabular}{l|c|c}
 n=  &  $p_n$ & $r_n$ \\
\hline
0 &  59 & 2 \\
1 & 117&4 \\
2 & 231 & 8 \\
3 & 453 & 16 \\ 
4 & 879 & 32
\end{tabular}



\end{enumerate}




\end{enumerate}














\end{document}
