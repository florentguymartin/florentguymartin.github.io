\documentclass{article}

\usepackage[latin1]{inputenc}
\usepackage[francais]{babel}
\usepackage{amsmath,amssymb,amsfonts}
\usepackage[T1]{fontenc}
\usepackage{mathrsfs}




\begin{document}
\large
\noindent 
\textbf{Universit� Pierre et Marie Curie - LM 121 - 2012/2013}\\
\begin{center}
\Large 
Contr�le continu n� 1
\end{center}

\normalsize


\medskip
\noindent
\textbf{Exercice 1:}\\
Calculer $\cos\left(\frac{\pi}{12}\right)$ et 
$\sin \left(\frac{\pi}{12}\right)$.\\
\emph{On pourra consid�rer $\frac{\pi}{3}- \frac{\pi}{4}$.}

\medskip
\noindent
\textbf{Exercice 2:}\\
\begin{enumerate}
\item 
D�terminer les nombres complexes $w$ tels que 
$w^2 = -7-24i$.\\ 
Pour information, 
$|-7-24i| = \sqrt{ 7^2 +24^2} = \sqrt{ 49 + 576}=\sqrt{625} = 25$.
\item 
En d�duire la forme cart�sienne des �l�ments de 
\[ \mathscr{S} = \{ z\in \mathbb{C} \ \big| \ z^4 = -7-24i \}\]
 puis repr�senter les points 
de $\mathscr{S}$ dans le plan.

\end{enumerate}


\medskip
\noindent
\textbf{Exercice 3:}\\
Calculer 
\[ \int_0^{\frac{\pi}{2}} \sin^5(t)dt \]

\medskip
\noindent
\textbf{Exercice 4:}\\
Soit $P(z) = 1+z+z^2+z^3+z^4+z^5$.
\begin{enumerate}
\item 
D�terminer l'ensemble 
$\mathscr{S}$ des racines de $P$, en pr�cisant leur forme cart�sienne.
\item 
Factoriser $P$.
\item 
En d�duire que  
\[\prod_{k=1}^5 (2-e^{\frac{ik\pi}{3}}) = 63 \]
\end{enumerate}



\end{document}

\bibliographystyle{plain}
\bibliography{bibli}