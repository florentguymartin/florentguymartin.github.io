\documentclass[a4paper, 10pt]{article}


\usepackage[utf8]{inputenc}
\usepackage[francais]{babel}
\usepackage{amsmath,amssymb,amsfonts}
\usepackage[T1]{fontenc}
\usepackage{mathrsfs}

\usepackage{theorem}

%\usepackage[cp850]{inputenc}
%\usepackage[textures]{epsfig}
%\usepackage{alltt}
%\renewcommand{\baselinestretch}{1,25}
%\psfigdriver{dvips}

\pagestyle{empty}
% my margins

\addtolength{\oddsidemargin}{-.875in}
\addtolength{\evensidemargin}{-.875in}
\addtolength{\textwidth}{1.75in}

\addtolength{\topmargin}{-.875in}
\addtolength{\textheight}{1.75in}
\theoremstyle{plain}
\theorembodyfont{\upshape}
\newtheorem{thm}{Th\'eoreme}
\newtheorem{cor}[thm]{Corollaire}
\newtheorem{lem}[thm]{Lemme}
\newtheorem{prop}[thm]{Proposition}
\newtheorem{defn}[thm]{D\'efinition}
\newtheorem{rem}[thm]{Remarque}
\newtheorem{ex}{Exercice}{\theorembodyfont{\upshape}}


\newcommand{\R}{\mathbb{R}}
\newcommand{\Z}{\mathbb{Z}}
\newcommand{\C}{\mathbb{C}}
\newcommand{\N}{\mathbb{N}}
\newcommand{\T}{\mathbb{T}}
\newcommand{\Q}{\mathbb{Q}}
\newcommand{\D}{\mathbb{D}}
\newcommand{\K}{\mathbb{K}}
\newcommand{\Li}{\mathcal{L}}
\newcommand{\M}{\mathcal{M}}
\newcommand{\F}{\mathbf{F}}
\renewcommand{\S}{\mathbb{S}}

\newcommand{\im}{\textup{Im}}
\newcommand{\BC}{\mathcal{B}}
\newcommand{\CC}{\mathcal{C}}
\newcommand{\DC}{\mathcal{D}}
\newcommand{\can}{_{\mathrm{can}}}

\newcommand{\Mat}{\textup{Mat}}
\newcommand{\Vect}{\mathrm{Vect}}
\newcommand{\rg}{\textup{rg}}
\newcommand{\cm}{\textup{Comat}}

\begin{document}
\noindent
\large
\textbf{Universit\'e Pierre et Marie Curie 
 - LM121 -
Ann\'ee 2012-2013}\\

\begin{center}
\Large
\textbf{Correction Interro n$^o$ 3}
\end{center}
\normalsize

\medskip
\noindent
\textbf{Exercice 1:}\\
\begin{enumerate}
\item 
$\det (A) = -9 $.
\item 
On en déduit que $A$ est inversible.
On calcule son inverse, ici avec la méthode du pivot de Gauss : 

\begin{align*}
\begin{pmatrix}
1 & -1 & 2 \\
2 & 1 & -2 \\
3 & -2 & 1 
\end{pmatrix}
&
\left| 
\begin{pmatrix}
1 & 0&0\\
0 & 1 & 0 \\
0 & 0 & 1
\end{pmatrix}
\right. 
& \xrightarrow[L_3 \leftarrow L_3-L_1]{L_2 \leftarrow L_2 - 2L_1}
& 
\begin{pmatrix}
1 & -1 & 2 \\
0 & 3 & -6 \\
0 & 1 & -5 
\end{pmatrix}
&
\left| 
\begin{pmatrix}
1 & 0&0\\
-2 & 1 & 0 \\
-3 & 0 & 1
\end{pmatrix}
\right. 
& \xrightarrow{L_2 \leftrightarrow L_3} \\
\begin{pmatrix}
1 & -1 & 2 \\
0 & 1 & -5 \\
0 & 3 & -6 
\end{pmatrix}
&
\left| 
\begin{pmatrix}
1 & 0&0\\
-3 & 0 & 1 \\
-2 & 1 & 0
\end{pmatrix}
\right. 
& \xrightarrow[L_3 \leftarrow L_3-3L_2]{L_1 \leftarrow L_1 + L_2} 
&
\begin{pmatrix}
1 & 0 & -3 \\
0 & 1 & -5 \\
0 & 0 & 9 
\end{pmatrix}
&
\left| 
\begin{pmatrix}
-2 & 0&1\\
-3 & 0 & 1 \\
7 & 1 & -3
\end{pmatrix}
\right. 
& \xrightarrow{L_3 \leftarrow \frac{L_3}{9}}  \\
\begin{pmatrix}
1 & 0 & -3 \\
0 & 1 & -5 \\
0 & 0 & 1 
\end{pmatrix}
&
\left| 
\begin{pmatrix}
-2 & 0&1\\
-3 & 0 & 1 \\
\frac{7}{9} & \frac{1}{9} & -\frac{1}{3}
\end{pmatrix}
\right. 
& \xrightarrow[L_2 \leftarrow L_2 +5L_3]{L_1 \leftarrow L_1+3L_3} 
&
\begin{pmatrix}
1 & 0 & 0 \\
0 & 1 & 0 \\
0 & 0 & 1 
\end{pmatrix}
&
\renewcommand\arraystretch{1.2}
\left| 
\begin{pmatrix}
\frac{1}{3} & \frac{1}{3}&0\\
\frac{8}{9} & \frac{5}{9} & \frac{-2}{3} \\
\frac{7}{9} & \frac{1}{9} & -\frac{1}{3}
\end{pmatrix}
\right. 
&
\end{align*}
On vérifie qu'on a bien en effet 
$
\renewcommand\arraystretch{1.2}
A^{-1} = \begin{pmatrix}
\frac{1}{3} & \frac{1}{3}&0\\
\frac{8}{9} & \frac{5}{9} & \frac{-2}{3} \\
\frac{7}{9} & \frac{1}{9} & -\frac{1}{3}
\end{pmatrix}$ en vérifiant que  
$A.
\renewcommand\arraystretch{1.2}
\begin{pmatrix}
\frac{1}{3} & \frac{1}{3}&0\\
\frac{8}{9} & \frac{5}{9} & \frac{-2}{3} \\
\frac{7}{9} & \frac{1}{9} & -\frac{1}{3}
\end{pmatrix}
=I_3$.
\item Ce système s'écrit sous la forme 
$A\cdot 
\begin{pmatrix}
x \\ y \\z
\end{pmatrix}
= 
\begin{pmatrix}
3 \\ 6 \\ 6
\end{pmatrix}
$.
Ainsi en multipliant des deux cotés par $A^{-1}$ on en déduit que 
si $\begin{pmatrix}
x\\y\\z
\end{pmatrix}$
 est solution, alors
\[\begin{pmatrix}
x\\y\\z
\end{pmatrix}
=A^{-1}  
\begin{pmatrix}
3\\6\\6
\end{pmatrix}
=
\renewcommand\arraystretch{1.2}
\begin{pmatrix}
\frac{1}{3} & \frac{1}{3}&0\\
\frac{8}{9} & \frac{5}{9} & \frac{-2}{3} \\
\frac{7}{9} & \frac{1}{9} & -\frac{1}{3}
\end{pmatrix}
\begin{pmatrix}
3\\6\\6
\end{pmatrix}
=
\begin{pmatrix}
3 \\2\\1
\end{pmatrix}
\]
Réciproquement,
$\begin{pmatrix}
3 \\2\\1
\end{pmatrix}$ est bien une solution, ce qu'on peut voir sans calcul :
$A
\begin{pmatrix}
3 \\2\\1
\end{pmatrix}
= 
A\cdot A^{-1} 
\begin{pmatrix}
3\\6\\6
\end{pmatrix}
= \begin{pmatrix}
3\\6\\6
\end{pmatrix}  
$.
\end{enumerate}

\bigskip
\noindent
\textbf{Exercice 2:}\\
Comme $e^{\frac{i\pi}{2}}=i$, 
on trouve 
$r_1(z)= iz+2$ et $r_2(z) = iz -4i+2$.
Ainsi 
$(r_2\circ r_1)(z) = r_2(iz+2) = i(iz+2)-4i+2 = -z-2i+2$.
Pour répondre à la question, on calcule 
$(r_2\circ r_1)( 2-i) = -i$.
Ainsi l'image de $(2,-1)$ par $r_2 \circ r_1$ est $(0,-1)$.\vspace{0.4cm}\\

\noindent
\textbf{Exercice 3:}\\
\[D = 
\begin{vmatrix}
1 & 1 & 1 & \cdots & 1 \\
1 & 1 & 0 & \cdots & 0 \\
1 & 0 & 1 & \ddots & \vdots \\
\vdots & \vdots & \ddots & \ddots & 0 \\
1 & 0 & \cdots & 0 &1
\end{vmatrix}
\xrightarrow{
\footnotesize
 \begin{matrix}
 C_1 \leftarrow C_1-C_2 \\
 \vdots \\
 C_1 \leftarrow C_1 -C_n
 \end{matrix}
 }
 \begin{vmatrix}
 2-n & 1 & 1 &\cdots & 1 \\
 0 & 1 & 0 &\cdots &0 \\
 0 & 0 & 1 & \ddots & \vdots \\
 \vdots &  & \ddots & \ddots & 0 \\
 0 & \cdots  & & 0 & 1 
 \end{vmatrix} 
 =2-n
\]
Ainsi $D=2-n$.
\vspace{0.4cm}\\
\noindent
\textbf{Exercice 4:}\\
\begin{enumerate}
\item 
\[ 
C^2 = 
\begin{pmatrix}
1 & 2 & 3 \\
-1 & 2 & 0 \\
1 & -1 & 1
\end{pmatrix}
\cdot 
\begin{pmatrix}
1 & 2 & 3 \\
-1 & 2 & 0 \\
1 & -1 & 1
\end{pmatrix} 
= 
\begin{pmatrix}
2&3&6\\
-3&2&-3 \\
3&-1&4
\end{pmatrix}
\]
\[C^3 =C\cdot C^2 =
\begin{pmatrix}
1 & 2 & 3 \\
-1 & 2 & 0 \\
1 & -1 & 1
\end{pmatrix}
\cdot
\begin{pmatrix}
2&3&6\\
-3&2&-3 \\
3&-1&4
\end{pmatrix}
= 
\begin{pmatrix}
5 & 4 & 12 \\
-8 & 1 & -12 \\
8 & 0 &13
\end{pmatrix}
\]
Donc $C^3-4C^2 = 
\begin{pmatrix}
5 & 4 & 12 \\
-8 & 1 & -12 \\
8 & 0 &13
\end{pmatrix}
-4
\begin{pmatrix}
2&3&6\\
-3&2&-3 \\
3&-1&4
\end{pmatrix}
=
\begin{pmatrix}
-3 & -8 & -12 \\
4 & -7 & 0 \\
-4 &4 & -3
\end{pmatrix}$.
\item 
On cherche donc à résoudre 
\[
\begin{pmatrix}
-3 & -8 & -12 \\
4 & -7 & 0 \\
-4 &4 & -3
\end{pmatrix} + \lambda 
\begin{pmatrix}
1 & 2 & 3 \\
-1 & 2 & 0 \\
1 & -1 & 1
\end{pmatrix}
 +\mu I_3
= 
\begin{pmatrix}
-3 & -8 & -12 \\
4 & -7 & 0 \\
-4 &4 & -3
\end{pmatrix}
+
\begin{pmatrix}
\lambda + \mu & 2\lambda & 3\lambda \\
-\lambda & 2 \lambda + \mu & 0 \\
\lambda & - \lambda & \lambda + \mu 
\end{pmatrix}
=0
 \]
 Soit 
 $\begin{pmatrix}
 -3 + \lambda + \mu & -8 +2\lambda & 3 \lambda -12 \\
 4- \lambda & -7 +2\lambda +\mu & 0 \\
 -4 + \lambda & 4 - \lambda & -3 +\lambda + \mu 
 \end{pmatrix}
 = \begin{pmatrix}
 0&0&0\\
 0&0&0\\
 0&0&0
 \end{pmatrix}
 $. \\
On identifie les coefficients.
Le coefficient dans la colonne $1$ et ligne $2$ nous donne 
$4-\lambda =0$ soit $\lambda = 4$.
Le coefficient dans la colonne et ligne $1$ donne 
$-3+\lambda +\mu =0$ soit $\mu = -1$. On vérifie que cette solution, 
$(\lambda , \mu ) =(4,-1)$ marche pour les autres coefficients. 
Au passage, c'est l'unique solution, 
qui nous donne donc 
$C^3 -4C^2 +4C -I_3=0$.
\item 
On en déduit l'égalité 
$C^3-4C^2 +4C = I_3 = C\cdot (C^2-4C+4I_3)= (C^2-4C +4I_3)\cdot C$.
On en déduit que $C$ est inversible, d'inverse 
$C^{-1} = C^2-4C+4I_3 = 
\begin{pmatrix}
2 & -5 & -6 \\
1 & -2 & -3 \\
-1 & 3 & 4
\end{pmatrix}
$.
\end{enumerate}


\end{document}


